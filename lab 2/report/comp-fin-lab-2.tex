\documentclass[11pt, fleqn]{article}

\usepackage[letterpaper, margin=0.7in]{geometry}
\usepackage{graphicx}
\usepackage{subcaption}
\usepackage{cleveref}
\usepackage{listings}
\usepackage{multicol}
\usepackage{tikz}
\usepackage{bm}
\usepackage{minted}
\usetikzlibrary{arrows}
\usepackage{wrapfig}
\usepackage{amsmath}
\usepackage{amssymb}
\usepackage[section]{placeins}
\usepackage{pgfplots,wrapfig}  
\pgfplotsset{compat=newest} 
\pgfplotsset{plot coordinates/math parser=false} 
\newlength\figureheight 
\newlength\figurewidth 
\setlength{\parindent}{0pt}
\newcommand\tab[1][1cm]{\hspace*{#1}}
\usepackage{algorithm}
\usepackage[noend]{algpseudocode}

\raggedbottom

\begin{document}

\begin{center}

\Large{COMP6212 Assignment 2 : Shakib-Bin Hamid 25250094 sh3g12}

\end{center}

\section{Black-Scholes Equation Check Solution Correctness}

\subsection{}
Cumulative normal distribution function, $\mathcal{N}(x) = \frac{1}{\sqrt{2\pi}} \int_x^{-\infty} exp(\frac{-y^2}{2})dy$. $\mathcal{N}'(x) = \frac{1}{\sqrt{2\pi}}exp(\frac{-x^2}{2})$.

\subsection{}
\vspace{-1cm}
\begin{eqnarray*}
&& d_2 = d_1 - \sigma\sqrt{T-t}\\
&\implies& \frac{d_2^2}{2} = \frac{d_1^2}{2} + \frac{\sigma^2(T-t)}{2} - d_1\sigma\sqrt{T-t}\\
&\implies& \frac{d_1^2}{2} - \frac{d_2^2}{2} = d_1\sigma\sqrt{T-t} - \frac{\sigma^2(T-t)}{2} = log(\frac{S}{K}) + r(T-t)\\
&\implies& log(\frac{S}{K}) = \frac{d_1^2}{2} - \frac{d_2^2}{2} - r(T-t)\\
&\implies& \frac{S}{K} = exp(\frac{d_1^2}{2} - \frac{d_2^2}{2} - r(T-t)) = \frac{exp(\frac{-d_2^2}{2})}{exp(\frac{-d_1^2}{2} - r(T-t))} = \frac{\mathcal{N}'(d_2)}{\mathcal{N}'(d_1)exp(r(T-t))}\\
&\implies& S\mathcal{N}'(d_1) = Kexp(-r(T-t))\mathcal{N}'(d_2) 
\end{eqnarray*}

\subsection{}
\vspace{-1cm}
\begin{eqnarray*}
&& d_1 = \frac{log(\frac{S}{K})+(r+\frac{\sigma^2}{2})(T-t)}{\sigma\sqrt{(T-t)}} \\
&\implies& \frac{\partial d_1}{\partial S} = \frac{1}{\sigma\sqrt{(T-t)}} \frac{\partial}{\partial S}(logS - logK + (r+\frac{\sigma^2}{2})(T-t)) = \frac{1}{\sigma S \sqrt{(T-t)}}\\
&Similarly& d_2 = d_1 - \sigma\sqrt{T-t} \implies \frac{\partial d_2}{\partial S} = \frac{\partial d_1}{\partial S}= \frac{1}{\sigma S \sqrt{(T-t)}}
\end{eqnarray*}

\subsection{}
$c = S\mathcal{N}(d_1) - Kexp(-r(T-t))\mathcal{N}(d_2)$

\begin{eqnarray*}
\frac{\partial c}{\partial t} &=& \frac{\partial}{\partial t}[S\mathcal{N}(d_1) - Kexp(-r(T-t))\mathcal{N}(d_2)] \\
&=& S\mathcal{N}'(d_1)\frac{\partial d_1}{\partial t} - Kexp(-r(T-t))[r\mathcal{N}(d_2) + \mathcal{N}'(d_2)\frac{\partial d_2}{\partial t}]\\
&=& S\mathcal{N}'(d_1)\frac{\partial d_1}{\partial t} - Kexp(-r(T-t))[r\mathcal{N}(d_2) + \mathcal{N}'(d_2)(\frac{\partial d_1}{\partial t} + \frac{\sigma}{2\sqrt{T-t}})]\\
&=& [S\mathcal{N}'(d_1) - Kexp(-r(T-t))\mathcal{N}'(d_2)]\frac{\partial d_1}{\partial t} - Kexp(-r(T-t))[r\mathcal{N}(d_2) + \frac{\sigma}{2\sqrt{T-t}}\mathcal{N}'(d_2)]\\
&=& -rKexp(-r(T-t))\mathcal{N}(d_2) - \frac{\sigma}{2\sqrt{T-t}}S\mathcal{N}'(d_1)
\end{eqnarray*}

\begin{eqnarray*}
\frac{\partial c}{\partial S} = \mathcal{N}(d_1) + \frac{1}{\sigma S \sqrt{(T-t)}}[S\mathcal{N}'(d_1) - Kexp(-r(T-t))\mathcal{N}'(d_2)] = \mathcal{N}(d_1)
\end{eqnarray*}

\subsection{}
$
\frac{\partial c}{\partial S} = \mathcal{N}(d_1)
\implies \frac{\partial^2 c}{\partial S^2} = \mathcal{N}'(d_1)\frac{\partial d_1}{\partial S} = \mathcal{N}'(d_1)\frac{1}{\sigma S \sqrt{(T-t)}}
$

\begin{eqnarray*}
\frac{\partial c}{\partial t} &=& -rKexp(-r(T-t))\mathcal{N}(d_2) - \frac{\sigma}{2\sqrt{T-t}}S\mathcal{N}'(d_1) \\
\frac{\sigma^2S^2}{2}\frac{\partial^2 c}{\partial S^2} &=& \mathcal{N}'(d_1)\frac{1}{\sigma S \sqrt{T-t}}\times\frac{\sigma^2S^2}{2} = \frac{\sigma}{2\sqrt{T-t}}S\mathcal{N}'(d_1)\\
rS\frac{\partial c}{\partial S} &=& \mathcal{N}(d_1)rS\\
-rc &=& -r(S\mathcal{N}(d_1) - Kexp(-r(T-t))\mathcal{N}(d_2)) = -rS\mathcal{N}(d_1) + rKexp(-r(T-t))\mathcal{N}(d_2)
\end{eqnarray*}

Adding the above - $\frac{\partial c}{\partial t} + \frac{\sigma^2S^2}{2}\frac{\partial^2 c}{\partial S^2} + rS\frac{\partial c}{\partial S} - rc = 0$. So, Black-Scholes PDE is correct for European call option.

\section{Black-Scholes Option Price vs Actual Option Price}

Black-Scholes equation has a closed form solution, $c = S\mathcal{N}(d_1) - Kexp(-r(T-t))\mathcal{N}(d_2)$, given that the volatility, i.e. Standard Deviation of log returns, $u_i = log(a_i)$ is constant for an asset \cite{hull}. In this formula, interest $r$, time to maturity, $T$, current stock price of the underlying asset, $S$ are constants of the domain. Only, volatility, $\sigma$ needs to be derived from past data. As long as the volatility does not change over time until maturity, the option price will be fair.

\begin{figure}[!h]
\begin{center}
	\resizebox {0.6\textwidth} {!} {% This file was created by matlab2tikz.
%
%The latest updates can be retrieved from
%  http://www.mathworks.com/matlabcentral/fileexchange/22022-matlab2tikz-matlab2tikz
%where you can also make suggestions and rate matlab2tikz.
%
\begin{tikzpicture}

\begin{axis}[%
width=3.054in,
height=2.704in,
at={(1.186in,0.493in)},
scale only axis,
xmin=34445,
xmax=34684,
xtick={34425,34516,34608,34700},
xticklabels={{01/04},{01/07},{01/10},{01/01}},
xlabel style={font=\color{white!15!black}},
xlabel={Date (dd/mm)},
ymin=0,
ymax=391.409468813534,
ylabel style={font=\color{white!15!black}},
ylabel={Call Option Price},
axis background/.style={fill=white},
title style={font=\bfseries},
title={BLS Estmations for Call Prices},
xmajorgrids,
ymajorgrids,
legend style={legend cell align=left, align=left, draw=white!15!black}
]
\addplot [color=blue]
  table[row sep=crcr]{%
34445	293.962061458953\\
34446	324.748682213877\\
34449	296.838189955824\\
34450	313.343297110431\\
34451	335.168403948059\\
34452	316.639241695459\\
34453	311.424308074159\\
34457	286.1613208871\\
34458	257.700832636223\\
34459	291.368751308424\\
34460	291.810978866829\\
34463	283.185658871319\\
34464	315.153046441561\\
34465	311.50025793169\\
34466	319.635853914616\\
34467	301.442760355779\\
34470	293.931672569954\\
34471	303.428410594077\\
34472	296.198726962152\\
34473	301.767323439007\\
34474	303.06002261796\\
34477	283.357338969967\\
34478	266.873836392207\\
34479	197.723132587847\\
34480	205.996129188917\\
34481	157.4193463612\\
34485	164.494056048449\\
34486	136.490225512443\\
34487	168.248767063893\\
34488	181.416315443238\\
34491	191.024448777178\\
34492	191.100988700702\\
34493	217.010802755608\\
34494	206.0718450596\\
34495	226.943708298562\\
34498	195.096116892172\\
34499	214.502772719191\\
34500	218.74079544202\\
34501	203.725745551777\\
34502	196.325172366353\\
34505	155.540314001417\\
34506	128.42802147081\\
34507	141.725907327601\\
34508	130.704078351365\\
34509	82.5350776824976\\
34512	95.7992372088972\\
34513	102.62923396418\\
34514	126.808116463787\\
34515	104.966187823647\\
34519	145.519469414101\\
34520	137.965997899952\\
34521	124.795874704193\\
34522	138.549632726645\\
34523	134.320923611141\\
34526	150.627653449724\\
34527	134.636580302275\\
34528	167.061848270921\\
34529	204.813659271357\\
34530	230.863373072792\\
34533	235.396903552327\\
34534	241.449904978819\\
34535	231.289238173587\\
34536	247.565008567039\\
34537	264.138494891758\\
34540	252.640031043219\\
34541	267.174574972219\\
34542	234.104459286687\\
34543	243.523056106743\\
34544	231.513118444642\\
34548	298.190741653879\\
34549	302.021107717399\\
34550	292.875560303393\\
34551	307.758682306061\\
34554	309.134480665287\\
34555	306.749409080624\\
34556	305.751349458511\\
34557	283.485687104915\\
34558	281.187234966024\\
34561	278.483786706606\\
34562	283.929031326485\\
34563	323.062115458493\\
34564	315.181091248685\\
34565	324.102894243758\\
34568	299.909925115723\\
34569	305.380827018083\\
34570	333.092614564477\\
34571	365.467712166334\\
34572	391.409468813534\\
34576	378.537553894303\\
34577	377.568786485348\\
34578	342.705589891185\\
34579	348.235997363206\\
34582	364.731289887946\\
34583	327.911075238349\\
34584	328.45425101089\\
34585	301.153018486498\\
34586	257.324370178482\\
34589	249.804549246991\\
34590	240.524412233614\\
34591	202.110440103148\\
34592	234.841679001758\\
34593	190.482947247643\\
34596	202.736298187557\\
34597	161.36481239087\\
34598	143.70722440195\\
34599	147.13121549922\\
34600	153.470653293608\\
34603	129.720113644678\\
34604	135.302919459562\\
34605	160.952915087227\\
34606	122.241531905123\\
34607	144.353526812703\\
34610	111.179632004436\\
34611	124.229117767868\\
34612	91.7100647797881\\
34613	110.343855908519\\
34614	119.292783634955\\
34617	146.244135911824\\
34618	183.426729013297\\
34619	207.466777374842\\
34620	251.514556398958\\
34621	212.261814123851\\
34624	226.314772855595\\
34625	189.980715353769\\
34626	165.469259661966\\
34627	167.834627255183\\
34628	139.808051235151\\
34631	140.886023354141\\
34632	111.562908941456\\
34633	114.399560057906\\
34634	136.010746636874\\
34635	186.314108520286\\
34638	197.931367657316\\
34639	193.915171758437\\
34640	183.028749628047\\
34641	201.175372319612\\
34642	196.360905393753\\
34645	165.013673948633\\
34646	160.375893342238\\
34647	196.591184948724\\
34648	198.607989135759\\
34649	169.676809956839\\
34652	180.816714268955\\
34653	226.643591321096\\
34654	238.28736046312\\
34655	217.175675887243\\
34656	220.495244020486\\
34659	208.983683395102\\
34660	168.798408111374\\
34661	119.209205239748\\
34662	128.944431294196\\
34663	123.866997688892\\
34666	134.777323864897\\
34667	147.69742725683\\
34668	164.542238464865\\
34669	124.452368202157\\
34670	104.257764033056\\
34673	115.423372424258\\
34674	98.0467080795315\\
34675	95.1388833062579\\
34676	94.5852609814619\\
34677	62.9794584136584\\
34680	31.7155024922945\\
34681	34.0971333279879\\
34682	58.4851469149444\\
34683	52.1739904868059\\
34684	88.4830476170091\\
};
\addlegendentry{Estimated}

\addplot [color=red]
  table[row sep=crcr]{%
34445	292\\
34446	306\\
34449	281\\
34450	303\\
34451	309\\
34452	326\\
34453	297\\
34457	266\\
34458	243\\
34459	274\\
34460	273\\
34463	271\\
34464	301\\
34465	294\\
34466	305\\
34467	280\\
34470	294\\
34471	282\\
34472	272\\
34473	289\\
34474	274\\
34477	251\\
34478	253\\
34479	191\\
34480	199\\
34481	155\\
34485	161\\
34486	172\\
34487	170\\
34488	168\\
34491	185\\
34492	185\\
34493	207\\
34494	210\\
34495	215\\
34498	211\\
34499	210\\
34500	203\\
34501	199\\
34502	184\\
34505	170\\
34506	156\\
34507	156\\
34508	145\\
34509	132\\
34512	132\\
34513	135\\
34514	138\\
34515	157\\
34519	156\\
34520	170\\
34521	151\\
34522	163\\
34523	163\\
34526	177\\
34527	161\\
34528	188\\
34529	222\\
34530	236\\
34533	215\\
34534	233\\
34535	243\\
34536	214\\
34537	253\\
34540	245\\
34541	256\\
34542	221\\
34543	225\\
34544	226\\
34548	273\\
34549	286\\
34550	273\\
34551	268\\
34554	292\\
34555	279\\
34556	287\\
34557	264\\
34558	245\\
34561	273\\
34562	270\\
34563	292\\
34564	313\\
34565	313\\
34568	294\\
34569	299\\
34570	327\\
34571	350\\
34572	379\\
34576	359\\
34577	359\\
34578	322\\
34579	327\\
34582	344\\
34583	305\\
34584	308\\
34585	284\\
34586	241\\
34589	230\\
34590	218\\
34591	191\\
34592	229\\
34593	179\\
34596	175\\
34597	171\\
34598	152\\
34599	162\\
34600	170\\
34603	146\\
34604	149\\
34605	157\\
34606	172\\
34607	145\\
34610	154\\
34611	132\\
34612	130\\
34613	134\\
34614	147\\
34617	172\\
34618	205\\
34619	214\\
34620	257\\
34621	219\\
34624	232\\
34625	200\\
34626	176\\
34627	178\\
34628	156\\
34631	147\\
34632	160\\
34633	122\\
34634	154\\
34635	201\\
34638	209\\
34639	199\\
34640	191\\
34641	212\\
34642	209\\
34645	165\\
34646	157\\
34647	191\\
34648	188\\
34649	174\\
34652	164\\
34653	234\\
34654	233\\
34655	219\\
34656	228\\
34659	212\\
34660	187\\
34661	128\\
34662	125\\
34663	118\\
34666	143\\
34667	153\\
34668	165\\
34669	125\\
34670	95\\
34673	101\\
34674	105\\
34675	93\\
34676	82\\
34677	80\\
34680	52\\
34681	35\\
34682	35\\
34683	47\\
34684	44\\
};
\addlegendentry{Actual}

\addplot [color=blue, forget plot]
  table[row sep=crcr]{%
34445	213.525802449143\\
34446	240.394569793467\\
34449	215.520824229043\\
34450	230.381982462853\\
34451	249.880497684403\\
34452	232.927973228489\\
34453	228.434698717012\\
34457	206.240080385507\\
34458	182.409500089869\\
34459	211.040878831393\\
34460	211.282617971636\\
34463	203.364120198323\\
34464	231.431433674845\\
34465	228.610602746171\\
34466	235.683053112367\\
34467	219.509819222015\\
34470	212.898805925442\\
34471	220.041149558879\\
34472	213.498919534683\\
34473	218.281782832693\\
34474	219.40407452405\\
34477	201.909623704631\\
34478	187.981954467108\\
34479	130.561980624833\\
34480	135.385494097967\\
34481	97.5098492343477\\
34485	107.064098137439\\
34486	87.6606526244191\\
34487	110.550385869133\\
34488	119.619520594892\\
34491	126.379976212061\\
34492	127.019922764963\\
34493	148.783163055105\\
34494	139.088361594269\\
34495	156.117230774339\\
34498	130.133544119244\\
34499	144.737200985919\\
34500	147.91965840951\\
34501	135.27500721985\\
34502	128.114334111213\\
34505	96.6920577956066\\
34506	76.1630172372577\\
34507	85.3286696127609\\
34508	76.7115938522732\\
34509	41.5245341726429\\
34512	50.2593625764594\\
34513	65.7581739572838\\
34514	90.5947283193777\\
34515	75.0761522815126\\
34519	101.821959996818\\
34520	96.28380571146\\
34521	88.9757328868818\\
34522	98.1141135341518\\
34523	94.5829742500059\\
34526	105.752543469263\\
34527	95.459519930821\\
34528	118.244496148416\\
34529	145.22036852879\\
34530	164.497039143384\\
34533	171.592918711486\\
34534	176.038958651809\\
34535	167.359812663575\\
34536	179.76917145471\\
34537	193.342895245803\\
34540	183.812745098584\\
34541	195.522008921498\\
34542	168.57199762542\\
34543	176.36146238688\\
34544	166.582405792282\\
34548	220.732969518644\\
34549	225.179635849622\\
34550	217.213663590566\\
34551	229.745181961412\\
34554	230.34751156454\\
34555	228.287456855982\\
34556	227.187154036082\\
34557	207.742403978494\\
34558	205.686773707444\\
34561	204.022477689336\\
34562	208.296991373474\\
34563	238.367973069365\\
34564	231.045965864735\\
34565	238.78624897925\\
34568	217.01324420027\\
34569	221.80548505374\\
34570	245.805997139468\\
34571	274.994838678628\\
34572	298.986212208009\\
34576	286.924117022477\\
34577	285.995338802755\\
34578	253.788559777387\\
34579	259.093121906883\\
34582	273.686272407625\\
34583	240.043779671879\\
34584	240.672618652769\\
34585	216.070121435216\\
34586	178.711292411786\\
34589	173.168693004003\\
34590	164.948664302159\\
34591	134.537500918932\\
34592	154.864784868334\\
34593	116.352908465754\\
34596	126.341812065963\\
34597	94.7681266238246\\
34598	83.8746180592993\\
34599	84.0097997715029\\
34600	88.5612443010464\\
34603	71.4614399486682\\
34604	74.618434775055\\
34605	92.4895997265264\\
34606	65.4870713937946\\
34607	81.2363403787162\\
34610	60.0615135158841\\
34611	65.1302833288462\\
34612	45.4364776308478\\
34613	56.7608378952273\\
34614	62.0939053722805\\
34617	80.3338444674912\\
34618	107.826802598153\\
34619	127.99530318437\\
34620	164.424177343371\\
34621	131.899273827049\\
34624	142.683506449917\\
34625	113.097376384957\\
34626	93.7298338747614\\
34627	95.1713219922658\\
34628	74.7791175842947\\
34631	75.4498602118836\\
34632	55.5848057352366\\
34633	61.2647025145066\\
34634	78.9203110499905\\
34635	113.76546340797\\
34638	122.695364720231\\
34639	120.355694075353\\
34640	111.645115429361\\
34641	124.909752963355\\
34642	120.55091661329\\
34645	95.5504948577711\\
34646	93.2605718809234\\
34647	119.749484930498\\
34648	121.042209977149\\
34649	98.5715988419263\\
34652	104.532576365129\\
34653	140.571009465315\\
34654	150.909074605582\\
34655	132.743350420577\\
34656	135.133895354782\\
34659	123.743000683574\\
34660	91.4248117203983\\
34661	56.8026189190482\\
34662	66.2387231731905\\
34663	61.8006180434843\\
34666	66.6620823026149\\
34667	75.3730488219114\\
34668	86.3719971602782\\
34669	57.2279519598987\\
34670	45.7656227013138\\
34673	48.4360369826557\\
34674	36.8063432333047\\
34675	33.7469261239921\\
34676	32.179769134631\\
34677	19.8851070218997\\
34680	5.51210494580465\\
34681	6.81173235092137\\
34682	20.820344281678\\
34683	12.916329113224\\
34684	18.0940634904764\\
};
\addplot [color=red, forget plot]
  table[row sep=crcr]{%
34445	225\\
34446	238\\
34449	215\\
34450	235\\
34451	240\\
34452	255\\
34453	229\\
34457	201\\
34458	200\\
34459	215\\
34460	207\\
34463	205\\
34464	232\\
34465	226\\
34466	235\\
34467	213\\
34470	225\\
34471	214\\
34472	206\\
34473	220\\
34474	205\\
34477	185\\
34478	187\\
34479	160\\
34480	152\\
34481	110\\
34485	125\\
34486	120\\
34487	116\\
34488	115\\
34491	129\\
34492	145\\
34493	142\\
34494	150\\
34495	155\\
34498	150\\
34499	149\\
34500	143\\
34501	140\\
34502	127\\
34505	116\\
34506	107\\
34507	108\\
34508	96\\
34509	105\\
34512	71\\
34513	98\\
34514	100\\
34515	104\\
34519	102\\
34520	125\\
34521	112\\
34522	110\\
34523	111\\
34526	128\\
34527	120\\
34528	123\\
34529	130\\
34530	170\\
34533	160\\
34534	167\\
34535	176\\
34536	160\\
34537	184\\
34540	177\\
34541	186\\
34542	156\\
34543	160\\
34544	160\\
34548	201\\
34549	212\\
34550	200\\
34551	196\\
34554	217\\
34555	205\\
34556	205\\
34557	220\\
34558	185\\
34561	199\\
34562	196\\
34563	216\\
34564	234\\
34565	238\\
34568	220\\
34569	222\\
34570	246\\
34571	246\\
34572	293\\
34576	274\\
34577	274\\
34578	240\\
34579	244\\
34582	260\\
34583	224\\
34584	227\\
34585	205\\
34586	167\\
34589	159\\
34590	150\\
34591	126\\
34592	157\\
34593	127\\
34596	135\\
34597	110\\
34598	115\\
34599	103\\
34600	109\\
34603	103\\
34604	91\\
34605	99\\
34606	110\\
34607	89\\
34610	96\\
34611	79\\
34612	78\\
34613	78\\
34614	90\\
34617	107\\
34618	135\\
34619	142\\
34620	179\\
34621	160\\
34624	157\\
34625	130\\
34626	117\\
34627	112\\
34628	92\\
34631	85\\
34632	58\\
34633	85\\
34634	78\\
34635	106\\
34638	133\\
34639	125\\
34640	118\\
34641	125\\
34642	132\\
34645	96\\
34646	97\\
34647	116\\
34648	113\\
34649	102\\
34652	106\\
34653	143\\
34654	163\\
34655	142\\
34656	141\\
34659	128\\
34660	109\\
34661	62\\
34662	66\\
34663	53\\
34666	70\\
34667	82\\
34668	82\\
34669	52\\
34670	42\\
34673	37\\
34674	37\\
34675	45\\
34676	22\\
34677	18\\
34680	7\\
34681	1\\
34682	1\\
34683	2\\
34684	0\\
};
\addplot [color=blue, forget plot]
  table[row sep=crcr]{%
34445	153.735095459557\\
34446	174.951290889265\\
34449	154.879779573163\\
34450	166.474947362822\\
34451	181.414012898945\\
34452	167.032938769956\\
34453	163.503425516121\\
34457	146.189783408238\\
34458	126.390756577345\\
34459	150.748356771367\\
34460	151.234997418207\\
34463	144.461773458847\\
34464	167.516547110136\\
34465	167.910562856757\\
34466	173.441368992179\\
34467	160.116056509894\\
34470	154.137140948192\\
34471	158.357666643643\\
34472	152.818722368657\\
34473	156.811937324712\\
34474	157.139859763958\\
34477	143.079972256517\\
34478	132.553212855986\\
34479	88.8029995371326\\
34480	102.027016653146\\
34481	75.6322117513166\\
34485	78.1564397619627\\
34486	63.0078252677738\\
34487	84.2094504325773\\
34488	96.7425451231197\\
34491	101.050040106264\\
34492	108.874311134706\\
34493	124.044237420725\\
34494	117.634122927504\\
34495	130.909459379017\\
34498	110.703288976873\\
34499	120.545060930273\\
34500	122.687511394162\\
34501	113.31296468197\\
34502	109.141751028307\\
34505	87.3332966833968\\
34506	71.5988511656283\\
34507	77.8445392861877\\
34508	71.398435546665\\
34509	46.6580699760473\\
34512	72.2659809628105\\
34513	78.670514046958\\
34514	91.6639319095152\\
34515	80.7086536355315\\
34519	101.138797415274\\
34520	96.6537147106999\\
34521	89.0799524697936\\
34522	96.3828856916282\\
34523	91.7586776630558\\
34526	99.9418132652654\\
34527	92.0894994810362\\
34528	109.36528626449\\
34529	130.244466809813\\
34530	144.91035969977\\
34533	148.978689582579\\
34534	152.218761923136\\
34535	144.750926866983\\
34536	154.618888183267\\
34537	163.779938887423\\
34540	156.132448528757\\
34541	164.918754509498\\
34542	143.525625095803\\
34543	146.870789924541\\
34544	140.546929044057\\
34548	181.543891840429\\
34549	186.41373937644\\
34550	179.827812466114\\
34551	189.402248995602\\
34554	189.587611653582\\
34555	187.958156300157\\
34556	186.729484995615\\
34557	171.008469203739\\
34558	169.345826516218\\
34561	161.423839371663\\
34562	164.887394870069\\
34563	189.531801027623\\
34564	183.343830089743\\
34565	185.620880099446\\
34568	177.013808720247\\
34569	187.320180919451\\
34570	201.649896901457\\
34571	224.597277136922\\
34572	242.873952023363\\
34576	232.039335399103\\
34577	231.140782502874\\
34578	205.252421860108\\
34579	209.598038596539\\
34582	219.958464629076\\
34583	193.086034339139\\
34584	192.215012178284\\
34585	172.688440987968\\
34586	144.403619522143\\
34589	140.376952660046\\
34590	133.385043885854\\
34591	95.6258361676478\\
34592	111.899983492434\\
34593	86.5850624758189\\
34596	92.5829497462346\\
34597	71.6042893362694\\
34598	63.4089506408966\\
34599	66.9032725276777\\
34600	69.4130390452858\\
34603	56.6895891089296\\
34604	58.5102600381729\\
34605	70.0915797021726\\
34606	51.5810601568405\\
34607	62.0704591357414\\
34610	47.9479234071337\\
34611	51.8911223670689\\
34612	37.3725517270417\\
34613	46.2443456425842\\
34614	50.0091230113168\\
34617	60.7907439937503\\
34618	79.3399963572722\\
34619	91.2992901944165\\
34620	120.5003214108\\
34621	96.6249723255621\\
34624	102.021426082226\\
34625	81.2334350987878\\
34626	68.9936956479669\\
34627	69.8231442716076\\
34628	56.3606696115407\\
34631	58.4068522966973\\
34632	45.1315293468598\\
34633	47.8848737183833\\
34634	55.8518996244773\\
34635	79.0749695510385\\
34638	86.3589051604934\\
34639	88.0157024971443\\
34640	81.4009367697251\\
34641	90.0786755837798\\
34642	86.3479445813243\\
34645	62.1770013465016\\
34646	55.6912036070689\\
34647	73.6222800566247\\
34648	76.0681803197901\\
34649	60.5711923600027\\
34652	62.173573294046\\
34653	86.0271216284939\\
34654	96.3872595309649\\
34655	82.649657232969\\
34656	83.7358175670663\\
34659	73.5317413960031\\
34660	51.8053894261525\\
34661	31.4313362840793\\
34662	39.9369151029698\\
34663	36.5090783557478\\
34666	38.1802451778367\\
34667	43.1604906939151\\
34668	48.7836816444803\\
34669	31.3226253652323\\
34670	22.7500669346164\\
34673	28.6402603301417\\
34674	21.9057069060066\\
34675	18.8202886196016\\
34676	16.603450866314\\
34677	9.66571316349604\\
34680	2.69253521438256\\
34681	1.71688588960295\\
34682	1.89335315911569\\
34683	0.481636696884522\\
34684	0.247620347356126\\
};
\addplot [color=red, forget plot]
  table[row sep=crcr]{%
34445	165\\
34446	179\\
34449	172\\
34450	175\\
34451	180\\
34452	192\\
34453	165\\
34457	164\\
34458	140\\
34459	153\\
34460	155\\
34463	140\\
34464	173\\
34465	175\\
34466	175\\
34467	180\\
34470	166\\
34471	160\\
34472	172\\
34473	162\\
34474	148\\
34477	131\\
34478	132\\
34479	89\\
34480	100\\
34481	82\\
34485	75\\
34486	56\\
34487	75\\
34488	74\\
34491	105\\
34492	100\\
34493	115\\
34494	101\\
34495	105\\
34498	101\\
34499	101\\
34500	96\\
34501	105\\
34502	83\\
34505	75\\
34506	69\\
34507	69\\
34508	82\\
34509	42\\
34512	53\\
34513	53\\
34514	63\\
34515	66\\
34519	64\\
34520	73\\
34521	65\\
34522	70\\
34523	68\\
34526	80\\
34527	85\\
34528	90\\
34529	90\\
34530	113\\
34533	120\\
34534	115\\
34535	110\\
34536	101\\
34537	120\\
34540	121\\
34541	127\\
34542	133\\
34543	110\\
34544	107\\
34548	140\\
34549	149\\
34550	155\\
34551	135\\
34554	152\\
34555	142\\
34556	145\\
34557	132\\
34558	117\\
34561	137\\
34562	142\\
34563	150\\
34564	165\\
34565	95\\
34568	150\\
34569	153\\
34570	174\\
34571	174\\
34572	215\\
34576	197\\
34577	197\\
34578	168\\
34579	171\\
34582	184\\
34583	153\\
34584	155\\
34585	137\\
34586	106\\
34589	110\\
34590	92\\
34591	90\\
34592	80\\
34593	104\\
34596	87\\
34597	74\\
34598	55\\
34599	58\\
34600	60\\
34603	57\\
34604	50\\
34605	54\\
34606	62\\
34607	47\\
34610	52\\
34611	50\\
34612	39\\
34613	43\\
34614	47\\
34617	60\\
34618	60\\
34619	87\\
34620	88\\
34621	100\\
34624	101\\
34625	73\\
34626	62\\
34627	69\\
34628	45\\
34631	50\\
34632	35\\
34633	33\\
34634	29\\
34635	44\\
34638	72\\
34639	67\\
34640	62\\
34641	65\\
34642	67\\
34645	52\\
34646	41\\
34647	59\\
34648	55\\
34649	50\\
34652	42\\
34653	70\\
34654	65\\
34655	75\\
34656	71\\
34659	66\\
34660	50\\
34661	23\\
34662	26\\
34663	17\\
34666	25\\
34667	28\\
34668	28\\
34669	24\\
34670	8\\
34673	5.5\\
34674	5\\
34675	5\\
34676	2.5\\
34677	1\\
34680	0\\
34681	0\\
34682	0\\
34683	0\\
34684	0\\
};
\addplot [color=blue, forget plot]
  table[row sep=crcr]{%
34445	108.49194415709\\
34446	125.186191306914\\
34449	109.234490734453\\
34450	119.411776266501\\
34451	131.711092573938\\
34452	119.970524791203\\
34453	117.43562650033\\
34457	103.659838545244\\
34458	90.4193685512373\\
34459	107.490361907178\\
34460	107.576572210791\\
34463	101.912298804273\\
34464	119.194252347611\\
34465	118.465260493875\\
34466	122.666760699796\\
34467	112.110863467196\\
34470	107.416052572744\\
34471	106.374963610517\\
34472	102.095103225671\\
34473	104.507595486449\\
34474	105.029464862144\\
34477	102.386799059101\\
34478	97.9580298007461\\
34479	66.5812606952295\\
34480	66.4261032500719\\
34481	57.4019314963992\\
34485	64.9281719240436\\
34486	52.6213875081462\\
34487	66.6526206758915\\
34488	87.3326617589353\\
34491	91.112287556442\\
34492	94.934020753085\\
34493	107.225663660585\\
34494	100.533749857799\\
34495	110.281830404229\\
34498	93.8861062854148\\
34499	103.021423794416\\
34500	102.321674196403\\
34501	94.7048976447597\\
34502	90.9201100762064\\
34505	71.7122042700677\\
34506	59.5820463122916\\
34507	64.2630415376809\\
34508	59.2599213992827\\
34509	39.7281075149813\\
34512	49.4010640975662\\
34513	56.4505767437034\\
34514	66.3407619363827\\
34515	57.9329038345066\\
34519	73.3437040704823\\
34520	69.7972670088542\\
34521	63.7544295243796\\
34522	70.2104908309844\\
34523	66.659321083212\\
34526	72.4044801686786\\
34527	65.6320110336803\\
34528	79.187920642428\\
34529	99.3713764733043\\
34530	111.978554355874\\
34533	117.710643426919\\
34534	127.396914006919\\
34535	121.288993295501\\
34536	128.392642915904\\
34537	138.22728929247\\
34540	133.852782801763\\
34541	140.99108875673\\
34542	123.743692172242\\
34543	129.134157754065\\
34544	122.634865779163\\
34548	155.319196622757\\
34549	159.801251542996\\
34550	154.211337596258\\
34551	161.91974682821\\
34554	160.952175694297\\
34555	159.420199668207\\
34556	154.380661680176\\
34557	139.842254289941\\
34558	137.54129561782\\
34561	138.958028628556\\
34562	135.235519072482\\
34563	150.230813697923\\
34564	145.699863936539\\
34565	148.469933935009\\
34568	124.523785231856\\
34569	127.415258332157\\
34570	139.514139400359\\
34571	158.287951820879\\
34572	175.55639766661\\
34576	165.469249364059\\
34577	165.739399693652\\
34578	144.11096686582\\
34579	147.257422938805\\
34582	155.882479794528\\
34583	133.982332705973\\
34584	133.360115327975\\
34585	117.492330891265\\
34586	97.3701664072653\\
34589	95.7010544748869\\
34590	90.2866110807602\\
34591	68.170942895209\\
34592	78.3872020476933\\
34593	59.3266404263372\\
34596	61.1703893807369\\
34597	46.4537036781907\\
34598	40.0665844453041\\
34599	46.9428381119765\\
34600	48.2361970412442\\
34603	39.2453591972409\\
34604	45.4232130601779\\
34605	53.5978235893365\\
34606	40.0472533643475\\
34607	45.5270573465768\\
34610	35.5672735490446\\
34611	38.9528575870925\\
34612	30.1697232035426\\
34613	35.5199800721821\\
34614	37.5697133424444\\
34617	44.0952105588224\\
34618	55.2010890381672\\
34619	63.4408920706376\\
34620	88.4131493457101\\
34621	72.8384854762244\\
34624	76.6707344520505\\
34625	61.7595979471052\\
34626	55.4175760291688\\
34627	55.4711420270851\\
34628	47.2501466655084\\
34631	45.6468854317576\\
34632	35.4090768943557\\
34633	41.7863987505901\\
34634	48.7461032903437\\
34635	66.078769318364\\
34638	66.507807634307\\
34639	68.2579750241434\\
34640	63.1563314181819\\
34641	69.6817359826741\\
34642	68.7850115065\\
34645	55.7341992516004\\
34646	59.1822646052652\\
34647	71.6929493526625\\
34648	74.1211939082939\\
34649	60.9542251071613\\
34652	61.5228280005203\\
34653	77.8957805887994\\
34654	83.6380007342257\\
34655	73.2058491621119\\
34656	72.8899388811678\\
34659	63.1065385464333\\
34660	46.9421359377106\\
34661	31.3236843120236\\
34662	39.8053368585382\\
34663	36.3097065469389\\
34666	36.0126142442649\\
34667	40.9649892723339\\
34668	44.0649209816053\\
34669	30.6136086139807\\
34670	27.8021676788079\\
34673	24.9558317414275\\
34674	18.8590897697348\\
34675	15.4709363022253\\
34676	13.0826510050022\\
34677	6.86745450836332\\
34680	1.08344549338854\\
34681	0.554017835043332\\
34682	0.425042068408118\\
34683	0.0540653677429903\\
34684	0.00459744093648262\\
};
\addplot [color=red, forget plot]
  table[row sep=crcr]{%
34445	120\\
34446	129\\
34449	112\\
34450	125\\
34451	129\\
34452	139\\
34453	121\\
34457	102\\
34458	100\\
34459	106\\
34460	106\\
34463	104\\
34464	122\\
34465	125\\
34466	125\\
34467	108\\
34470	117\\
34471	109\\
34472	103\\
34473	113\\
34474	80\\
34477	102\\
34478	90\\
34479	91\\
34480	58\\
34481	39\\
34485	37\\
34486	30\\
34487	50\\
34488	50\\
34491	66\\
34492	65\\
34493	64\\
34494	65\\
34495	65\\
34498	65\\
34499	64\\
34500	60\\
34501	58\\
34502	51\\
34505	45\\
34506	42\\
34507	42\\
34508	36\\
34509	24\\
34512	33\\
34513	31\\
34514	37\\
34515	39\\
34519	38\\
34520	44\\
34521	36\\
34522	41\\
34523	41\\
34526	46\\
34527	40\\
34528	55\\
34529	66\\
34530	46\\
34533	70\\
34534	72\\
34535	78\\
34536	62\\
34537	83\\
34540	78\\
34541	83\\
34542	64\\
34543	66\\
34544	66\\
34548	91\\
34549	98\\
34550	90\\
34551	87\\
34554	100\\
34555	92\\
34556	97\\
34557	102\\
34558	73\\
34561	88\\
34562	85\\
34563	98\\
34564	110\\
34565	105\\
34568	95\\
34569	98\\
34570	104\\
34571	128\\
34572	120\\
34576	148\\
34577	144\\
34578	130\\
34579	116\\
34582	105\\
34583	118\\
34584	108\\
34585	83\\
34586	60\\
34589	55\\
34590	49\\
34591	45\\
34592	48\\
34593	47\\
34596	39\\
34597	40\\
34598	24\\
34599	27\\
34600	32\\
34603	20\\
34604	20\\
34605	23\\
34606	28\\
34607	19\\
34610	26\\
34611	15\\
34612	16\\
34613	16\\
34614	18\\
34617	22\\
34618	22\\
34619	37\\
34620	55\\
34621	47\\
34624	55\\
34625	32\\
34626	35\\
34627	24\\
34628	20\\
34631	20\\
34632	10\\
34633	13\\
34634	15\\
34635	18\\
34638	31\\
34639	27\\
34640	34\\
34641	22\\
34642	32\\
34645	15\\
34646	13\\
34647	21\\
34648	20\\
34649	15\\
34652	18\\
34653	28\\
34654	32\\
34655	28\\
34656	24\\
34659	21\\
34660	15\\
34661	5\\
34662	5.5\\
34663	3\\
34666	6\\
34667	7\\
34668	5.5\\
34669	2\\
34670	1\\
34673	0\\
34674	0\\
34675	0.5\\
34676	0\\
34677	0\\
34680	0\\
34681	0\\
34682	0\\
34683	0\\
34684	0\\
};
\addplot [color=blue, forget plot]
  table[row sep=crcr]{%
34445	83.8127090900859\\
34446	95.9784656749341\\
34449	84.2564745081731\\
34450	92.3655284124104\\
34451	101.620943619359\\
34452	92.0158282135149\\
34453	90.2208672050758\\
34457	79.810192589259\\
34458	70.2365172287703\\
34459	84.6416144006885\\
34460	88.3674913988341\\
34463	83.5781325488774\\
34464	96.9330734067041\\
34465	97.1371054189176\\
34466	100.229855163286\\
34467	92.2333013734021\\
34470	89.3609062810622\\
34471	93.2213761139838\\
34472	89.9380503916991\\
34473	91.6926787539767\\
34474	92.0907620359362\\
34477	82.9637893567515\\
34478	77.0755649064108\\
34479	53.6474231621528\\
34480	55.5073240125265\\
34481	42.6445168961566\\
34485	41.8978375731145\\
34486	33.389358122736\\
34487	42.864727141893\\
34488	46.5479703953642\\
34491	60.8534011999146\\
34492	62.8616354044341\\
34493	71.6808388638724\\
34494	68.6358693767003\\
34495	77.8404755885637\\
34498	69.8018439310127\\
34499	76.0584416416697\\
34500	77.037591193432\\
34501	70.7986745455557\\
34502	66.7481564829675\\
34505	52.3554722460698\\
34506	42.4715353458272\\
34507	46.5628821949387\\
34508	42.6796779819597\\
34509	28.2119021346506\\
34512	39.1383023803342\\
34513	49.4860988814406\\
34514	57.7841967948495\\
34515	51.2557687375248\\
34519	64.0075173059448\\
34520	60.9990364612401\\
34521	55.8312152107105\\
34522	61.2574630996512\\
34523	60.7247111537554\\
34526	65.0751542056516\\
34527	60.4611496173758\\
34528	71.9625705578711\\
34529	87.0422046921162\\
34530	100.748470835941\\
34533	101.850087312057\\
34534	103.683003122272\\
34535	98.239880307483\\
34536	103.688334479921\\
34537	111.546320882826\\
34540	105.130302394485\\
34541	110.801855947356\\
34542	96.7466215690129\\
34543	102.036749751485\\
34544	96.73267896171\\
34548	122.893409223153\\
34549	128.060899010354\\
34550	123.406720523938\\
34551	129.886837311589\\
34554	128.621322148354\\
34555	127.363347301443\\
34556	126.514786715657\\
34557	115.623036107862\\
34558	113.445201411795\\
34561	110.480795762519\\
34562	111.021280341574\\
34563	128.054938104965\\
34564	124.599892280166\\
34565	127.810217025814\\
34568	112.022337015663\\
34569	107.752135223565\\
34570	116.966466516898\\
34571	132.579712526457\\
34572	143.889871751977\\
34576	133.993957847242\\
34577	126.96501300503\\
34578	107.431731885541\\
34579	111.760725724483\\
34582	118.014873785166\\
34583	100.445949664339\\
34584	100.662630386185\\
34585	88.3319215896204\\
34586	71.9828986853531\\
34589	73.8319625132337\\
34590	68.3654804647971\\
34591	48.1697141886604\\
34592	53.1758336657648\\
34593	39.3690980026528\\
34596	54.1752010541429\\
34597	41.7605414476685\\
34598	44.1766834230918\\
34599	59.4717584334259\\
34600	59.815270417967\\
34603	49.9342169124853\\
34604	55.2942543435792\\
34605	62.1346250664205\\
34606	49.1619121814977\\
34607	55.5715314833632\\
34610	45.2022707180658\\
34611	48.0920801763391\\
34612	38.0587070916697\\
34613	43.3167153447315\\
34614	48.8470968886165\\
34617	55.707136836378\\
34618	69.7372623916888\\
34619	80.1638019016826\\
34620	98.7870090712784\\
34621	82.0269072489295\\
34624	83.8366377757296\\
34625	70.1874372017969\\
34626	59.723761885998\\
34627	67.1734433222683\\
34628	59.4883124178747\\
34631	69.1099488772177\\
34632	60.1184729229608\\
34633	87.896688522413\\
34634	96.940718339992\\
34635	149.38082514495\\
34638	156.38354382521\\
34639	151.887792505737\\
34640	145.128077601478\\
34641	150.131274008453\\
34642	146.532921024787\\
34645	126.181220304126\\
34646	129.501233722899\\
34647	142.455386487231\\
34648	146.077877551172\\
34649	131.309603389195\\
34652	127.706605837397\\
34653	145.456531478901\\
34654	170.081642599678\\
34655	157.38727082979\\
34656	155.056443760795\\
34659	138.714190209293\\
34660	118.839264779922\\
34661	102.155281668939\\
34662	101.620335645656\\
34663	95.349182284838\\
34666	87.1858231576107\\
34667	87.2016709551586\\
34668	89.9775275776711\\
34669	72.2777470359592\\
34670	60.1757311826779\\
34673	49.8256477519651\\
34674	39.8802122208263\\
34675	33.2305650311761\\
34676	28.770649080342\\
34677	18.9140875353566\\
34680	5.24775208268446\\
34681	3.11821040766367\\
34682	2.13812938645705\\
34683	0.479879579035295\\
34684	0.0468196151632734\\
};
\addplot [color=red, forget plot]
  table[row sep=crcr]{%
34445	80\\
34446	90\\
34449	77\\
34450	87\\
34451	90\\
34452	98\\
34453	83\\
34457	68\\
34458	57\\
34459	71\\
34460	71\\
34463	69\\
34464	84\\
34465	80\\
34466	85\\
34467	72\\
34470	79\\
34471	73\\
34472	68\\
34473	76\\
34474	80\\
34477	69\\
34478	55\\
34479	45\\
34480	34\\
34481	32\\
34485	30\\
34486	35\\
34487	26\\
34488	42\\
34491	32\\
34492	31\\
34493	38\\
34494	45\\
34495	30\\
34498	39\\
34499	38\\
34500	36\\
34501	35\\
34502	30\\
34505	27\\
34506	24\\
34507	24\\
34508	19\\
34509	11\\
34512	18\\
34513	16\\
34514	20\\
34515	22\\
34519	21\\
34520	25\\
34521	19\\
34522	23\\
34523	22\\
34526	27\\
34527	22\\
34528	28\\
34529	40\\
34530	46\\
34533	42\\
34534	42\\
34535	46\\
34536	35\\
34537	44\\
34540	45\\
34541	49\\
34542	35\\
34543	37\\
34544	37\\
34548	54\\
34549	60\\
34550	53\\
34551	51\\
34554	60\\
34555	54\\
34556	59\\
34557	49\\
34558	42\\
34561	52\\
34562	49\\
34563	59\\
34564	67\\
34565	66\\
34568	56\\
34569	58\\
34570	69\\
34571	69\\
34572	75\\
34576	81\\
34577	80\\
34578	62\\
34579	70\\
34582	65\\
34583	53\\
34584	60\\
34585	48\\
34586	30\\
34589	30\\
34590	25\\
34591	33\\
34592	30\\
34593	14\\
34596	16\\
34597	30\\
34598	12\\
34599	11\\
34600	12\\
34603	7\\
34604	7.5\\
34605	8.5\\
34606	11\\
34607	6.5\\
34610	7\\
34611	6\\
34612	7\\
34613	4\\
34614	5.5\\
34617	9\\
34618	14\\
34619	22\\
34620	27\\
34621	24\\
34624	22\\
34625	18\\
34626	8\\
34627	12\\
34628	4\\
34631	6\\
34632	1\\
34633	0.5\\
34634	3.5\\
34635	10\\
34638	10\\
34639	8\\
34640	7\\
34641	10\\
34642	9.5\\
34645	3\\
34646	2\\
34647	5\\
34648	4.5\\
34649	3\\
34652	1.5\\
34653	11\\
34654	8\\
34655	9\\
34656	8\\
34659	7\\
34660	2\\
34661	1.5\\
34662	0\\
34663	0\\
34666	0.5\\
34667	1\\
34668	1\\
34669	0\\
34670	0\\
34673	0\\
34674	0\\
34675	0\\
34676	0\\
34677	0\\
34680	0\\
34681	0\\
34682	0\\
34683	0\\
34684	0\\
};
\end{axis}

\begin{axis}[%
width=3.054in,
height=2.704in,
at={(5.204in,0.493in)},
scale only axis,
xmin=34445,
xmax=34684,
xtick={34425,34516,34608,34700},
xticklabels={{01/04},{01/07},{01/10},{01/01}},
xlabel style={font=\color{white!15!black}},
xlabel={Date (dd/mm)},
ymin=0,
ymax=449,
ylabel style={font=\color{white!15!black}},
ylabel={Put Option Price},
axis background/.style={fill=white},
title style={font=\bfseries},
title={BLS Estmations for Put Prices},
xmajorgrids,
ymajorgrids,
legend style={legend cell align=left, align=left, draw=white!15!black}
]
\addplot [color=blue]
  table[row sep=crcr]{%
34445	36.3258608674528\\
34446	29.0449407196863\\
34449	33.1522448453564\\
34450	30.3314640675561\\
34451	27.3392396161\\
34452	30.0096597248703\\
34453	31.272623312236\\
34457	35.2330824202504\\
34458	40.3998656682026\\
34459	33.8849634571409\\
34460	35.0209126418241\\
34463	35.6781635584126\\
34464	29.6490464789837\\
34465	28.9474440434023\\
34466	26.6282283055556\\
34467	29.6505776578294\\
34470	30.6826317643017\\
34471	25.3739263536295\\
34472	26.2219397306515\\
34473	23.7640113597005\\
34474	21.7201411446872\\
34477	23.8601049538432\\
34478	26.7742226621966\\
34479	41.1441578419583\\
34480	41.7724766100899\\
34481	56.4262739881707\\
34485	54.5968498292427\\
34486	66.2724475199184\\
34487	55.107088750462\\
34488	52.0428214131451\\
34491	44.2888702692877\\
34492	42.1372740842733\\
34493	33.2354517234121\\
34494	34.6956959373056\\
34495	29.6740119650719\\
34498	37.5525061653292\\
34499	30.7355282302987\\
34500	29.2124325280248\\
34501	32.4264909465243\\
34502	34.7095200499903\\
34505	46.3999168380074\\
34506	55.3872451748975\\
34507	50.5388581923912\\
34508	53.6268388463492\\
34509	68.5878271631725\\
34512	61.2896442772267\\
34513	57.1656504893349\\
34514	45.8896037931552\\
34515	54.4477395083534\\
34519	39.6489537921298\\
34520	42.0973441708384\\
34521	45.2666525824942\\
34522	40.1212518777464\\
34523	38.7324814342131\\
34526	31.8152164096657\\
34527	37.4610209283203\\
34528	27.4121519915718\\
34529	20.2323596624864\\
34530	15.1591329371967\\
34533	16.4782912915114\\
34534	15.0745525876751\\
34535	15.9485229436996\\
34536	15.8882548991842\\
34537	14.4224949197777\\
34540	16.7234854954542\\
34541	14.6231461869783\\
34542	19.5940536189672\\
34543	19.2443915580105\\
34544	21.0715794132257\\
34548	11.2952634967602\\
34549	13.5722504727376\\
34550	14.416699184172\\
34551	12.7294168744964\\
34554	12.0142094235585\\
34555	12.6252294046072\\
34556	12.6994244039422\\
34557	14.5536896878246\\
34558	14.8199305108873\\
34561	14.3916011315073\\
34562	16.0823142839434\\
34563	11.485232415664\\
34564	12.4918807619868\\
34565	10.786068075738\\
34568	11.9588857777991\\
34569	11.9007915331821\\
34570	9.28703933838221\\
34571	7.39999568986252\\
34572	5.80899098327873\\
34576	6.98765164307446\\
34577	6.92709625038103\\
34578	9.12409834106154\\
34579	10.2562998192618\\
34582	8.43795325122272\\
34583	11.6602893266485\\
34584	11.4704152697706\\
34585	13.5715332620552\\
34586	18.8376734588732\\
34589	22.0831424340009\\
34590	23.0321942402937\\
34591	29.1309527272064\\
34592	23.8097373971324\\
34593	32.5098593008349\\
34596	28.6640413658678\\
34597	39.6419221032477\\
34598	45.7083944839251\\
34599	44.0693570520236\\
34600	41.0673450035314\\
34603	47.2821155854641\\
34604	45.3975846982327\\
34605	36.4950633111223\\
34606	48.2779979693335\\
34607	39.6407812118008\\
34610	53.2124840969731\\
34611	47.7378689772713\\
34612	61.4137246203506\\
34613	54.0332243131419\\
34614	48.0783415588679\\
34617	37.0713203514878\\
34618	25.961508698907\\
34619	20.9883061337603\\
34620	17.397953957108\\
34621	25.6450096483183\\
34624	22.6096221943824\\
34625	29.2997357992652\\
34626	33.3463479763735\\
34627	32.7502072527803\\
34628	39.7052931030476\\
34631	38.4711220909572\\
34632	46.9137515169075\\
34633	56.4706492889331\\
34634	51.4110249966254\\
34635	37.7496813376559\\
34638	32.0541017724945\\
34639	32.7031458275666\\
34640	34.3352475640976\\
34641	29.0817201184465\\
34642	28.8479623155973\\
34645	34.249962993457\\
34646	35.1863770103301\\
34647	26.1857866534582\\
34648	26.4811122196664\\
34649	31.6034094544423\\
34652	25.0494477111361\\
34653	16.2807678994991\\
34654	19.2037583301956\\
34655	20.8343132796586\\
34656	19.3815684825162\\
34659	17.9169341901834\\
34660	24.0558267694056\\
34661	36.2780739618269\\
34662	36.6625174826231\\
34663	35.8675004616115\\
34666	27.9573635661596\\
34667	25.0361290479406\\
34668	19.7107186580743\\
34669	29.7123968792981\\
34670	34.223711105484\\
34673	23.8827210528644\\
34674	26.0530179978369\\
34675	24.3468124983553\\
34676	21.9397102884284\\
34677	29.7593085829837\\
34680	32.3809783463387\\
34681	26.7504864127052\\
34682	12.6952415844896\\
34683	9.16330777598819\\
34684	4.43366496192886\\
};
\addlegendentry{Estimated}

\addplot [color=red]
  table[row sep=crcr]{%
34445	80\\
34446	74\\
34449	82\\
34450	72\\
34451	70\\
34452	63\\
34453	72\\
34457	77\\
34458	87\\
34459	73\\
34460	73\\
34463	72\\
34464	65\\
34465	64\\
34466	59\\
34467	66\\
34470	60\\
34471	64\\
34472	67\\
34473	60\\
34474	58\\
34477	66\\
34478	75\\
34479	94\\
34480	89\\
34481	100\\
34485	106\\
34486	132\\
34487	108\\
34488	108\\
34491	100\\
34492	95\\
34493	82\\
34494	88\\
34495	77\\
34498	77\\
34499	77\\
34500	80\\
34501	90\\
34502	90\\
34505	101\\
34506	115\\
34507	114\\
34508	127\\
34509	152\\
34512	150\\
34513	140\\
34514	150\\
34515	120\\
34519	118\\
34520	107\\
34521	121\\
34522	112\\
34523	110\\
34526	98\\
34527	108\\
34528	90\\
34529	88\\
34530	68\\
34533	70\\
34534	65\\
34535	49\\
34536	60\\
34537	46\\
34540	47\\
34541	42\\
34542	52\\
34543	50\\
34544	49\\
34548	34\\
34549	31\\
34550	34\\
34551	34\\
34554	28\\
34555	30\\
34556	31\\
34557	36\\
34558	45\\
34561	32\\
34562	35\\
34563	29\\
34564	24\\
34565	24\\
34568	28\\
34569	26\\
34570	21\\
34571	18\\
34572	13\\
34576	14\\
34577	14\\
34578	19\\
34579	18\\
34582	22\\
34583	20\\
34584	19\\
34585	22\\
34586	31\\
34589	33\\
34590	36\\
34591	44\\
34592	40\\
34593	38\\
34596	50\\
34597	60\\
34598	55\\
34599	58\\
34600	54\\
34603	60\\
34604	60\\
34605	57\\
34606	49\\
34607	70\\
34610	53\\
34611	56\\
34612	75\\
34613	67\\
34614	56\\
34617	51\\
34618	49\\
34619	30\\
34620	19\\
34621	24\\
34624	21\\
34625	27\\
34626	33\\
34627	31\\
34628	40\\
34631	37\\
34632	83\\
34633	54\\
34634	38\\
34635	39\\
34638	26\\
34639	24\\
34640	26\\
34641	26\\
34642	20\\
34645	26\\
34646	31\\
34647	20\\
34648	20\\
34649	22\\
34652	23\\
34653	10\\
34654	10\\
34655	9\\
34656	9\\
34659	9\\
34660	12\\
34661	25\\
34662	20\\
34663	24\\
34666	14\\
34667	12\\
34668	6\\
34669	8\\
34670	9\\
34673	8\\
34674	8.5\\
34675	7\\
34676	9\\
34677	9\\
34680	6\\
34681	9\\
34682	8\\
34683	0.5\\
34684	0\\
};
\addlegendentry{Actual}

\addplot [color=blue, forget plot]
  table[row sep=crcr]{%
34445	50.8977912516143\\
34446	40.7542153556387\\
34449	46.3688765176736\\
34450	42.4810560924464\\
34451	37.6938931166096\\
34452	41.5881060874609\\
34453	43.4366751398688\\
34457	49.0138545657634\\
34458	56.7970717084843\\
34459	48.0807181507973\\
34460	49.9408183553827\\
34463	50.9582484988534\\
34464	42.5358610029978\\
34465	44.3728924846957\\
34466	41.4997577785107\\
34467	45.8772385644619\\
34470	47.4848354042224\\
34471	40.4003363590125\\
34472	41.7468021459746\\
34473	39.5623021773911\\
34474	38.3771072338941\\
34477	42.3454897636468\\
34478	46.34914249171\\
34479	68.946756964122\\
34480	62.1157538596299\\
34481	88.8986501873535\\
34485	87.4457756298152\\
34486	105.248289114796\\
34487	85.4295542259015\\
34488	73.9475099452452\\
34491	67.9225637846948\\
34492	68.0693972693816\\
34493	57.3022854955868\\
34494	61.3303990963195\\
34495	53.062776735002\\
34498	64.2935892948067\\
34499	53.9770618373468\\
34500	51.7692979657959\\
34501	57.9636886398038\\
34502	61.0439217630931\\
34505	79.7236316284573\\
34506	93.8080553131665\\
34507	86.3411613198357\\
34508	91.4733399349361\\
34509	117.00800404772\\
34512	105.152718398799\\
34513	99.3073511614543\\
34514	82.9601013926549\\
34515	97.4587886231791\\
34519	70.3410049452011\\
34520	74.2926752574829\\
34521	79.875055105746\\
34522	71.4165630111468\\
34523	75.0571253202613\\
34526	64.193785409851\\
34527	73.3687391272574\\
34528	56.6001170470511\\
34529	41.9349505872442\\
34530	38.8335307744599\\
34533	36.9129471794414\\
34534	34.488909124506\\
34535	38.6576022165215\\
34536	33.2781816903414\\
34537	28.6960484871726\\
34540	32.4870844778404\\
34541	28.6244030214289\\
34542	37.1493704409229\\
34543	33.921820319167\\
34544	36.9724686234508\\
34548	20.5855606997079\\
34549	23.5183440673281\\
34550	25.3738044059431\\
34551	22.7860320168016\\
34554	21.7795330037387\\
34555	22.7273583483797\\
34556	22.7655266163856\\
34557	27.6943770866529\\
34558	27.8911986814913\\
34561	29.3866389266968\\
34562	26.7383608532687\\
34563	18.9812028477879\\
34564	21.2807647522758\\
34565	20.0211993723784\\
34568	25.1949431875149\\
34569	23.9238198696287\\
34570	19.0968812460717\\
34571	14.9444575069082\\
34572	11.6198525046365\\
34576	14.3473717843689\\
34577	14.245314114399\\
34578	19.0133931918585\\
34579	19.6196521117212\\
34582	16.076943575204\\
34583	21.950056723949\\
34584	21.3948085430192\\
34585	25.3538790408026\\
34586	36.6412434346381\\
34589	39.0169029447758\\
34590	40.8197229133144\\
34591	52.4780105033559\\
34592	41.7132769238199\\
34593	56.083316831538\\
34596	50.4453119651035\\
34597	66.2387441753124\\
34598	75.7075961597372\\
34599	73.3343497836436\\
34600	68.6833635620662\\
34603	78.5311601612693\\
34604	75.5472750276044\\
34605	62.2459323871271\\
34606	81.0971728442851\\
34607	68.5553653165291\\
34610	83.0519662776135\\
34611	74.9427531531626\\
34612	95.7674928151332\\
34613	82.6983973339713\\
34614	76.7519440258525\\
34617	60.3621601039158\\
34618	47.8907443177131\\
34619	41.370844826826\\
34620	29.0725434884059\\
34621	40.7472558293157\\
34624	35.4841433527156\\
34625	46.0241162840567\\
34626	53.2376999174965\\
34627	52.2129562796406\\
34628	63.2362513384601\\
34631	61.4134037507235\\
34632	75.2297523410459\\
34633	72.7056950791323\\
34634	60.6496428320436\\
34635	39.7909285931154\\
34638	33.1162221970217\\
34639	34.4123759077801\\
34640	37.0472934611221\\
34641	29.0784704419129\\
34642	30.2590735954511\\
34645	37.3413546396432\\
34646	41.2405859494609\\
34647	28.0801329069338\\
34648	28.2333638394845\\
34649	36.5143148583651\\
34652	28.0930011681639\\
34653	16.0549589172648\\
34654	17.0655174381645\\
34655	20.2163955688131\\
34656	19.5481044979617\\
34659	19.2123871382804\\
34660	29.3082623663399\\
34661	50.8652392892852\\
34662	49.3583563028412\\
34663	49.9093233047927\\
34666	40.2759447044598\\
34667	35.3094357883976\\
34668	27.8788303745282\\
34669	41.8049636687838\\
34670	50.7488063237465\\
34673	42.2168194310943\\
34674	48.661805212437\\
34675	47.826347455195\\
34676	46.0185367355175\\
34677	66.1859309180807\\
34680	86.4187389735221\\
34681	80.7309712583115\\
34682	52.8196098039029\\
34683	54.0334997559125\\
34684	22.0427224521902\\
};
\addplot [color=red, forget plot]
  table[row sep=crcr]{%
34445	110\\
34446	102\\
34449	112\\
34450	101\\
34451	98\\
34452	88\\
34453	98\\
34457	108\\
34458	125\\
34459	104\\
34460	110\\
34463	110\\
34464	88\\
34465	92\\
34466	86\\
34467	96\\
34470	88\\
34471	93\\
34472	97\\
34473	89\\
34474	95\\
34477	97\\
34478	100\\
34479	95\\
34480	128\\
34481	150\\
34485	175\\
34486	160\\
34487	151\\
34488	152\\
34491	136\\
34492	150\\
34493	130\\
34494	125\\
34495	120\\
34498	114\\
34499	114\\
34500	130\\
34501	120\\
34502	130\\
34505	144\\
34506	163\\
34507	163\\
34508	175\\
34509	200\\
34512	214\\
34513	190\\
34514	170\\
34515	164\\
34519	163\\
34520	150\\
34521	175\\
34522	155\\
34523	155\\
34526	140\\
34527	155\\
34528	142\\
34529	104\\
34530	110\\
34533	105\\
34534	86\\
34535	85\\
34536	96\\
34537	74\\
34540	76\\
34541	70\\
34542	85\\
34543	82\\
34544	81\\
34548	58\\
34549	65\\
34550	57\\
34551	57\\
34554	48\\
34555	52\\
34556	60\\
34557	60\\
34558	75\\
34561	58\\
34562	65\\
34563	49\\
34564	42\\
34565	52\\
34568	50\\
34569	45\\
34570	37\\
34571	35\\
34572	25\\
34576	24\\
34577	27\\
34578	34\\
34579	32\\
34582	38\\
34583	35\\
34584	34\\
34585	45\\
34586	55\\
34589	55\\
34590	65\\
34591	64\\
34592	72\\
34593	70\\
34596	73\\
34597	89\\
34598	90\\
34599	94\\
34600	93\\
34603	100\\
34604	107\\
34605	97\\
34606	86\\
34607	104\\
34610	93\\
34611	97\\
34612	125\\
34613	117\\
34614	117\\
34617	77\\
34618	59\\
34619	54\\
34620	38\\
34621	43\\
34624	43\\
34625	55\\
34626	66\\
34627	64\\
34628	77\\
34631	80\\
34632	80\\
34633	81\\
34634	74\\
34635	73\\
34638	53\\
34639	49\\
34640	52\\
34641	42\\
34642	42\\
34645	60\\
34646	63\\
34647	45\\
34648	45\\
34649	49\\
34652	51\\
34653	30\\
34654	26\\
34655	20\\
34656	21\\
34659	23\\
34660	33\\
34661	60\\
34662	53\\
34663	58\\
34666	38\\
34667	35\\
34668	30\\
34669	24\\
34670	42\\
34673	40\\
34674	40\\
34675	31\\
34676	48\\
34677	56\\
34680	44\\
34681	50\\
34682	70\\
34683	40\\
34684	60\\
};
\addplot [color=blue, forget plot]
  table[row sep=crcr]{%
34445	89.926440038111\\
34446	77.5832375161244\\
34449	81.5747049357099\\
34450	74.3095528573917\\
34451	52.9848982005432\\
34452	50.3181505243705\\
34453	52.5610122693837\\
34457	60.3842403447563\\
34458	71.4771439173203\\
34459	58.4285277189449\\
34460	57.614868546334\\
34463	59.6790433560807\\
34464	49.4886108686838\\
34465	51.3212197664079\\
34466	47.6681173953541\\
34467	53.9635180124739\\
34470	56.8384807072766\\
34471	52.3060983042305\\
34472	54.3041978586924\\
34473	52.3075601841517\\
34474	52.1855354319508\\
34477	57.8519277825189\\
34478	63.9277809004743\\
34479	94.742917829418\\
34480	98.8474040852445\\
34481	128.668787640444\\
34485	128.527613688912\\
34486	151.279259936462\\
34487	124.7311236757\\
34488	112.379481164069\\
34491	103.559977957213\\
34492	103.761589664915\\
34493	88.0407730703407\\
34494	93.5064055055939\\
34495	81.5207164070771\\
34498	98.4144724717678\\
34499	85.2026795761772\\
34500	82.3162715154872\\
34501	90.1401760929423\\
34502	94.1570193455811\\
34505	120.692027626984\\
34506	142.17294829252\\
34507	131.41198136093\\
34508	139.24126626271\\
34509	179.239879856465\\
34512	164.38679006233\\
34513	156.40425987862\\
34514	133.43715359047\\
34515	153.76482820584\\
34519	116.521333848554\\
34520	122.372800078265\\
34521	131.467776258687\\
34522	119.900037595988\\
34523	122.759291051816\\
34526	107.744628311887\\
34527	120.980805386516\\
34528	96.0028589292551\\
34529	74.6102136406025\\
34530	64.0250453556464\\
34533	61.5843186685522\\
34534	58.3200615680548\\
34535	61.6639835871333\\
34536	53.7218385705232\\
34537	48.6233681142753\\
34540	53.0679985586498\\
34541	46.9383963495904\\
34542	60.493234815563\\
34543	55.1143014805548\\
34544	60.8444969944433\\
34548	34.0772802481303\\
34549	36.9552603418028\\
34550	47.5346962000831\\
34551	51.0632702581643\\
34554	49.5548015401466\\
34555	50.8664372793901\\
34556	51.1654300413927\\
34557	58.8302639600256\\
34558	59.3101565339152\\
34561	54.3378546742194\\
34562	53.7336497939243\\
34563	38.4203929675983\\
34564	40.8547574134044\\
34565	37.9859680174185\\
34568	44.2582342309928\\
34569	42.3620299748987\\
34570	34.1669690282497\\
34571	26.4680780463368\\
34572	22.8592261794419\\
34576	24.2339867817823\\
34577	24.2352649099712\\
34578	32.3297639559796\\
34579	33.1050409291371\\
34582	28.3608242036697\\
34583	38.2599492379711\\
34584	37.9020541137916\\
34585	44.0083946832289\\
34586	59.3569009291116\\
34589	65.8356648853874\\
34590	69.0880714426355\\
34591	86.3107999489544\\
34592	70.4968447205904\\
34593	92.1560073085057\\
34596	84.5343813171244\\
34597	108.281203298896\\
34598	120.97677226724\\
34599	118.519893007641\\
34600	113.027783277342\\
34603	128.473289165827\\
34604	124.484848496931\\
34605	105.429231931866\\
34606	132.899453533011\\
34607	114.55825574132\\
34610	138.576964773994\\
34611	127.023855890852\\
34612	156.357609932765\\
34613	138.856897612286\\
34614	130.626783430538\\
34617	108.178258882388\\
34618	85.2108891899784\\
34619	73.6253099321243\\
34620	53.5234355267241\\
34621	69.8664128741852\\
34624	61.5112167241762\\
34625	79.1674431004503\\
34626	91.7355763324636\\
34627	84.9917378079365\\
34628	98.3830254161137\\
34631	97.5789613437828\\
34632	120.15258817676\\
34633	117.360080386403\\
34634	99.9113672066901\\
34635	66.2190897433049\\
34638	61.2221384162444\\
34639	61.8220038872632\\
34640	67.4558839724953\\
34641	56.4061937218755\\
34642	58.5192743369112\\
34645	75.6279802774141\\
34646	79.2471687686848\\
34647	57.2019812153133\\
34648	57.443231286298\\
34649	71.6062591740085\\
34652	62.2673162492761\\
34653	38.5934962942797\\
34654	37.4034611637364\\
34655	45.0798418761624\\
34656	42.6922974290999\\
34659	44.4374453824557\\
34660	65.7462007684089\\
34661	102.977077943203\\
34662	97.3646552316936\\
34663	99.0707169937091\\
34666	86.8384736638368\\
34667	75.6751523485123\\
34668	64.3560607761863\\
34669	92.8435562934274\\
34670	110.283080060867\\
34673	96.5852162331571\\
34674	111.040552514294\\
34675	112.363882335624\\
34676	111.581243640536\\
34677	144.375646454352\\
34680	179.465171454991\\
34681	174.958041593384\\
34682	143.480200484894\\
34683	148.973529470601\\
34684	111.486561971104\\
};
\addplot [color=red, forget plot]
  table[row sep=crcr]{%
34445	149\\
34446	140\\
34449	150\\
34450	138\\
34451	134\\
34452	123\\
34453	138\\
34457	151\\
34458	155\\
34459	145\\
34460	158\\
34463	145\\
34464	126\\
34465	131\\
34466	145\\
34467	130\\
34470	126\\
34471	132\\
34472	138\\
34473	127\\
34474	126\\
34477	132\\
34478	137\\
34479	187\\
34480	179\\
34481	224\\
34485	207\\
34486	229\\
34487	207\\
34488	208\\
34491	189\\
34492	190\\
34493	169\\
34494	167\\
34495	161\\
34498	163\\
34499	163\\
34500	168\\
34501	171\\
34502	170\\
34505	200\\
34506	222\\
34507	222\\
34508	237\\
34509	285\\
34512	285\\
34513	254\\
34514	230\\
34515	224\\
34519	222\\
34520	205\\
34521	227\\
34522	212\\
34523	210\\
34526	193\\
34527	209\\
34528	181\\
34529	149\\
34530	138\\
34533	131\\
34534	131\\
34535	123\\
34536	143\\
34537	128\\
34540	119\\
34541	108\\
34542	115\\
34543	127\\
34544	126\\
34548	95\\
34549	65\\
34550	97\\
34551	95\\
34554	81\\
34555	87\\
34556	96\\
34557	98\\
34558	109\\
34561	90\\
34562	93\\
34563	81\\
34564	71\\
34565	77\\
34568	79\\
34569	80\\
34570	77\\
34571	60\\
34572	60\\
34576	56\\
34577	47\\
34578	59\\
34579	57\\
34582	68\\
34583	63\\
34584	61\\
34585	71\\
34586	97\\
34589	90\\
34590	107\\
34591	105\\
34592	115\\
34593	114\\
34596	120\\
34597	140\\
34598	158\\
34599	147\\
34600	150\\
34603	163\\
34604	158\\
34605	151\\
34606	137\\
34607	161\\
34610	148\\
34611	154\\
34612	189\\
34613	175\\
34614	154\\
34617	127\\
34618	102\\
34619	95\\
34620	96\\
34621	88\\
34624	79\\
34625	98\\
34626	97\\
34627	102\\
34628	128\\
34631	135\\
34632	152\\
34633	140\\
34634	128\\
34635	93\\
34638	84\\
34639	91\\
34640	83\\
34641	98\\
34642	82\\
34645	99\\
34646	114\\
34647	87\\
34648	89\\
34649	91\\
34652	99\\
34653	63\\
34654	52\\
34655	50\\
34656	51\\
34659	57\\
34660	74\\
34661	116\\
34662	102\\
34663	123\\
34666	113\\
34667	81\\
34668	65\\
34669	88\\
34670	105\\
34673	102\\
34674	109\\
34675	115\\
34676	127\\
34677	156\\
34680	175\\
34681	160\\
34682	144\\
34683	127\\
34684	150\\
};
\addplot [color=blue, forget plot]
  table[row sep=crcr]{%
34445	104.484622828172\\
34446	89.6991117486773\\
34449	99.90753402004\\
34450	92.0943495758547\\
34451	82.2770520386675\\
34452	90.399181977765\\
34453	93.3564309690812\\
34457	102.33371756329\\
34458	118.248356224215\\
34459	99.8961840379104\\
34460	99.3532219769404\\
34463	102.470717525232\\
34464	85.2220118910463\\
34465	88.3628882396188\\
34466	84.0504898490551\\
34467	92.8694689811216\\
34470	96.4310294194945\\
34471	85.4340619492912\\
34472	88.4978678123471\\
34473	84.4515500239972\\
34474	82.485379234347\\
34477	91.3629670320602\\
34478	100.885624899044\\
34479	145.910992113324\\
34480	144.247253620047\\
34481	184.503602679697\\
34485	182.904081524988\\
34486	211.691549899047\\
34487	177.152335765175\\
34488	161.873339827936\\
34491	152.388882983903\\
34492	152.958516761092\\
34493	130.724613698852\\
34494	138.227987932783\\
34495	121.730214205206\\
34498	145.365526715494\\
34499	128.017326768207\\
34500	123.883173616682\\
34501	135.291028386815\\
34502	141.100423803643\\
34505	178.137292234424\\
34506	205.570862025796\\
34507	191.115722976969\\
34508	203.08525188776\\
34509	260.711486243562\\
34512	242.1538879148\\
34513	233.228322271673\\
34514	203.601892239782\\
34515	228.982923770642\\
34519	182.737171435067\\
34520	190.39490499798\\
34521	202.869555284904\\
34522	186.926904668629\\
34523	192.172953321592\\
34526	173.467364473852\\
34527	190.738048287534\\
34528	157.548769295842\\
34529	125.622967848356\\
34530	109.133751018733\\
34533	105.725797757347\\
34534	101.443310728934\\
34535	108.42996398553\\
34536	96.5367451851066\\
34537	86.820968231616\\
34540	94.6994749910968\\
34541	85.6252762226236\\
34542	106.997399962393\\
34543	99.5952001708879\\
34544	107.994490732046\\
34548	66.6823187333218\\
34549	68.6346687467358\\
34550	72.7127917412606\\
34551	64.8552386975434\\
34554	63.3189477531835\\
34555	65.126617133883\\
34556	65.6665115579663\\
34557	76.1113480579845\\
34558	77.9884255260945\\
34561	72.9201993353201\\
34562	71.3454324519964\\
34563	48.7904670158869\\
34564	52.5923542218884\\
34565	49.0410226398055\\
34568	59.0399467276006\\
34569	56.4320038707301\\
34570	43.9492581978661\\
34571	33.4207580386399\\
34572	25.7019873592521\\
34576	30.8785689018753\\
34577	31.5216847812504\\
34578	42.9042645095635\\
34579	43.0864552925987\\
34582	36.5007929188398\\
34583	50.3242019561121\\
34584	52.7485414489911\\
34585	64.1065969134745\\
34586	88.053522198516\\
34589	95.2659469515272\\
34590	100.689834190393\\
34591	125.474965913463\\
34592	103.204470355688\\
34593	147.986467704274\\
34596	138.852762490094\\
34597	178.513349926866\\
34598	193.970996465177\\
34599	190.773760080214\\
34600	184.222755660178\\
34603	204.143678417126\\
34604	199.461989813183\\
34605	175.83175400713\\
34606	210.467016635997\\
34607	188.455876009068\\
34610	218.497852120709\\
34611	204.651028842457\\
34612	239.856368390565\\
34613	218.031236879265\\
34614	208.166171057181\\
34617	180.96762732413\\
34618	149.636904965909\\
34619	132.93297626966\\
34620	103.89991165363\\
34621	130.463859790656\\
34624	119.735944151923\\
34625	144.610229504142\\
34626	162.481160313429\\
34627	160.536275792445\\
34628	184.199502730618\\
34631	182.021402610865\\
34632	209.656940811202\\
34633	206.358241076925\\
34634	185.870769963506\\
34635	142.368831226892\\
34638	133.44965229574\\
34639	135.406974736145\\
34640	143.426039157131\\
34641	128.364725659345\\
34642	131.579216328656\\
34645	154.949955710341\\
34646	159.912546696894\\
34647	129.32474966368\\
34648	127.711769002014\\
34649	150.066444077359\\
34652	137.605060803699\\
34653	101.964739241038\\
34654	97.1221694840367\\
34655	110.594587861001\\
34656	107.686829067837\\
34659	113.06037083559\\
34660	144.268924131252\\
34661	191.254218916775\\
34662	182.502445429969\\
34663	186.253244783185\\
34666	172.595937640028\\
34667	159.618614319083\\
34668	142.93072254098\\
34669	179.628195106089\\
34670	200.960740932937\\
34673	187.042884886033\\
34674	205.307298580005\\
34675	207.776051124913\\
34676	207.774005222168\\
34677	243.264472804866\\
34680	279.350404894414\\
34681	274.880149106074\\
34682	243.409981150234\\
34683	248.939900299911\\
34684	211.469906584214\\
};
\addplot [color=red, forget plot]
  table[row sep=crcr]{%
34445	197\\
34446	186\\
34449	202\\
34450	185\\
34451	180\\
34452	167\\
34453	186\\
34457	203\\
34458	223\\
34459	195\\
34460	196\\
34463	196\\
34464	173\\
34465	179\\
34466	169\\
34467	185\\
34470	173\\
34471	181\\
34472	188\\
34473	175\\
34474	176\\
34477	192\\
34478	190\\
34479	249\\
34480	240\\
34481	292\\
34485	300\\
34486	298\\
34487	274\\
34488	275\\
34491	254\\
34492	255\\
34493	230\\
34494	227\\
34495	221\\
34498	223\\
34499	240\\
34500	229\\
34501	233\\
34502	248\\
34505	268\\
34506	291\\
34507	291\\
34508	308\\
34509	364\\
34512	327\\
34513	327\\
34514	301\\
34515	294\\
34519	293\\
34520	272\\
34521	298\\
34522	280\\
34523	279\\
34526	260\\
34527	278\\
34528	254\\
34529	208\\
34530	194\\
34533	210\\
34534	187\\
34535	178\\
34536	203\\
34537	169\\
34540	173\\
34541	161\\
34542	170\\
34543	184\\
34544	183\\
34548	144\\
34549	145\\
34550	143\\
34551	145\\
34554	127\\
34555	135\\
34556	134\\
34557	148\\
34558	162\\
34561	139\\
34562	142\\
34563	127\\
34564	115\\
34565	118\\
34568	123\\
34569	118\\
34570	102\\
34571	92\\
34572	74\\
34576	80\\
34577	78\\
34578	95\\
34579	94\\
34582	84\\
34583	104\\
34584	101\\
34585	115\\
34586	145\\
34589	153\\
34590	160\\
34591	186\\
34592	106\\
34593	114\\
34596	181\\
34597	206\\
34598	228\\
34599	215\\
34600	208\\
34603	233\\
34604	228\\
34605	218\\
34606	201\\
34607	231\\
34610	216\\
34611	224\\
34612	260\\
34613	250\\
34614	224\\
34617	191\\
34618	158\\
34619	150\\
34620	115\\
34621	130\\
34624	130\\
34625	155\\
34626	177\\
34627	175\\
34628	198\\
34631	207\\
34632	225\\
34633	212\\
34634	198\\
34635	152\\
34638	138\\
34639	150\\
34640	158\\
34641	137\\
34642	137\\
34645	178\\
34646	185\\
34647	149\\
34648	151\\
34649	162\\
34652	169\\
34653	111\\
34654	93\\
34655	108\\
34656	107\\
34659	118\\
34660	140\\
34661	200\\
34662	194\\
34663	206\\
34666	172\\
34667	160\\
34668	134\\
34669	183\\
34670	218\\
34673	203\\
34674	200\\
34675	205\\
34676	224\\
34677	255\\
34680	275\\
34681	281\\
34682	265\\
34683	230\\
34684	256\\
};
\addplot [color=blue, forget plot]
  table[row sep=crcr]{%
34445	150.126311941912\\
34446	130.815039253246\\
34449	147.649935843973\\
34450	137.192188666467\\
34451	123.301418510562\\
34452	134.373671062823\\
34453	138.186677346396\\
34457	154.50172753779\\
34458	174.896251936429\\
34459	150.454766005773\\
34460	148.916685951958\\
34463	153.713158069706\\
34464	131.877392487712\\
34465	133.510914812542\\
34466	127.127614595071\\
34467	139.126186316176\\
34470	144.046603909702\\
34471	131.95624050629\\
34472	136.520968289277\\
34473	131.473440170544\\
34474	129.471202653814\\
34477	142.432385523557\\
34478	155.258968090299\\
34479	214.044914911818\\
34480	210.858390676517\\
34481	260.956497179591\\
34485	257.852711450291\\
34486	291.911880510488\\
34487	252.961663243205\\
34488	237.034217662459\\
34491	223.878139079457\\
34492	224.241963730504\\
34493	197.834394444032\\
34494	208.086519210995\\
34495	187.695100583554\\
34498	218.546468768845\\
34499	197.008841960936\\
34500	192.504982939216\\
34501	207.027437146059\\
34502	214.444376715604\\
34505	259.496333208021\\
34506	291.846095821568\\
34507	275.185663464157\\
34508	289.019435568709\\
34509	353.226634547796\\
34512	332.756995010678\\
34513	322.619417210989\\
34514	289.65974346184\\
34515	318.214565976906\\
34519	266.556978532768\\
34520	275.605715259759\\
34521	291.185461244405\\
34522	273.651042024713\\
34523	279.502167530108\\
34526	258.700464231218\\
34527	278.453380280963\\
34528	240.177561869332\\
34529	201.273671152328\\
34530	177.4418199366\\
34533	173.100111209524\\
34534	167.290448651094\\
34535	176.021637234873\\
34536	160.345461036515\\
34537	146.298292310893\\
34540	156.911282455585\\
34541	144.291746580341\\
34542	173.742830224046\\
34543	165.550661799531\\
34544	176.493456949489\\
34548	119.077444760992\\
34549	119.040694113452\\
34550	125.77072168107\\
34551	114.589764523721\\
34554	112.949243015234\\
34555	115.09284156689\\
34556	115.855485118058\\
34557	132.043528173366\\
34558	134.373838693701\\
34561	131.678228194187\\
34562	128.23502050252\\
34563	96.0769236861393\\
34564	101.959098946288\\
34565	95.3877050696792\\
34568	112.242817834581\\
34569	108.711188332098\\
34570	88.7012102794947\\
34571	69.5498790669142\\
34572	55.4632642949118\\
34576	63.7598571222272\\
34577	64.8471429311048\\
34578	84.9961036351392\\
34579	84.0252640100891\\
34582	73.4951392032949\\
34583	96.5305734193685\\
34584	98.3190230958458\\
34585	116.759758650543\\
34586	151.419617520673\\
34589	159.391053071151\\
34590	166.99449282211\\
34591	201.438497070157\\
34592	171.71132488357\\
34593	214.418131635975\\
34596	203.246694587606\\
34597	245.496047222489\\
34598	265.169444921605\\
34599	261.355776336416\\
34600	253.615357073334\\
34603	279.979386214383\\
34604	274.240892221011\\
34605	244.903900564319\\
34606	288.753200838137\\
34607	261.952891489545\\
34610	300.436834501136\\
34611	284.221280492477\\
34612	326.172221035358\\
34613	301.01117349222\\
34614	289.706168623362\\
34617	257.8434600613\\
34618	218.272776914507\\
34619	195.197038748439\\
34620	154.4951814288\\
34621	190.772511056796\\
34624	177.918494293102\\
34625	212.747210644346\\
34626	237.229252235469\\
34627	234.827340249795\\
34628	264.898826169345\\
34631	263.443652049139\\
34632	296.966223900466\\
34633	293.520070632702\\
34634	269.790189818958\\
34635	216.298099814245\\
34638	204.991490729462\\
34639	208.571845056919\\
34640	219.563824659248\\
34641	200.712280678476\\
34642	205.45105448725\\
34645	237.385415622228\\
34646	242.901556145793\\
34647	204.85214502955\\
34648	202.927794773584\\
34649	232.531988836104\\
34652	220.061821019515\\
34653	173.825574048967\\
34654	163.841711624726\\
34655	183.828385617311\\
34656	180.402566012868\\
34659	191.207693795925\\
34660	232.147485294738\\
34661	285.944812335397\\
34662	276.019013915217\\
34663	281.030719535092\\
34666	267.656458562083\\
34667	253.711690188411\\
34668	235.77300665009\\
34669	277.770267522992\\
34670	299.812510746564\\
34673	286.447889503814\\
34674	304.993132849552\\
34675	307.538742863621\\
34676	307.584460388748\\
34677	343.130271140372\\
34680	379.268246080054\\
34681	374.814417256633\\
34682	343.360678222155\\
34683	348.90702899138\\
34684	311.453469579073\\
};
\addplot [color=red, forget plot]
  table[row sep=crcr]{%
34445	257\\
34446	245\\
34449	264\\
34450	244\\
34451	238\\
34452	222\\
34453	245\\
34457	266\\
34458	289\\
34459	257\\
34460	258\\
34463	258\\
34464	231\\
34465	238\\
34466	227\\
34467	246\\
34470	232\\
34471	242\\
34472	250\\
34473	235\\
34474	237\\
34477	257\\
34478	254\\
34479	322\\
34480	312\\
34481	370\\
34485	351\\
34486	378\\
34487	351\\
34488	352\\
34491	328\\
34492	329\\
34493	301\\
34494	299\\
34495	291\\
34498	294\\
34499	295\\
34500	301\\
34501	306\\
34502	323\\
34505	345\\
34506	370\\
34507	369\\
34508	388\\
34509	449\\
34512	409\\
34513	409\\
34514	380\\
34515	373\\
34519	372\\
34520	349\\
34521	378\\
34522	359\\
34523	357\\
34526	337\\
34527	358\\
34528	321\\
34529	278\\
34530	263\\
34533	255\\
34534	254\\
34535	244\\
34536	273\\
34537	233\\
34540	238\\
34541	225\\
34542	258\\
34543	253\\
34544	252\\
34548	206\\
34549	194\\
34550	205\\
34551	207\\
34554	186\\
34555	196\\
34556	194\\
34557	212\\
34558	229\\
34561	201\\
34562	205\\
34563	186\\
34564	170\\
34565	167\\
34568	182\\
34569	176\\
34570	155\\
34571	142\\
34572	118\\
34576	128\\
34577	126\\
34578	150\\
34579	146\\
34582	133\\
34583	160\\
34584	158\\
34585	176\\
34586	214\\
34589	210\\
34590	233\\
34591	262\\
34592	221\\
34593	233\\
34596	256\\
34597	286\\
34598	311\\
34599	296\\
34600	288\\
34603	318\\
34604	312\\
34605	301\\
34606	282\\
34607	316\\
34610	300\\
34611	309\\
34612	348\\
34613	328\\
34614	309\\
34617	273\\
34618	234\\
34619	222\\
34620	181\\
34621	216\\
34624	201\\
34625	234\\
34626	259\\
34627	256\\
34628	283\\
34631	295\\
34632	320\\
34633	329\\
34634	283\\
34635	231\\
34638	219\\
34639	230\\
34640	239\\
34641	214\\
34642	215\\
34645	264\\
34646	273\\
34647	232\\
34648	234\\
34649	248\\
34652	257\\
34653	186\\
34654	172\\
34655	187\\
34656	183\\
34659	245\\
34660	223\\
34661	294\\
34662	285\\
34663	303\\
34666	267\\
34667	254\\
34668	229\\
34669	281\\
34670	317\\
34673	303\\
34674	300\\
34675	310\\
34676	323\\
34677	327\\
34680	376\\
34681	391\\
34682	344\\
34683	327\\
34684	350\\
};
\end{axis}
\end{tikzpicture}% }
	\caption{Black-Scholes vs Actual Option Prices}
	\label{fig:q2-prices}
	\vspace{-0.8cm}
\end{center}
\end{figure}

\begin{algorithm}[H]
\caption{Calculating Black-Scholes Option Price}
\label{alg:black-sholes}
\begin{algorithmic}
\State $r = 0.06$ // interest rate
\For {$i \in \{1,...,3T/4\}$} // T = total days = 222
\State $idx = T/4 + i$ // current day
\State $S = stocks(idx)$ // today's stock price
\For {$j \in \{1,...,N\}$} // N = number of options = 10
\State $K = strikePrices(j)$ // strike price of the option
\State $\tau = (dates(T,j)+1 - dates(idx,j))/365$ // maturity time in years
\State $\sigma = std(log(returns(i:T/4+i-1, j)))$ // volatility as SD of log returns
\State $[p, c] = blsprice(S, K, r, \tau, \sigma)$ // matlab function for Black-Scholes price
\If {$isCallOption(j)$}
\State $estPrices(i,j) = c$
\Else
\State $estPrices(i,j) = p$
\EndIf
\EndFor
\EndFor
\end{algorithmic}
\end{algorithm}

Algorithm \ref{alg:black-sholes} describes how I calculated the option prices for the 10 options from $T/4 + 1$ to $T$ days. Then I plotted the estimated option prices against the actual traded prices in Figure \ref{fig:q2-prices}.

\begin{figure}[!h]
   \centering 
   \begin{subfigure}[b]{0.45\textwidth}
     	\resizebox {\textwidth} {!} {% This file was created by matlab2tikz.
%
%The latest updates can be retrieved from
%  http://www.mathworks.com/matlabcentral/fileexchange/22022-matlab2tikz-matlab2tikz
%where you can also make suggestions and rate matlab2tikz.
%
\definecolor{mycolor1}{rgb}{0.00000,0.44700,0.74100}%
\definecolor{mycolor2}{rgb}{0.85000,0.32500,0.09800}%
\definecolor{mycolor3}{rgb}{0.92900,0.69400,0.12500}%
\definecolor{mycolor4}{rgb}{0.49400,0.18400,0.55600}%
\definecolor{mycolor5}{rgb}{0.46600,0.67400,0.18800}%
%
\begin{tikzpicture}

\begin{axis}[%
width=3.05in,
height=2.683in,
at={(1.185in,0.494in)},
scale only axis,
xmin=34445,
xmax=34684,
xtick={34425,34516,34608,34700},
xticklabels={{01/04},{01/07},{01/10},{01/01}},
xlabel style={font=\color{white!15!black}},
xlabel={Date (dd/mm)},
ymin=0.00459744093648262,
ymax=162.081642599678,
ylabel style={font=\color{white!15!black}},
ylabel={Call Price Abs Error},
axis background/.style={fill=white},
title style={font=\bfseries},
title={Call Pricing Absolute Errors},
xmajorgrids,
ymajorgrids
]
\addplot [color=mycolor1, forget plot]
  table[row sep=crcr]{%
34445	1.96206145895349\\
34446	18.7486822138771\\
34449	15.838189955824\\
34450	10.3432971104307\\
34451	26.1684039480588\\
34452	9.36075830454138\\
34453	14.4243080741594\\
34457	20.1613208870999\\
34458	14.7008326362234\\
34459	17.3687513084242\\
34460	18.8109788668294\\
34463	12.1856588713185\\
34464	14.1530464415609\\
34465	17.5002579316897\\
34466	14.6358539146158\\
34467	21.4427603557788\\
34470	0.0683274300463381\\
34471	21.4284105940769\\
34472	24.1987269621523\\
34473	12.7673234390068\\
34474	29.0600226179599\\
34477	32.3573389699668\\
34478	13.8738363922066\\
34479	6.72313258784743\\
34480	6.99612918891671\\
34481	2.41934636119959\\
34485	3.49405604844924\\
34486	35.5097744875573\\
34487	1.75123293610704\\
34488	13.4163154432381\\
34491	6.02444877717835\\
34492	6.10098870070169\\
34493	10.010802755608\\
34494	3.92815494039996\\
34495	11.9437082985619\\
34498	15.9038831078278\\
34499	4.50277271919094\\
34500	15.7407954420205\\
34501	4.72574555177744\\
34502	12.3251723663529\\
34505	14.4596859985827\\
34506	27.5719785291897\\
34507	14.2740926723991\\
34508	14.2959216486354\\
34509	49.4649223175024\\
34512	36.2007627911028\\
34513	32.37076603582\\
34514	11.191883536213\\
34515	52.0338121763532\\
34519	10.480530585899\\
34520	32.0340021000484\\
34521	26.204125295807\\
34522	24.4503672733549\\
34523	28.6790763888594\\
34526	26.3723465502762\\
34527	26.3634196977246\\
34528	20.9381517290785\\
34529	17.1863407286428\\
34530	5.13662692720845\\
34533	20.3969035523273\\
34534	8.44990497881872\\
34535	11.7107618264131\\
34536	33.5650085670391\\
34537	11.1384948917585\\
34540	7.64003104321864\\
34541	11.1745749722186\\
34542	13.1044592866874\\
34543	18.5230561067433\\
34544	5.51311844464226\\
34548	25.1907416538788\\
34549	16.021107717399\\
34550	19.8755603033928\\
34551	39.7586823060606\\
34554	17.1344806652869\\
34555	27.749409080624\\
34556	18.7513494585105\\
34557	19.4856871049146\\
34558	36.1872349660239\\
34561	5.48378670660577\\
34562	13.9290313264846\\
34563	31.0621154584928\\
34564	2.18109124868488\\
34565	11.1028942437583\\
34568	5.90992511572313\\
34569	6.38082701808253\\
34570	6.09261456447666\\
34571	15.4677121663344\\
34572	12.4094688135342\\
34576	19.5375538943026\\
34577	18.5687864853476\\
34578	20.7055898911849\\
34579	21.235997363206\\
34582	20.7312898879459\\
34583	22.9110752383494\\
34584	20.4542510108904\\
34585	17.1530184864982\\
34586	16.3243701784818\\
34589	19.8045492469905\\
34590	22.5244122336139\\
34591	11.1104401031484\\
34592	5.84167900175817\\
34593	11.4829472476426\\
34596	27.7362981875567\\
34597	9.6351876091303\\
34598	8.29277559804996\\
34599	14.8687845007798\\
34600	16.5293467063916\\
34603	16.2798863553221\\
34604	13.6970805404376\\
34605	3.95291508722676\\
34606	49.7584680948771\\
34607	0.646473187297033\\
34610	42.8203679955641\\
34611	7.77088223213195\\
34612	38.2899352202119\\
34613	23.6561440914811\\
34614	27.7072163650446\\
34617	25.7558640881762\\
34618	21.5732709867029\\
34619	6.53322262515758\\
34620	5.48544360104233\\
34621	6.73818587614869\\
34624	5.68522714440496\\
34625	10.0192846462305\\
34626	10.5307403380339\\
34627	10.1653727448174\\
34628	16.1919487648493\\
34631	6.11397664585911\\
34632	48.4370910585435\\
34633	7.60043994209354\\
34634	17.9892533631264\\
34635	14.6858914797135\\
34638	11.0686323426839\\
34639	5.08482824156317\\
34640	7.97125037195337\\
34641	10.8246276803879\\
34642	12.639094606247\\
34645	0.0136739486329134\\
34646	3.37589334223821\\
34647	5.59118494872428\\
34648	10.607989135759\\
34649	4.32319004316105\\
34652	16.8167142689549\\
34653	7.35640867890424\\
34654	5.28736046311951\\
34655	1.82432411275749\\
34656	7.50475597951436\\
34659	3.01631660489784\\
34660	18.2015918886259\\
34661	8.79079476025208\\
34662	3.94443129419642\\
34663	5.8669976888923\\
34666	8.22267613510257\\
34667	5.30257274316955\\
34668	0.457761535134978\\
34669	0.547631797843223\\
34670	9.25776403305599\\
34673	14.4233724242576\\
34674	6.95329192046847\\
34675	2.13888330625787\\
34676	12.5852609814619\\
34677	17.0205415863416\\
34680	20.2844975077055\\
34681	0.902866672012124\\
34682	23.4851469149444\\
34683	5.17399048680591\\
34684	44.4830476170091\\
};
\addplot [color=mycolor2, forget plot]
  table[row sep=crcr]{%
34445	11.4741975508573\\
34446	2.39456979346687\\
34449	0.520824229042773\\
34450	4.61801753714735\\
34451	9.88049768440305\\
34452	22.0720267715105\\
34453	0.565301282988457\\
34457	5.24008038550664\\
34458	17.5904999101308\\
34459	3.95912116860745\\
34460	4.28261797163577\\
34463	1.63587980167677\\
34464	0.568566325154734\\
34465	2.61060274617103\\
34466	0.683053112366906\\
34467	6.50981922201481\\
34470	12.101194074558\\
34471	6.04114955887917\\
34472	7.49891953468341\\
34473	1.71821716730665\\
34474	14.40407452405\\
34477	16.9096237046315\\
34478	0.981954467108153\\
34479	29.438019375167\\
34480	16.614505902033\\
34481	12.4901507656523\\
34485	17.9359018625607\\
34486	32.3393473755809\\
34487	5.44961413086708\\
34488	4.61952059489204\\
34491	2.62002378793863\\
34492	17.9800772350366\\
34493	6.78316305510521\\
34494	10.9116384057311\\
34495	1.11723077433908\\
34498	19.8664558807557\\
34499	4.26279901408088\\
34500	4.91965840950957\\
34501	4.72499278015039\\
34502	1.11433411121334\\
34505	19.3079422043934\\
34506	30.8369827627423\\
34507	22.6713303872391\\
34508	19.2884061477268\\
34509	63.4754658273571\\
34512	20.7406374235406\\
34513	32.2418260427162\\
34514	9.40527168062226\\
34515	28.9238477184874\\
34519	0.178040003182332\\
34520	28.71619428854\\
34521	23.0242671131182\\
34522	11.8858864658482\\
34523	16.4170257499941\\
34526	22.2474565307368\\
34527	24.540480069179\\
34528	4.75550385158408\\
34529	15.2203685287895\\
34530	5.50296085661557\\
34533	11.5929187114857\\
34534	9.03895865180857\\
34535	8.64018733642479\\
34536	19.7691714547095\\
34537	9.34289524580345\\
34540	6.8127450985844\\
34541	9.52200892149767\\
34542	12.57199762542\\
34543	16.3614623868802\\
34544	6.58240579228232\\
34548	19.732969518644\\
34549	13.1796358496217\\
34550	17.2136635905658\\
34551	33.7451819614116\\
34554	13.3475115645397\\
34555	23.2874568559823\\
34556	22.1871540360817\\
34557	12.2575960215058\\
34558	20.6867737074435\\
34561	5.02247768933557\\
34562	12.2969913734737\\
34563	22.3679730693648\\
34564	2.95403413526537\\
34565	0.786248979250104\\
34568	2.98675579972996\\
34569	0.194514946260369\\
34570	0.194002860531782\\
34571	28.9948386786282\\
34572	5.98621220800942\\
34576	12.9241170224768\\
34577	11.9953388027552\\
34578	13.7885597773866\\
34579	15.0931219068825\\
34582	13.6862724076254\\
34583	16.0437796718788\\
34584	13.6726186527694\\
34585	11.0701214352157\\
34586	11.711292411786\\
34589	14.1686930040032\\
34590	14.9486643021587\\
34591	8.53750091893244\\
34592	2.13521513166597\\
34593	10.6470915342461\\
34596	8.6581879340365\\
34597	15.2318733761754\\
34598	31.1253819407007\\
34599	18.9902002284971\\
34600	20.4387556989536\\
34603	31.5385600513318\\
34604	16.381565224945\\
34605	6.51040027347358\\
34606	44.5129286062054\\
34607	7.76365962128375\\
34610	35.9384864841159\\
34611	13.8697166711538\\
34612	32.5635223691522\\
34613	21.2391621047727\\
34614	27.9060946277195\\
34617	26.6661555325088\\
34618	27.1731974018467\\
34619	14.0046968156303\\
34620	14.575822656629\\
34621	28.100726172951\\
34624	14.3164935500831\\
34625	16.9026236150426\\
34626	23.2701661252386\\
34627	16.8286780077342\\
34628	17.2208824157053\\
34631	9.55013978811644\\
34632	2.41519426476339\\
34633	23.7352974854934\\
34634	0.92031104999046\\
34635	7.76546340796995\\
34638	10.3046352797687\\
34639	4.64430592464714\\
34640	6.35488457063934\\
34641	0.0902470366454509\\
34642	11.4490833867103\\
34645	0.449505142228873\\
34646	3.73942811907659\\
34647	3.7494849304976\\
34648	8.04220997714901\\
34649	3.42840115807371\\
34652	1.46742363487056\\
34653	2.42899053468454\\
34654	12.0909253944178\\
34655	9.25664957942308\\
34656	5.86610464521846\\
34659	4.25699931642566\\
34660	17.5751882796017\\
34661	5.1973810809518\\
34662	0.238723173190465\\
34663	8.80061804348429\\
34666	3.33791769738514\\
34667	6.62695117808858\\
34668	4.37199716027817\\
34669	5.22795195989875\\
34670	3.76562270131376\\
34673	11.4360369826557\\
34674	0.193656766695312\\
34675	11.2530738760079\\
34676	10.179769134631\\
34677	1.88510702189967\\
34680	1.48789505419535\\
34681	5.81173235092137\\
34682	19.820344281678\\
34683	10.916329113224\\
34684	18.0940634904764\\
};
\addplot [color=mycolor3, forget plot]
  table[row sep=crcr]{%
34445	11.264904540443\\
34446	4.04870911073522\\
34449	17.1202204268372\\
34450	8.52505263717831\\
34451	1.41401289894475\\
34452	24.9670612300438\\
34453	1.4965744838787\\
34457	17.8102165917617\\
34458	13.6092434226548\\
34459	2.25164322863293\\
34460	3.76500258179294\\
34463	4.46177345884735\\
34464	5.48345288986434\\
34465	7.08943714324278\\
34466	1.55863100782108\\
34467	19.8839434901058\\
34470	11.8628590518083\\
34471	1.64233335635708\\
34472	19.1812776313434\\
34473	5.18806267528794\\
34474	9.13985976395838\\
34477	12.0799722565173\\
34478	0.553212855985521\\
34479	0.197000462867436\\
34480	2.02701665314612\\
34481	6.36778824868338\\
34485	3.1564397619627\\
34486	7.0078252677738\\
34487	9.20945043257734\\
34488	22.7425451231197\\
34491	3.94995989373615\\
34492	8.87431113470575\\
34493	9.04423742072458\\
34494	16.6341229275038\\
34495	25.9094593790167\\
34498	9.70328897687341\\
34499	19.5450609302729\\
34500	26.6875113941624\\
34501	8.31296468197002\\
34502	26.1417510283075\\
34505	12.3332966833968\\
34506	2.59885116562828\\
34507	8.84453928618768\\
34508	10.601564453335\\
34509	4.65806997604727\\
34512	19.2659809628105\\
34513	25.670514046958\\
34514	28.6639319095152\\
34515	14.7086536355315\\
34519	37.1387974152735\\
34520	23.6537147106999\\
34521	24.0799524697936\\
34522	26.3828856916282\\
34523	23.7586776630558\\
34526	19.9418132652654\\
34527	7.08949948103623\\
34528	19.3652862644904\\
34529	40.244466809813\\
34530	31.9103596997695\\
34533	28.9786895825789\\
34534	37.2187619231363\\
34535	34.750926866983\\
34536	53.6188881832666\\
34537	43.7799388874228\\
34540	35.1324485287571\\
34541	37.9187545094978\\
34542	10.5256250958025\\
34543	36.8707899245408\\
34544	33.5469290440574\\
34548	41.5438918404286\\
34549	37.4137393764404\\
34550	24.8278124661144\\
34551	54.4022489956017\\
34554	37.5876116535817\\
34555	45.958156300157\\
34556	41.7294849956147\\
34557	39.0084692037387\\
34558	52.3458265162176\\
34561	24.4238393716628\\
34562	22.8873948700691\\
34563	39.531801027623\\
34564	18.3438300897426\\
34565	90.6208800994464\\
34568	27.0138087202474\\
34569	34.3201809194511\\
34570	27.6498969014574\\
34571	50.5972771369215\\
34572	27.8739520233626\\
34576	35.0393353991035\\
34577	34.1407825028737\\
34578	37.2524218601079\\
34579	38.5980385965386\\
34582	35.9584646290759\\
34583	40.0860343391391\\
34584	37.2150121782843\\
34585	35.6884409879676\\
34586	38.4036195221431\\
34589	30.3769526600461\\
34590	41.3850438858535\\
34591	5.62583616764778\\
34592	31.8999834924339\\
34593	17.4149375241811\\
34596	5.58294974623459\\
34597	2.3957106637306\\
34598	8.40895064089659\\
34599	8.9032725276777\\
34600	9.41303904528581\\
34603	0.310410891070433\\
34604	8.51026003817287\\
34605	16.0915797021726\\
34606	10.4189398431595\\
34607	15.0704591357414\\
34610	4.05207659286634\\
34611	1.89112236706887\\
34612	1.6274482729583\\
34613	3.24434564258422\\
34614	3.00912301131677\\
34617	0.790743993750311\\
34618	19.3399963572722\\
34619	4.29929019441647\\
34620	32.5003214108001\\
34621	3.37502767443789\\
34624	1.02142608222584\\
34625	8.23343509878782\\
34626	6.99369564796689\\
34627	0.823144271607589\\
34628	11.3606696115407\\
34631	8.40685229669725\\
34632	10.1315293468598\\
34633	14.8848737183833\\
34634	26.8518996244773\\
34635	35.0749695510385\\
34638	14.3589051604934\\
34639	21.0157024971443\\
34640	19.4009367697251\\
34641	25.0786755837798\\
34642	19.3479445813243\\
34645	10.1770013465016\\
34646	14.6912036070689\\
34647	14.6222800566247\\
34648	21.0681803197901\\
34649	10.5711923600027\\
34652	20.173573294046\\
34653	16.0271216284939\\
34654	31.3872595309649\\
34655	7.64965723296905\\
34656	12.7358175670663\\
34659	7.5317413960031\\
34660	1.80538942615249\\
34661	8.43133628407929\\
34662	13.9369151029698\\
34663	19.5090783557478\\
34666	13.1802451778367\\
34667	15.1604906939151\\
34668	20.7836816444803\\
34669	7.32262536523228\\
34670	14.7500669346164\\
34673	23.1402603301417\\
34674	16.9057069060066\\
34675	13.8202886196016\\
34676	14.103450866314\\
34677	8.66571316349604\\
34680	2.69253521438256\\
34681	1.71688588960295\\
34682	1.89335315911569\\
34683	0.481636696884522\\
34684	0.247620347356126\\
};
\addplot [color=mycolor4, forget plot]
  table[row sep=crcr]{%
34445	11.5080558429102\\
34446	3.81380869308623\\
34449	2.76550926554705\\
34450	5.58822373349926\\
34451	2.711092573938\\
34452	19.0294752087973\\
34453	3.56437349966995\\
34457	1.65983854524438\\
34458	9.58063144876269\\
34459	1.49036190717788\\
34460	1.57657221079057\\
34463	2.08770119572728\\
34464	2.80574765238862\\
34465	6.53473950612465\\
34466	2.33323930020356\\
34467	4.11086346719617\\
34470	9.58394742725591\\
34471	2.62503638948283\\
34472	0.90489677432879\\
34473	8.49240451355081\\
34474	25.0294648621439\\
34477	0.386799059101349\\
34478	7.95802980074609\\
34479	24.4187393047705\\
34480	8.42610325007195\\
34481	18.4019314963992\\
34485	27.9281719240436\\
34486	22.6213875081462\\
34487	16.6526206758915\\
34488	37.3326617589353\\
34491	25.112287556442\\
34492	29.934020753085\\
34493	43.2256636605853\\
34494	35.5337498577992\\
34495	45.2818304042291\\
34498	28.8861062854148\\
34499	39.021423794416\\
34500	42.3216741964029\\
34501	36.7048976447597\\
34502	39.9201100762064\\
34505	26.7122042700677\\
34506	17.5820463122916\\
34507	22.2630415376809\\
34508	23.2599213992827\\
34509	15.7281075149813\\
34512	16.4010640975662\\
34513	25.4505767437034\\
34514	29.3407619363827\\
34515	18.9329038345066\\
34519	35.3437040704823\\
34520	25.7972670088542\\
34521	27.7544295243796\\
34522	29.2104908309844\\
34523	25.659321083212\\
34526	26.4044801686786\\
34527	25.6320110336803\\
34528	24.187920642428\\
34529	33.3713764733043\\
34530	65.9785543558742\\
34533	47.710643426919\\
34534	55.3969140069189\\
34535	43.2889932955013\\
34536	66.3926429159037\\
34537	55.2272892924698\\
34540	55.8527828017632\\
34541	57.9910887567303\\
34542	59.7436921722424\\
34543	63.1341577540647\\
34544	56.634865779163\\
34548	64.3191966227566\\
34549	61.8012515429964\\
34550	64.2113375962576\\
34551	74.9197468282098\\
34554	60.9521756942968\\
34555	67.420199668207\\
34556	57.3806616801762\\
34557	37.8422542899409\\
34558	64.54129561782\\
34561	50.9580286285557\\
34562	50.2355190724816\\
34563	52.2308136979232\\
34564	35.6998639365393\\
34565	43.4699339350091\\
34568	29.5237852318558\\
34569	29.4152583321572\\
34570	35.5141394003592\\
34571	30.2879518208792\\
34572	55.5563976666101\\
34576	17.4692493640589\\
34577	21.7393996936523\\
34578	14.11096686582\\
34579	31.2574229388049\\
34582	50.8824797945281\\
34583	15.9823327059728\\
34584	25.3601153279751\\
34585	34.4923308912651\\
34586	37.3701664072653\\
34589	40.7010544748869\\
34590	41.2866110807602\\
34591	23.170942895209\\
34592	30.3872020476933\\
34593	12.3266404263372\\
34596	22.1703893807369\\
34597	6.45370367819066\\
34598	16.0665844453041\\
34599	19.9428381119765\\
34600	16.2361970412442\\
34603	19.2453591972409\\
34604	25.4232130601779\\
34605	30.5978235893365\\
34606	12.0472533643475\\
34607	26.5270573465768\\
34610	9.56727354904456\\
34611	23.9528575870925\\
34612	14.1697232035426\\
34613	19.5199800721821\\
34614	19.5697133424444\\
34617	22.0952105588224\\
34618	33.2010890381672\\
34619	26.4408920706376\\
34620	33.4131493457101\\
34621	25.8384854762244\\
34624	21.6707344520505\\
34625	29.7595979471052\\
34626	20.4175760291688\\
34627	31.4711420270851\\
34628	27.2501466655084\\
34631	25.6468854317576\\
34632	25.4090768943557\\
34633	28.7863987505901\\
34634	33.7461032903437\\
34635	48.078769318364\\
34638	35.507807634307\\
34639	41.2579750241434\\
34640	29.1563314181819\\
34641	47.6817359826741\\
34642	36.7850115065\\
34645	40.7341992516004\\
34646	46.1822646052652\\
34647	50.6929493526625\\
34648	54.1211939082939\\
34649	45.9542251071613\\
34652	43.5228280005203\\
34653	49.8957805887994\\
34654	51.6380007342257\\
34655	45.2058491621119\\
34656	48.8899388811678\\
34659	42.1065385464333\\
34660	31.9421359377106\\
34661	26.3236843120236\\
34662	34.3053368585382\\
34663	33.3097065469389\\
34666	30.0126142442649\\
34667	33.9649892723339\\
34668	38.5649209816053\\
34669	28.6136086139807\\
34670	26.8021676788079\\
34673	24.9558317414275\\
34674	18.8590897697348\\
34675	14.9709363022253\\
34676	13.0826510050022\\
34677	6.86745450836332\\
34680	1.08344549338854\\
34681	0.554017835043332\\
34682	0.425042068408118\\
34683	0.0540653677429903\\
34684	0.00459744093648262\\
};
\addplot [color=mycolor5, forget plot]
  table[row sep=crcr]{%
34445	3.81270909008595\\
34446	5.97846567493411\\
34449	7.25647450817314\\
34450	5.36552841241041\\
34451	11.6209436193594\\
34452	5.98417178648515\\
34453	7.22086720507582\\
34457	11.810192589259\\
34458	13.2365172287703\\
34459	13.6416144006885\\
34460	17.3674913988341\\
34463	14.5781325488774\\
34464	12.9330734067041\\
34465	17.1371054189176\\
34466	15.2298551632862\\
34467	20.2333013734021\\
34470	10.3609062810622\\
34471	20.2213761139838\\
34472	21.9380503916991\\
34473	15.6926787539767\\
34474	12.0907620359362\\
34477	13.9637893567515\\
34478	22.0755649064108\\
34479	8.64742316215279\\
34480	21.5073240125265\\
34481	10.6445168961566\\
34485	11.8978375731145\\
34486	1.61064187726402\\
34487	16.864727141893\\
34488	4.54797039536425\\
34491	28.8534011999146\\
34492	31.8616354044341\\
34493	33.6808388638724\\
34494	23.6358693767003\\
34495	47.8404755885637\\
34498	30.8018439310127\\
34499	38.0584416416697\\
34500	41.037591193432\\
34501	35.7986745455557\\
34502	36.7481564829675\\
34505	25.3554722460698\\
34506	18.4715353458272\\
34507	22.5628821949387\\
34508	23.6796779819597\\
34509	17.2119021346506\\
34512	21.1383023803342\\
34513	33.4860988814406\\
34514	37.7841967948495\\
34515	29.2557687375248\\
34519	43.0075173059448\\
34520	35.9990364612401\\
34521	36.8312152107105\\
34522	38.2574630996512\\
34523	38.7247111537554\\
34526	38.0751542056516\\
34527	38.4611496173758\\
34528	43.9625705578711\\
34529	47.0422046921162\\
34530	54.7484708359411\\
34533	59.8500873120574\\
34534	61.6830031222717\\
34535	52.239880307483\\
34536	68.6883344799205\\
34537	67.5463208828255\\
34540	60.1303023944849\\
34541	61.8018559473562\\
34542	61.7466215690129\\
34543	65.0367497514851\\
34544	59.73267896171\\
34548	68.8934092231525\\
34549	68.0608990103544\\
34550	70.4067205239376\\
34551	78.8868373115888\\
34554	68.6213221483545\\
34555	73.3633473014427\\
34556	67.5147867156566\\
34557	66.6230361078622\\
34558	71.4452014117953\\
34561	58.4807957625192\\
34562	62.0212803415739\\
34563	69.054938104965\\
34564	57.5998922801664\\
34565	61.8102170258142\\
34568	56.0223370156632\\
34569	49.7521352235653\\
34570	47.966466516898\\
34571	63.5797125264571\\
34572	68.8898717519767\\
34576	52.9939578472422\\
34577	46.9650130050302\\
34578	45.431731885541\\
34579	41.7607257244831\\
34582	53.0148737851659\\
34583	47.445949664339\\
34584	40.6626303861854\\
34585	40.3319215896204\\
34586	41.9828986853531\\
34589	43.8319625132337\\
34590	43.3654804647971\\
34591	15.1697141886604\\
34592	23.1758336657648\\
34593	25.3690980026528\\
34596	38.1752010541429\\
34597	11.7605414476685\\
34598	32.1766834230918\\
34599	48.4717584334259\\
34600	47.815270417967\\
34603	42.9342169124853\\
34604	47.7942543435792\\
34605	53.6346250664205\\
34606	38.1619121814977\\
34607	49.0715314833632\\
34610	38.2022707180658\\
34611	42.0920801763391\\
34612	31.0587070916697\\
34613	39.3167153447315\\
34614	43.3470968886165\\
34617	46.707136836378\\
34618	55.7372623916888\\
34619	58.1638019016826\\
34620	71.7870090712784\\
34621	58.0269072489295\\
34624	61.8366377757296\\
34625	52.1874372017969\\
34626	51.723761885998\\
34627	55.1734433222683\\
34628	55.4883124178747\\
34631	63.1099488772177\\
34632	59.1184729229608\\
34633	87.396688522413\\
34634	93.440718339992\\
34635	139.38082514495\\
34638	146.38354382521\\
34639	143.887792505737\\
34640	138.128077601478\\
34641	140.131274008453\\
34642	137.032921024787\\
34645	123.181220304126\\
34646	127.501233722899\\
34647	137.455386487231\\
34648	141.577877551172\\
34649	128.309603389195\\
34652	126.206605837397\\
34653	134.456531478901\\
34654	162.081642599678\\
34655	148.38727082979\\
34656	147.056443760795\\
34659	131.714190209293\\
34660	116.839264779922\\
34661	100.655281668939\\
34662	101.620335645656\\
34663	95.349182284838\\
34666	86.6858231576107\\
34667	86.2016709551586\\
34668	88.9775275776711\\
34669	72.2777470359592\\
34670	60.1757311826779\\
34673	49.8256477519651\\
34674	39.8802122208263\\
34675	33.2305650311761\\
34676	28.770649080342\\
34677	18.9140875353566\\
34680	5.24775208268446\\
34681	3.11821040766367\\
34682	2.13812938645705\\
34683	0.479879579035295\\
34684	0.0468196151632734\\
};
\end{axis}

\begin{axis}[%
width=3.05in,
height=2.683in,
at={(5.198in,0.494in)},
scale only axis,
xmin=34445,
xmax=34684,
xtick={34425,34516,34608,34700},
xticklabels={{01/04},{01/07},{01/10},{01/01}},
xlabel style={font=\color{white!15!black}},
xlabel={Date (dd/mm)},
ymin=0.212484096973071,
ymax=120.61320993767,
ylabel style={font=\color{white!15!black}},
ylabel={Put Price Abs Error},
axis background/.style={fill=white},
title style={font=\bfseries},
title={Put Pricing Absolute Errors},
xmajorgrids,
ymajorgrids
]
\addplot [color=mycolor1, forget plot]
  table[row sep=crcr]{%
34445	43.6741391325472\\
34446	44.9550592803137\\
34449	48.8477551546436\\
34450	41.6685359324439\\
34451	42.6607603839\\
34452	32.9903402751297\\
34453	40.727376687764\\
34457	41.7669175797496\\
34458	46.6001343317974\\
34459	39.1150365428591\\
34460	37.9790873581759\\
34463	36.3218364415874\\
34464	35.3509535210163\\
34465	35.0525559565977\\
34466	32.3717716944444\\
34467	36.3494223421706\\
34470	29.3173682356983\\
34471	38.6260736463705\\
34472	40.7780602693485\\
34473	36.2359886402995\\
34474	36.2798588553128\\
34477	42.1398950461568\\
34478	48.2257773378034\\
34479	52.8558421580417\\
34480	47.2275233899101\\
34481	43.5737260118293\\
34485	51.4031501707573\\
34486	65.7275524800816\\
34487	52.892911249538\\
34488	55.9571785868549\\
34491	55.7111297307123\\
34492	52.8627259157267\\
34493	48.7645482765879\\
34494	53.3043040626944\\
34495	47.3259880349281\\
34498	39.4474938346708\\
34499	46.2644717697013\\
34500	50.7875674719752\\
34501	57.5735090534757\\
34502	55.2904799500097\\
34505	54.6000831619926\\
34506	59.6127548251025\\
34507	63.4611418076088\\
34508	73.3731611536508\\
34509	83.4121728368275\\
34512	88.7103557227733\\
34513	82.8343495106651\\
34514	104.110396206845\\
34515	65.5522604916466\\
34519	78.3510462078702\\
34520	64.9026558291616\\
34521	75.7333474175058\\
34522	71.8787481222536\\
34523	71.2675185657869\\
34526	66.1847835903343\\
34527	70.5389790716797\\
34528	62.5878480084282\\
34529	67.7676403375136\\
34530	52.8408670628033\\
34533	53.5217087084886\\
34534	49.9254474123249\\
34535	33.0514770563004\\
34536	44.1117451008158\\
34537	31.5775050802223\\
34540	30.2765145045458\\
34541	27.3768538130217\\
34542	32.4059463810328\\
34543	30.7556084419895\\
34544	27.9284205867743\\
34548	22.7047365032398\\
34549	17.4277495272624\\
34550	19.583300815828\\
34551	21.2705831255036\\
34554	15.9857905764415\\
34555	17.3747705953928\\
34556	18.3005755960578\\
34557	21.4463103121754\\
34558	30.1800694891127\\
34561	17.6083988684927\\
34562	18.9176857160566\\
34563	17.514767584336\\
34564	11.5081192380132\\
34565	13.213931924262\\
34568	16.0411142222009\\
34569	14.0992084668179\\
34570	11.7129606616178\\
34571	10.6000043101375\\
34572	7.19100901672127\\
34576	7.01234835692554\\
34577	7.07290374961897\\
34578	9.87590165893846\\
34579	7.74370018073819\\
34582	13.5620467487773\\
34583	8.33971067335153\\
34584	7.52958473022937\\
34585	8.4284667379448\\
34586	12.1623265411268\\
34589	10.9168575659991\\
34590	12.9678057597063\\
34591	14.8690472727936\\
34592	16.1902626028676\\
34593	5.49014069916507\\
34596	21.3359586341322\\
34597	20.3580778967523\\
34598	9.29160551607492\\
34599	13.9306429479764\\
34600	12.9326549964686\\
34603	12.7178844145359\\
34604	14.6024153017673\\
34605	20.5049366888777\\
34606	0.722002030666545\\
34607	30.3592187881992\\
34610	0.212484096973071\\
34611	8.26213102272868\\
34612	13.5862753796494\\
34613	12.9667756868581\\
34614	7.92165844113208\\
34617	13.9286796485122\\
34618	23.038491301093\\
34619	9.01169386623974\\
34620	1.60204604289203\\
34621	1.64500964831825\\
34624	1.60962219438238\\
34625	2.29973579926525\\
34626	0.346347976373522\\
34627	1.75020725278034\\
34628	0.294706896952448\\
34631	1.47112209095724\\
34632	36.0862484830925\\
34633	2.47064928893315\\
34634	13.4110249966254\\
34635	1.25031866234406\\
34638	6.0541017724945\\
34639	8.70314582756657\\
34640	8.33524756409759\\
34641	3.08172011844647\\
34642	8.84796231559733\\
34645	8.24996299345696\\
34646	4.18637701033015\\
34647	6.18578665345819\\
34648	6.48111221966644\\
34649	9.60340945444227\\
34652	2.04944771113605\\
34653	6.2807678994991\\
34654	9.20375833019563\\
34655	11.8343132796586\\
34656	10.3815684825162\\
34659	8.91693419018344\\
34660	12.0558267694056\\
34661	11.2780739618269\\
34662	16.6625174826231\\
34663	11.8675004616115\\
34666	13.9573635661596\\
34667	13.0361290479406\\
34668	13.7107186580743\\
34669	21.7123968792981\\
34670	25.223711105484\\
34673	15.8827210528644\\
34674	17.5530179978369\\
34675	17.3468124983553\\
34676	12.9397102884284\\
34677	20.7593085829837\\
34680	26.3809783463387\\
34681	17.7504864127052\\
34682	4.69524158448962\\
34683	8.66330777598819\\
34684	4.43366496192886\\
};
\addplot [color=mycolor2, forget plot]
  table[row sep=crcr]{%
34445	59.1022087483857\\
34446	61.2457846443613\\
34449	65.6311234823264\\
34450	58.5189439075536\\
34451	60.3061068833904\\
34452	46.4118939125391\\
34453	54.5633248601312\\
34457	58.9861454342366\\
34458	68.2029282915157\\
34459	55.9192818492027\\
34460	60.0591816446173\\
34463	59.0417515011466\\
34464	45.4641389970022\\
34465	47.6271075153043\\
34466	44.5002422214893\\
34467	50.1227614355381\\
34470	40.5151645957776\\
34471	52.5996636409875\\
34472	55.2531978540254\\
34473	49.4376978226089\\
34474	56.6228927661059\\
34477	54.6545102363532\\
34478	53.65085750829\\
34479	26.053243035878\\
34480	65.8842461403701\\
34481	61.1013498126465\\
34485	87.5542243701848\\
34486	54.7517108852044\\
34487	65.5704457740985\\
34488	78.0524900547548\\
34491	68.0774362153052\\
34492	81.9306027306184\\
34493	72.6977145044132\\
34494	63.6696009036805\\
34495	66.937223264998\\
34498	49.7064107051933\\
34499	60.0229381626532\\
34500	78.2307020342041\\
34501	62.0363113601962\\
34502	68.9560782369069\\
34505	64.2763683715427\\
34506	69.1919446868335\\
34507	76.6588386801643\\
34508	83.5266600650639\\
34509	82.9919959522804\\
34512	108.847281601201\\
34513	90.6926488385457\\
34514	87.0398986073451\\
34515	66.5412113768209\\
34519	92.6589950547989\\
34520	75.7073247425171\\
34521	95.124944894254\\
34522	83.5834369888532\\
34523	79.9428746797387\\
34526	75.806214590149\\
34527	81.6312608727426\\
34528	85.3998829529489\\
34529	62.0650494127558\\
34530	71.1664692255401\\
34533	68.0870528205586\\
34534	51.511090875494\\
34535	46.3423977834785\\
34536	62.7218183096586\\
34537	45.3039515128274\\
34540	43.5129155221596\\
34541	41.3755969785711\\
34542	47.8506295590771\\
34543	48.078179680833\\
34544	44.0275313765492\\
34548	37.4144393002921\\
34549	41.4816559326719\\
34550	31.6261955940569\\
34551	34.2139679831984\\
34554	26.2204669962613\\
34555	29.2726416516203\\
34556	37.2344733836144\\
34557	32.3056229133471\\
34558	47.1088013185087\\
34561	28.6133610733032\\
34562	38.2616391467313\\
34563	30.0187971522121\\
34564	20.7192352477242\\
34565	31.9788006276216\\
34568	24.8050568124851\\
34569	21.0761801303713\\
34570	17.9031187539283\\
34571	20.0555424930918\\
34572	13.3801474953635\\
34576	9.65262821563107\\
34577	12.754685885601\\
34578	14.9866068081415\\
34579	12.3803478882788\\
34582	21.923056424796\\
34583	13.049943276051\\
34584	12.6051914569808\\
34585	19.6461209591974\\
34586	18.3587565653619\\
34589	15.9830970552242\\
34590	24.1802770866856\\
34591	11.5219894966441\\
34592	30.2867230761801\\
34593	13.916683168462\\
34596	22.5546880348965\\
34597	22.7612558246876\\
34598	14.2924038402628\\
34599	20.6656502163564\\
34600	24.3166364379338\\
34603	21.4688398387307\\
34604	31.4527249723956\\
34605	34.7540676128729\\
34606	4.90282715571493\\
34607	35.4446346834709\\
34610	9.94803372238653\\
34611	22.0572468468374\\
34612	29.2325071848668\\
34613	34.3016026660287\\
34614	40.2480559741475\\
34617	16.6378398960842\\
34618	11.1092556822869\\
34619	12.629155173174\\
34620	8.92745651159407\\
34621	2.25274417068431\\
34624	7.51585664728441\\
34625	8.97588371594327\\
34626	12.7623000825035\\
34627	11.7870437203594\\
34628	13.7637486615399\\
34631	18.5865962492765\\
34632	4.77024765895408\\
34633	8.29430492086772\\
34634	13.3503571679564\\
34635	33.2090714068846\\
34638	19.8837778029783\\
34639	14.5876240922199\\
34640	14.9527065388779\\
34641	12.9215295580871\\
34642	11.7409264045489\\
34645	22.6586453603568\\
34646	21.7594140505391\\
34647	16.9198670930662\\
34648	16.7666361605155\\
34649	12.4856851416349\\
34652	22.9069988318361\\
34653	13.9450410827352\\
34654	8.93448256183547\\
34655	0.216395568813141\\
34656	1.45189550203827\\
34659	3.78761286171959\\
34660	3.69173763366007\\
34661	9.13476071071477\\
34662	3.64164369715877\\
34663	8.09067669520732\\
34666	2.27594470445979\\
34667	0.30943578839765\\
34668	2.12116962547179\\
34669	17.8049636687838\\
34670	8.74880632374652\\
34673	2.21681943109434\\
34674	8.661805212437\\
34675	16.826347455195\\
34676	1.9814632644825\\
34677	10.1859309180807\\
34680	42.4187389735221\\
34681	30.7309712583115\\
34682	17.1803901960971\\
34683	14.0334997559125\\
34684	37.9572775478098\\
};
\addplot [color=mycolor3, forget plot]
  table[row sep=crcr]{%
34445	59.073559961889\\
34446	62.4167624838756\\
34449	68.4252950642901\\
34450	63.6904471426083\\
34451	81.0151017994568\\
34452	72.6818494756295\\
34453	85.4389877306163\\
34457	90.6157596552437\\
34458	83.5228560826797\\
34459	86.5714722810551\\
34460	100.385131453666\\
34463	85.3209566439193\\
34464	76.5113891313162\\
34465	79.6787802335921\\
34466	97.3318826046459\\
34467	76.0364819875261\\
34470	69.1615192927234\\
34471	79.6939016957695\\
34472	83.6958021413076\\
34473	74.6924398158483\\
34474	73.8144645680492\\
34477	74.1480722174811\\
34478	73.0722190995257\\
34479	92.257082170582\\
34480	80.1525959147555\\
34481	95.3312123595563\\
34485	78.4723863110876\\
34486	77.7207400635382\\
34487	82.2688763243004\\
34488	95.6205188359311\\
34491	85.4400220427867\\
34492	86.2384103350851\\
34493	80.9592269296593\\
34494	73.4935944944061\\
34495	79.4792835929229\\
34498	64.5855275282322\\
34499	77.7973204238228\\
34500	85.6837284845128\\
34501	80.8598239070577\\
34502	75.8429806544189\\
34505	79.3079723730159\\
34506	79.82705170748\\
34507	90.5880186390702\\
34508	97.7587337372897\\
34509	105.760120143535\\
34512	120.61320993767\\
34513	97.5957401213805\\
34514	96.5628464095305\\
34515	70.2351717941597\\
34519	105.478666151446\\
34520	82.6271999217349\\
34521	95.5322237413129\\
34522	92.0999624040121\\
34523	87.2407089481842\\
34526	85.2553716881127\\
34527	88.0191946134837\\
34528	84.9971410707449\\
34529	74.3897863593975\\
34530	73.9749546443536\\
34533	69.4156813314478\\
34534	72.6799384319452\\
34535	61.3360164128667\\
34536	89.2781614294768\\
34537	79.3766318857247\\
34540	65.9320014413502\\
34541	61.0616036504096\\
34542	54.506765184437\\
34543	71.8856985194452\\
34544	65.1555030055567\\
34548	60.9227197518697\\
34549	28.0447396581972\\
34550	49.4653037999169\\
34551	43.9367297418357\\
34554	31.4451984598534\\
34555	36.1335627206099\\
34556	44.8345699586073\\
34557	39.1697360399744\\
34558	49.6898434660848\\
34561	35.6621453257806\\
34562	39.2663502060757\\
34563	42.5796070324017\\
34564	30.1452425865956\\
34565	39.0140319825815\\
34568	34.7417657690072\\
34569	37.6379700251013\\
34570	42.8330309717503\\
34571	33.5319219536632\\
34572	37.1407738205581\\
34576	31.7660132182177\\
34577	22.7647350900288\\
34578	26.6702360440204\\
34579	23.8949590708629\\
34582	39.6391757963303\\
34583	24.7400507620289\\
34584	23.0979458862084\\
34585	26.9916053167711\\
34586	37.6430990708884\\
34589	24.1643351146126\\
34590	37.9119285573645\\
34591	18.6892000510456\\
34592	44.5031552794096\\
34593	21.8439926914943\\
34596	35.4656186828756\\
34597	31.7187967011041\\
34598	37.0232277327598\\
34599	28.4801069923594\\
34600	36.9722167226582\\
34603	34.5267108341727\\
34604	33.5151515030693\\
34605	45.5707680681337\\
34606	4.10054646698904\\
34607	46.44174425868\\
34610	9.42303522600605\\
34611	26.976144109148\\
34612	32.6423900672348\\
34613	36.1431023877144\\
34614	23.3732165694616\\
34617	18.821741117612\\
34618	16.7891108100216\\
34619	21.3746900678757\\
34620	42.4765644732759\\
34621	18.1335871258148\\
34624	17.4887832758238\\
34625	18.8325568995497\\
34626	5.26442366753645\\
34627	17.0082621920635\\
34628	29.6169745838863\\
34631	37.4210386562172\\
34632	31.8474118232402\\
34633	22.6399196135972\\
34634	28.0886327933099\\
34635	26.7809102566951\\
34638	22.7778615837556\\
34639	29.1779961127368\\
34640	15.5441160275047\\
34641	41.5938062781245\\
34642	23.4807256630888\\
34645	23.3720197225859\\
34646	34.7528312313152\\
34647	29.7980187846867\\
34648	31.556768713702\\
34649	19.3937408259915\\
34652	36.7326837507239\\
34653	24.4065037057203\\
34654	14.5965388362636\\
34655	4.92015812383761\\
34656	8.30770257090012\\
34659	12.5625546175443\\
34660	8.25379923159107\\
34661	13.0229220567967\\
34662	4.63534476830637\\
34663	23.9292830062909\\
34666	26.1615263361632\\
34667	5.32484765148774\\
34668	0.643939223813732\\
34669	4.84355629342735\\
34670	5.28308006086718\\
34673	5.4147837668429\\
34674	2.04055251429418\\
34675	2.63611766437634\\
34676	15.4187563594642\\
34677	11.6243535456483\\
34680	4.46517145499138\\
34681	14.9580415933838\\
34682	0.51979951510566\\
34683	21.9735294706006\\
34684	38.5134380288955\\
};
\addplot [color=mycolor4, forget plot]
  table[row sep=crcr]{%
34445	92.5153771718283\\
34446	96.3008882513227\\
34449	102.09246597996\\
34450	92.9056504241453\\
34451	97.7229479613325\\
34452	76.600818022235\\
34453	92.6435690309188\\
34457	100.66628243671\\
34458	104.751643775785\\
34459	95.1038159620896\\
34460	96.6467780230596\\
34463	93.5292824747683\\
34464	87.7779881089537\\
34465	90.6371117603812\\
34466	84.9495101509449\\
34467	92.1305310188784\\
34470	76.5689705805055\\
34471	95.5659380507088\\
34472	99.5021321876529\\
34473	90.5484499760028\\
34474	93.514620765653\\
34477	100.63703296794\\
34478	89.1143751009563\\
34479	103.089007886676\\
34480	95.7527463799531\\
34481	107.496397320303\\
34485	117.095918475012\\
34486	86.3084501009535\\
34487	96.8476642348251\\
34488	113.126660172064\\
34491	101.611117016097\\
34492	102.041483238908\\
34493	99.2753863011483\\
34494	88.7720120672166\\
34495	99.2697857947935\\
34498	77.6344732845064\\
34499	111.982673231793\\
34500	105.116826383318\\
34501	97.7089716131845\\
34502	106.899576196357\\
34505	89.8627077655756\\
34506	85.429137974204\\
34507	99.8842770230312\\
34508	104.91474811224\\
34509	103.288513756438\\
34512	84.8461120851998\\
34513	93.7716777283272\\
34514	97.3981077602184\\
34515	65.0170762293578\\
34519	110.262828564933\\
34520	81.6050950020199\\
34521	95.1304447150956\\
34522	93.0730953313714\\
34523	86.8270466784084\\
34526	86.5326355261477\\
34527	87.2619517124658\\
34528	96.4512307041582\\
34529	82.3770321516445\\
34530	84.8662489812673\\
34533	104.274202242653\\
34534	85.5566892710663\\
34535	69.5700360144704\\
34536	106.463254814893\\
34537	82.179031768384\\
34540	78.3005250089032\\
34541	75.3747237773764\\
34542	63.0026000376074\\
34543	84.4047998291121\\
34544	75.0055092679538\\
34548	77.3176812666782\\
34549	76.3653312532642\\
34550	70.2872082587394\\
34551	80.1447613024566\\
34554	63.6810522468165\\
34555	69.873382866117\\
34556	68.3334884420337\\
34557	71.8886519420155\\
34558	84.0115744739055\\
34561	66.0798006646799\\
34562	70.6545675480036\\
34563	78.2095329841131\\
34564	62.4076457781116\\
34565	68.9589773601945\\
34568	63.9600532723994\\
34569	61.5679961292699\\
34570	58.0507418021339\\
34571	58.5792419613601\\
34572	48.2980126407479\\
34576	49.1214310981247\\
34577	46.4783152187496\\
34578	52.0957354904365\\
34579	50.9135447074013\\
34582	47.4992070811602\\
34583	53.6757980438879\\
34584	48.2514585510089\\
34585	50.8934030865255\\
34586	56.946477801484\\
34589	57.7340530484728\\
34590	59.3101658096066\\
34591	60.5250340865373\\
34592	2.79552964431196\\
34593	33.9864677042742\\
34596	42.1472375099056\\
34597	27.4866500731337\\
34598	34.0290035348235\\
34599	24.2262399197857\\
34600	23.7772443398217\\
34603	28.8563215828735\\
34604	28.5380101868172\\
34605	42.1682459928697\\
34606	9.46701663599652\\
34607	42.5441239909319\\
34610	2.49785212070947\\
34611	19.3489711575426\\
34612	20.1436316094346\\
34613	31.9687631207353\\
34614	15.8338289428189\\
34617	10.0323726758702\\
34618	8.36309503409075\\
34619	17.0670237303402\\
34620	11.1000883463701\\
34621	0.463859790655533\\
34624	10.2640558480773\\
34625	10.3897704958581\\
34626	14.5188396865715\\
34627	14.4637242075551\\
34628	13.800497269382\\
34631	24.9785973891353\\
34632	15.3430591887982\\
34633	5.64175892307458\\
34634	12.1292300364944\\
34635	9.63116877310813\\
34638	4.55034770426028\\
34639	14.5930252638555\\
34640	14.5739608428694\\
34641	8.63527434065509\\
34642	5.420783671344\\
34645	23.0500442896591\\
34646	25.087453303106\\
34647	19.67525033632\\
34648	23.2882309979859\\
34649	11.9335559226411\\
34652	31.394939196301\\
34653	9.03526075896207\\
34654	4.1221694840367\\
34655	2.59458786100095\\
34656	0.686829067836698\\
34659	4.93962916441023\\
34660	4.26892413125188\\
34661	8.74578108322476\\
34662	11.4975545700313\\
34663	19.746755216815\\
34666	0.595937640027842\\
34667	0.381385680916537\\
34668	8.93072254098024\\
34669	3.37180489391085\\
34670	17.0392590670631\\
34673	15.9571151139671\\
34674	5.30729858000495\\
34675	2.77605112491347\\
34676	16.2259947778321\\
34677	11.7355271951337\\
34680	4.35040489441417\\
34681	6.11985089392647\\
34682	21.5900188497658\\
34683	18.9399002999107\\
34684	44.5300934157858\\
};
\addplot [color=mycolor5, forget plot]
  table[row sep=crcr]{%
34445	106.873688058088\\
34446	114.184960746754\\
34449	116.350064156027\\
34450	106.807811333533\\
34451	114.698581489438\\
34452	87.6263289371768\\
34453	106.813322653604\\
34457	111.49827246221\\
34458	114.103748063571\\
34459	106.545233994227\\
34460	109.083314048042\\
34463	104.286841930294\\
34464	99.1226075122884\\
34465	104.489085187458\\
34466	99.872385404929\\
34467	106.873813683824\\
34470	87.9533960902979\\
34471	110.04375949371\\
34472	113.479031710723\\
34473	103.526559829456\\
34474	107.528797346186\\
34477	114.567614476443\\
34478	98.7410319097012\\
34479	107.955085088182\\
34480	101.141609323483\\
34481	109.043502820409\\
34485	93.1472885497087\\
34486	86.0881194895123\\
34487	98.0383367567952\\
34488	114.965782337541\\
34491	104.121860920543\\
34492	104.758036269496\\
34493	103.165605555968\\
34494	90.913480789005\\
34495	103.304899416446\\
34498	75.4535312311546\\
34499	97.9911580390644\\
34500	108.495017060784\\
34501	98.9725628539413\\
34502	108.555623284396\\
34505	85.503666791979\\
34506	78.1539041784322\\
34507	93.8143365358428\\
34508	98.9805644312914\\
34509	95.7733654522044\\
34512	76.2430049893223\\
34513	86.3805827890105\\
34514	90.3402565381598\\
34515	54.7854340230938\\
34519	105.443021467232\\
34520	73.3942847402413\\
34521	86.8145387555951\\
34522	85.3489579752872\\
34523	77.4978324698918\\
34526	78.2995357687823\\
34527	79.5466197190372\\
34528	80.8224381306677\\
34529	76.7263288476724\\
34530	85.5581800634\\
34533	81.8998887904763\\
34534	86.7095513489057\\
34535	67.9783627651273\\
34536	112.654538963485\\
34537	86.7017076891075\\
34540	81.0887175444145\\
34541	80.7082534196593\\
34542	84.2571697759545\\
34543	87.4493382004694\\
34544	75.5065430505115\\
34548	86.9225552390085\\
34549	74.9593058865476\\
34550	79.2292783189296\\
34551	92.4102354762795\\
34554	73.0507569847664\\
34555	80.90715843311\\
34556	78.1445148819421\\
34557	79.9564718266338\\
34558	94.6261613062993\\
34561	69.3217718058127\\
34562	76.76497949748\\
34563	89.9230763138607\\
34564	68.0409010537123\\
34565	71.6122949303208\\
34568	69.7571821654187\\
34569	67.2888116679023\\
34570	66.2987897205053\\
34571	72.4501209330858\\
34572	62.5367357050882\\
34576	64.2401428777728\\
34577	61.1528570688952\\
34578	65.0038963648608\\
34579	61.9747359899109\\
34582	59.5048607967051\\
34583	63.4694265806315\\
34584	59.6809769041542\\
34585	59.2402413494574\\
34586	62.5803824793275\\
34589	50.6089469288486\\
34590	66.0055071778902\\
34591	60.5615029298428\\
34592	49.2886751164297\\
34593	18.5818683640255\\
34596	52.7533054123942\\
34597	40.5039527775111\\
34598	45.8305550783948\\
34599	34.6442236635839\\
34600	34.3846429266659\\
34603	38.0206137856167\\
34604	37.7591077789893\\
34605	56.096099435681\\
34606	6.75320083813722\\
34607	54.0471085104546\\
34610	0.436834501135763\\
34611	24.778719507523\\
34612	21.8277789646418\\
34613	26.9888265077798\\
34614	19.2938313766376\\
34617	15.1565399387005\\
34618	15.7272230854933\\
34619	26.8029612515606\\
34620	26.5048185712003\\
34621	25.2274889432038\\
34624	23.081505706898\\
34625	21.2527893556539\\
34626	21.7707477645313\\
34627	21.1726597502047\\
34628	18.1011738306552\\
34631	31.556347950861\\
34632	23.0337760995344\\
34633	35.4799293672982\\
34634	13.209810181042\\
34635	14.7019001857552\\
34638	14.0085092705376\\
34639	21.4281549430812\\
34640	19.436175340752\\
34641	13.2877193215236\\
34642	9.54894551275038\\
34645	26.6145843777722\\
34646	30.0984438542073\\
34647	27.1478549704502\\
34648	31.0722052264164\\
34649	15.4680111638959\\
34652	36.9381789804852\\
34653	12.1744259510328\\
34654	8.15828837527442\\
34655	3.17161438268886\\
34656	2.59743398713226\\
34659	53.792306204075\\
34660	9.14748529473763\\
34661	8.05518766460318\\
34662	8.98098608478267\\
34663	21.9692804649085\\
34666	0.656458562083117\\
34667	0.288309811589443\\
34668	6.77300665009034\\
34669	3.22973247700838\\
34670	17.1874892534356\\
34673	16.5521104961863\\
34674	4.99313284955178\\
34675	2.46125713637912\\
34676	15.4155396112524\\
34677	16.1302711403723\\
34680	3.26824608005427\\
34681	16.1855827433669\\
34682	0.639321777844998\\
34683	21.9070289913802\\
34684	38.5465304209265\\
};
\end{axis}
\end{tikzpicture}% }
		\label{fig:q2-error}
    \end{subfigure}
    ~
    \begin{subfigure}[b]{0.45\textwidth}
       	\resizebox {\textwidth} {!} {% This file was created by matlab2tikz.
%
%The latest updates can be retrieved from
%  http://www.mathworks.com/matlabcentral/fileexchange/22022-matlab2tikz-matlab2tikz
%where you can also make suggestions and rate matlab2tikz.
%
\begin{tikzpicture}

\begin{axis}[%
width=2.882in,
height=2.616in,
at={(1.305in,0.507in)},
scale only axis,
unbounded coords=jump,
xmin=0.5,
xmax=5.5,
xtick={1,2,3,4,5},
xlabel style={font=\color{white!15!black}},
xlabel={Call Options},
ymin=-8.09925481700059,
ymax=170.185494857615,
ylabel style={font=\color{white!15!black}},
ylabel={Absolute Error},
axis background/.style={fill=white},
title style={font=\bfseries},
title={Call Option Pricing Absolute Errors},
xmajorgrids,
ymajorgrids
]
\addplot [color=black, dashed, forget plot]
  table[row sep=crcr]{%
1	20.4399141462496\\
1	39.7586823060606\\
};
\addplot [color=black, dashed, forget plot]
  table[row sep=crcr]{%
2	17.8495513744533\\
2	35.9384864841159\\
};
\addplot [color=black, dashed, forget plot]
  table[row sep=crcr]{%
3	27.4908748561549\\
3	54.4022489956017\\
};
\addplot [color=black, dashed, forget plot]
  table[row sep=crcr]{%
4	41.1270310810077\\
4	74.9197468282098\\
};
\addplot [color=black, dashed, forget plot]
  table[row sep=crcr]{%
5	64.6724904452281\\
5	127.501233722899\\
};
\addplot [color=black, dashed, forget plot]
  table[row sep=crcr]{%
1	0.0136739486329134\\
1	6.79196238722864\\
};
\addplot [color=black, dashed, forget plot]
  table[row sep=crcr]{%
2	0.0902470366454509\\
2	4.66447763852295\\
};
\addplot [color=black, dashed, forget plot]
  table[row sep=crcr]{%
3	0.197000462867436\\
3	7.14778095208524\\
};
\addplot [color=black, dashed, forget plot]
  table[row sep=crcr]{%
4	0.00459744093648262\\
4	16.8567778479334\\
};
\addplot [color=black, dashed, forget plot]
  table[row sep=crcr]{%
5	0.0468196151632734\\
5	22.7161200626452\\
};
\addplot [color=black, forget plot]
  table[row sep=crcr]{%
0.875	39.7586823060606\\
1.125	39.7586823060606\\
};
\addplot [color=black, forget plot]
  table[row sep=crcr]{%
1.875	35.9384864841159\\
2.125	35.9384864841159\\
};
\addplot [color=black, forget plot]
  table[row sep=crcr]{%
2.875	54.4022489956017\\
3.125	54.4022489956017\\
};
\addplot [color=black, forget plot]
  table[row sep=crcr]{%
3.875	74.9197468282098\\
4.125	74.9197468282098\\
};
\addplot [color=black, forget plot]
  table[row sep=crcr]{%
4.875	127.501233722899\\
5.125	127.501233722899\\
};
\addplot [color=black, forget plot]
  table[row sep=crcr]{%
0.875	0.0136739486329134\\
1.125	0.0136739486329134\\
};
\addplot [color=black, forget plot]
  table[row sep=crcr]{%
1.875	0.0902470366454509\\
2.125	0.0902470366454509\\
};
\addplot [color=black, forget plot]
  table[row sep=crcr]{%
2.875	0.197000462867436\\
3.125	0.197000462867436\\
};
\addplot [color=black, forget plot]
  table[row sep=crcr]{%
3.875	0.00459744093648262\\
4.125	0.00459744093648262\\
};
\addplot [color=black, forget plot]
  table[row sep=crcr]{%
4.875	0.0468196151632734\\
5.125	0.0468196151632734\\
};
\addplot [color=blue, forget plot]
  table[row sep=crcr]{%
0.75	6.79196238722864\\
0.75	20.4399141462496\\
1.25	20.4399141462496\\
1.25	6.79196238722864\\
0.75	6.79196238722864\\
};
\addplot [color=blue, forget plot]
  table[row sep=crcr]{%
1.75	4.66447763852295\\
1.75	17.8495513744533\\
2.25	17.8495513744533\\
2.25	4.66447763852295\\
1.75	4.66447763852295\\
};
\addplot [color=blue, forget plot]
  table[row sep=crcr]{%
2.75	7.14778095208524\\
2.75	27.4908748561549\\
3.25	27.4908748561549\\
3.25	7.14778095208524\\
2.75	7.14778095208524\\
};
\addplot [color=blue, forget plot]
  table[row sep=crcr]{%
3.75	16.8567778479334\\
3.75	41.1270310810077\\
4.25	41.1270310810077\\
4.25	16.8567778479334\\
3.75	16.8567778479334\\
};
\addplot [color=blue, forget plot]
  table[row sep=crcr]{%
4.75	22.7161200626452\\
4.75	64.6724904452281\\
5.25	64.6724904452281\\
5.25	22.7161200626452\\
4.75	22.7161200626452\\
};
\addplot [color=red, forget plot]
  table[row sep=crcr]{%
0.75	13.8738363922066\\
1.25	13.8738363922066\\
};
\addplot [color=red, forget plot]
  table[row sep=crcr]{%
1.75	11.2530738760079\\
2.25	11.2530738760079\\
};
\addplot [color=red, forget plot]
  table[row sep=crcr]{%
2.75	14.8848737183833\\
3.25	14.8848737183833\\
};
\addplot [color=red, forget plot]
  table[row sep=crcr]{%
3.75	27.7544295243796\\
4.25	27.7544295243796\\
};
\addplot [color=red, forget plot]
  table[row sep=crcr]{%
4.75	43.9625705578711\\
5.25	43.9625705578711\\
};
\addplot [color=black, draw=none, mark=+, mark options={solid, red}, forget plot]
  table[row sep=crcr]{%
1	42.8203679955641\\
1	44.4830476170091\\
1	48.4370910585435\\
1	49.4649223175024\\
1	49.7584680948771\\
1	52.0338121763532\\
};
\addplot [color=black, draw=none, mark=+, mark options={solid, red}, forget plot]
  table[row sep=crcr]{%
2	44.5129286062054\\
2	63.4754658273571\\
};
\addplot [color=black, draw=none, mark=+, mark options={solid, red}, forget plot]
  table[row sep=crcr]{%
3	90.6208800994464\\
};
\addplot [color=black, draw=none, mark=+, mark options={solid, red}, forget plot]
  table[row sep=crcr]{%
nan	nan\\
};
\addplot [color=black, draw=none, mark=+, mark options={solid, red}, forget plot]
  table[row sep=crcr]{%
5	128.309603389195\\
5	131.714190209293\\
5	134.456531478901\\
5	137.032921024787\\
5	137.455386487231\\
5	138.128077601478\\
5	139.38082514495\\
5	140.131274008453\\
5	141.577877551172\\
5	143.887792505737\\
5	146.38354382521\\
5	147.056443760795\\
5	148.38727082979\\
5	162.081642599678\\
};
\end{axis}

\begin{axis}[%
width=2.867in,
height=2.616in,
at={(5.273in,0.507in)},
scale only axis,
unbounded coords=jump,
xmin=0.5,
xmax=5.5,
xtick={1,2,3,4,5},
xlabel style={font=\color{white!15!black}},
xlabel={Put Options},
ymin=-5.8075521950618,
ymax=126.633246229705,
ylabel style={font=\color{white!15!black}},
ylabel={Absolute Error},
axis background/.style={fill=white},
title style={font=\bfseries},
title={Put Option Pricing Absolute Errors},
xmajorgrids,
ymajorgrids
]
\addplot [color=black, dashed, forget plot]
  table[row sep=crcr]{%
1	43.6490358523677\\
1	88.7103557227733\\
};
\addplot [color=black, dashed, forget plot]
  table[row sep=crcr]{%
2	59.0278499844191\\
2	108.847281601201\\
};
\addplot [color=black, dashed, forget plot]
  table[row sep=crcr]{%
3	77.7781753337517\\
3	120.61320993767\\
};
\addplot [color=black, dashed, forget plot]
  table[row sep=crcr]{%
4	90.6149463142866\\
4	117.095918475012\\
};
\addplot [color=black, dashed, forget plot]
  table[row sep=crcr]{%
5	92.0360468044609\\
5	116.350064156027\\
};
\addplot [color=black, dashed, forget plot]
  table[row sep=crcr]{%
1	0.212484096973071\\
1	10.0023183648329\\
};
\addplot [color=black, dashed, forget plot]
  table[row sep=crcr]{%
2	0.216395568813141\\
2	13.9671557510295\\
};
\addplot [color=black, dashed, forget plot]
  table[row sep=crcr]{%
3	0.51979951510566\\
3	23.9035400547199\\
};
\addplot [color=black, dashed, forget plot]
  table[row sep=crcr]{%
4	0.381385680916537\\
4	16.4293108501398\\
};
\addplot [color=black, dashed, forget plot]
  table[row sep=crcr]{%
5	0.288309811589443\\
5	23.0457085013753\\
};
\addplot [color=black, forget plot]
  table[row sep=crcr]{%
0.875	88.7103557227733\\
1.125	88.7103557227733\\
};
\addplot [color=black, forget plot]
  table[row sep=crcr]{%
1.875	108.847281601201\\
2.125	108.847281601201\\
};
\addplot [color=black, forget plot]
  table[row sep=crcr]{%
2.875	120.61320993767\\
3.125	120.61320993767\\
};
\addplot [color=black, forget plot]
  table[row sep=crcr]{%
3.875	117.095918475012\\
4.125	117.095918475012\\
};
\addplot [color=black, forget plot]
  table[row sep=crcr]{%
4.875	116.350064156027\\
5.125	116.350064156027\\
};
\addplot [color=black, forget plot]
  table[row sep=crcr]{%
0.875	0.212484096973071\\
1.125	0.212484096973071\\
};
\addplot [color=black, forget plot]
  table[row sep=crcr]{%
1.875	0.216395568813141\\
2.125	0.216395568813141\\
};
\addplot [color=black, forget plot]
  table[row sep=crcr]{%
2.875	0.51979951510566\\
3.125	0.51979951510566\\
};
\addplot [color=black, forget plot]
  table[row sep=crcr]{%
3.875	0.381385680916537\\
4.125	0.381385680916537\\
};
\addplot [color=black, forget plot]
  table[row sep=crcr]{%
4.875	0.288309811589443\\
5.125	0.288309811589443\\
};
\addplot [color=blue, forget plot]
  table[row sep=crcr]{%
0.75	10.0023183648329\\
0.75	43.6490358523677\\
1.25	43.6490358523677\\
1.25	10.0023183648329\\
0.75	10.0023183648329\\
};
\addplot [color=blue, forget plot]
  table[row sep=crcr]{%
1.75	13.9671557510295\\
1.75	59.0278499844191\\
2.25	59.0278499844191\\
2.25	13.9671557510295\\
1.75	13.9671557510295\\
};
\addplot [color=blue, forget plot]
  table[row sep=crcr]{%
2.75	23.9035400547199\\
2.75	77.7781753337517\\
3.25	77.7781753337517\\
3.25	23.9035400547199\\
2.75	23.9035400547199\\
};
\addplot [color=blue, forget plot]
  table[row sep=crcr]{%
3.75	16.4293108501398\\
3.75	90.6149463142866\\
4.25	90.6149463142866\\
4.25	16.4293108501398\\
3.75	16.4293108501398\\
};
\addplot [color=blue, forget plot]
  table[row sep=crcr]{%
4.75	23.0457085013753\\
4.75	92.0360468044609\\
5.25	92.0360468044609\\
5.25	23.0457085013753\\
4.75	23.0457085013753\\
};
\addplot [color=red, forget plot]
  table[row sep=crcr]{%
0.75	19.583300815828\\
1.25	19.583300815828\\
};
\addplot [color=red, forget plot]
  table[row sep=crcr]{%
1.75	31.6261955940569\\
2.25	31.6261955940569\\
};
\addplot [color=red, forget plot]
  table[row sep=crcr]{%
2.75	39.1697360399744\\
3.25	39.1697360399744\\
};
\addplot [color=red, forget plot]
  table[row sep=crcr]{%
3.75	62.4076457781116\\
4.25	62.4076457781116\\
};
\addplot [color=red, forget plot]
  table[row sep=crcr]{%
4.75	67.9783627651273\\
5.25	67.9783627651273\\
};
\addplot [color=black, draw=none, mark=+, mark options={solid, red}, forget plot]
  table[row sep=crcr]{%
1	104.110396206845\\
};
\addplot [color=black, draw=none, mark=+, mark options={solid, red}, forget plot]
  table[row sep=crcr]{%
nan	nan\\
};
\addplot [color=black, draw=none, mark=+, mark options={solid, red}, forget plot]
  table[row sep=crcr]{%
nan	nan\\
};
\addplot [color=black, draw=none, mark=+, mark options={solid, red}, forget plot]
  table[row sep=crcr]{%
nan	nan\\
};
\addplot [color=black, draw=none, mark=+, mark options={solid, red}, forget plot]
  table[row sep=crcr]{%
nan	nan\\
};
\end{axis}
\end{tikzpicture}% }
        \label{fig:q2-error-boxplot}
    \end{subfigure}
	\vspace{-0.5cm}
    \caption{Absolute Error between Black-Scholes and Actual Option Prices}
	\label{fig:q2-error-both}
\end{figure}

\begin{wrapfigure}{!h}{0.45\textwidth}
\vspace{-1.5cm}
\begin{center}
	\resizebox {0.4\textwidth} {!} {% This file was created by matlab2tikz.
%
%The latest updates can be retrieved from
%  http://www.mathworks.com/matlabcentral/fileexchange/22022-matlab2tikz-matlab2tikz
%where you can also make suggestions and rate matlab2tikz.
%
\definecolor{mycolor1}{rgb}{0.00000,0.44700,0.74100}%
\definecolor{mycolor2}{rgb}{0.85000,0.32500,0.09800}%
\definecolor{mycolor3}{rgb}{0.49400,0.18400,0.55600}%
\definecolor{mycolor4}{rgb}{0.46600,0.67400,0.18800}%
%
\begin{tikzpicture}

\begin{axis}[%
width=3.888in,
height=1.366in,
at={(0.703in,7.073in)},
scale only axis,
xmin=34445,
xmax=34684,
xtick={34425,34455,34486,34516,34547,34578,34608,34639,34669,34700},
xticklabels={{01/04},{01/05},{01/06},{01/07},{01/08},{01/09},{01/10},{01/11},{01/12},{01/01}},
ymin=0.0717875021740368,
ymax=0.670497565510376,
ylabel style={font=\color{white!15!black}},
ylabel={Value},
axis background/.style={fill=white},
title style={font=\bfseries},
title={$\sigma\text{: Volatility}$},
xmajorgrids,
ymajorgrids
]
\addplot [color=mycolor1, forget plot]
  table[row sep=crcr]{%
34445	0.072174158665011\\
34446	0.0717875021740368\\
34449	0.0721081212314408\\
34450	0.0727596909557625\\
34451	0.0726284220038499\\
34452	0.0721566069871888\\
34453	0.0726888794107518\\
34457	0.0734270359406684\\
34458	0.0745187206715815\\
34459	0.0752805456229311\\
34460	0.0770774557954604\\
34463	0.0767811023034568\\
34464	0.0766915382011215\\
34465	0.0776392110967477\\
34466	0.0774086836760777\\
34467	0.0776526385016974\\
34470	0.0782188873674324\\
34471	0.0757210303209449\\
34472	0.0755897127617172\\
34473	0.074571678787986\\
34474	0.0736951334148076\\
34477	0.073790512876073\\
34478	0.0745915562698013\\
34479	0.0737745975290772\\
34480	0.0821987526659823\\
34481	0.0819834291307861\\
34485	0.0880720880940582\\
34486	0.0883398554713\\
34487	0.0881523964107483\\
34488	0.0861892869819902\\
34491	0.0852861193088532\\
34492	0.0863650990204557\\
34493	0.0860923308134936\\
34494	0.0869349470674757\\
34495	0.0869551651980913\\
34498	0.0869225054365023\\
34499	0.0852807191998381\\
34500	0.0850331656311313\\
34501	0.0849047613472267\\
34502	0.0849063168960453\\
34505	0.08538058062714\\
34506	0.0841133862662646\\
34507	0.0846909179606757\\
34508	0.0846070916013497\\
34509	0.0794839912371304\\
34512	0.0802088789077319\\
34513	0.0802196731603628\\
34514	0.080372169571437\\
34515	0.0795761802160135\\
34519	0.0816908250442697\\
34520	0.0816940566588828\\
34521	0.0808896422473749\\
34522	0.0810931236483834\\
34523	0.0819942607190514\\
34526	0.0814122011967536\\
34527	0.0820383366421992\\
34528	0.0824727138530659\\
34529	0.0848873504251955\\
34530	0.0874224885846762\\
34533	0.0878318172331205\\
34534	0.0882554050119743\\
34535	0.0882449778638515\\
34536	0.0872292165401006\\
34537	0.0880713932816014\\
34540	0.0895366306515552\\
34541	0.0896335071136564\\
34542	0.0898470841138594\\
34543	0.0907749684434616\\
34544	0.0907968665382936\\
34548	0.0906233039604377\\
34549	0.0936978583074162\\
34550	0.0936758467603169\\
34551	0.0937183472447836\\
34554	0.0936230248191373\\
34555	0.0939933803038501\\
34556	0.0939159901144205\\
34557	0.0931884856153359\\
34558	0.0939159486784333\\
34561	0.0861009602172393\\
34562	0.0871165327129804\\
34563	0.0796290518994557\\
34564	0.080081973077816\\
34565	0.080126847729614\\
34568	0.0800798938653106\\
34569	0.0806529864612229\\
34570	0.0797737530122539\\
34571	0.0805121893542027\\
34572	0.079675301057479\\
34576	0.0802395492812388\\
34577	0.080705384916187\\
34578	0.0806216672379107\\
34579	0.0822040598154597\\
34582	0.0820044103168124\\
34583	0.0821002973419636\\
34584	0.083145689357672\\
34585	0.0822355199087733\\
34586	0.082130921349821\\
34589	0.0855182795633091\\
34590	0.085119931450486\\
34591	0.084377282633198\\
34592	0.08654266245737\\
34593	0.089729741683645\\
34596	0.0962350060244339\\
34597	0.0947438544909664\\
34598	0.094799190860977\\
34599	0.0954350377837906\\
34600	0.0944566687447967\\
34603	0.0941099164449127\\
34604	0.0963850648784965\\
34605	0.095726990888543\\
34606	0.0951631954490235\\
34607	0.0935792129620513\\
34610	0.0934088609130781\\
34611	0.0933958993695243\\
34612	0.0948039561671414\\
34613	0.0939915418724626\\
34614	0.093880738669308\\
34617	0.0935154448002031\\
34618	0.0931612051294437\\
34619	0.0963415813829917\\
34620	0.0963328533443371\\
34621	0.0975091887349544\\
34624	0.0999635858577407\\
34625	0.100275145578502\\
34626	0.0988110208757673\\
34627	0.0999128647327256\\
34628	0.0998128556052828\\
34631	0.10122327524536\\
34632	0.100557029848941\\
34633	0.101314505921686\\
34634	0.107187313145317\\
34635	0.111961487140921\\
34638	0.11771359280168\\
34639	0.116846230715251\\
34640	0.116997864108454\\
34641	0.116507792618792\\
34642	0.117039667027871\\
34645	0.117039159081638\\
34646	0.120908692888242\\
34647	0.120961705009608\\
34648	0.123489109901938\\
34649	0.123016951959739\\
34652	0.122677677256158\\
34653	0.122709992879051\\
34654	0.132635319991987\\
34655	0.13189648526208\\
34656	0.132079782967177\\
34659	0.132000964811332\\
34660	0.131391887348668\\
34661	0.13235677120162\\
34662	0.141317201273425\\
34663	0.139799590768194\\
34666	0.139858359956449\\
34667	0.142465445605986\\
34668	0.141766266399594\\
34669	0.139870308826816\\
34670	0.140945330780401\\
34673	0.145573508052566\\
34674	0.145896736988472\\
34675	0.145263993831865\\
34676	0.145750175320327\\
34677	0.146356238840025\\
34680	0.145099644960972\\
34681	0.155842034712037\\
34682	0.16368404265595\\
34683	0.162901044895112\\
34684	0.167536806961005\\
};
\addplot [color=mycolor1, forget plot]
  table[row sep=crcr]{%
34445	0.0844886490003252\\
34446	0.0837568283273915\\
34449	0.0841684614317449\\
34450	0.0849798044174744\\
34451	0.0849243437712324\\
34452	0.0843890729804099\\
34453	0.0849761342890215\\
34457	0.085826170120603\\
34458	0.0871748606586728\\
34459	0.0871742832256072\\
34460	0.0877192656819691\\
34463	0.0874123792906821\\
34464	0.0873174597694498\\
34465	0.0885828814600233\\
34466	0.0883145943141211\\
34467	0.0885681039467589\\
34470	0.0892276082869081\\
34471	0.0863744891737641\\
34472	0.0862362712258156\\
34473	0.0858130236040265\\
34474	0.0856700517611617\\
34477	0.0858928769856282\\
34478	0.0868271544748096\\
34479	0.0859110959784865\\
34480	0.0877824820678269\\
34481	0.0878957938825688\\
34485	0.0967389660593585\\
34486	0.0987477512821511\\
34487	0.0977551941061849\\
34488	0.0954821030699018\\
34491	0.0944603188842589\\
34492	0.0958843334631239\\
34493	0.0972624969557712\\
34494	0.0966781656314634\\
34495	0.0969523784707403\\
34498	0.0969659430840404\\
34499	0.0950871033026955\\
34500	0.0947700202537731\\
34501	0.0941549503091354\\
34502	0.0926432339575073\\
34505	0.0933319873756572\\
34506	0.0917543809080479\\
34507	0.0917476745697304\\
34508	0.0909815114660559\\
34509	0.0841580825512451\\
34512	0.0854037192520346\\
34513	0.0995070205010959\\
34514	0.110020667809453\\
34515	0.10920456343816\\
34519	0.109188572493355\\
34520	0.109187708608916\\
34521	0.111406973164261\\
34522	0.111078122039813\\
34523	0.11106896932451\\
34526	0.11087651193559\\
34527	0.112781005076487\\
34528	0.11263188744272\\
34529	0.11205994222813\\
34530	0.111590557886242\\
34533	0.117825532485808\\
34534	0.117679653607092\\
34535	0.117072455077497\\
34536	0.116041776753812\\
34537	0.116727834819866\\
34540	0.117916007932748\\
34541	0.117920926176676\\
34542	0.118132292655308\\
34543	0.119157019437224\\
34544	0.11920946633894\\
34548	0.119046910862778\\
34549	0.122565448021029\\
34550	0.122552488440829\\
34551	0.122618951805462\\
34554	0.122541641486569\\
34555	0.123008735265699\\
34556	0.122874665249024\\
34557	0.12204783294418\\
34558	0.122401130100361\\
34561	0.122847824111767\\
34562	0.122981114777096\\
34563	0.114298451515074\\
34564	0.113780436926834\\
34565	0.113947915927268\\
34568	0.113768307550013\\
34569	0.114420030498049\\
34570	0.113527709878293\\
34571	0.113290662438652\\
34572	0.113219484950175\\
34576	0.115271973630063\\
34577	0.115738413511877\\
34578	0.115592665369645\\
34579	0.117215408425449\\
34582	0.11701441834561\\
34583	0.11715743657232\\
34584	0.118267224934877\\
34585	0.117433548720811\\
34586	0.117785503281818\\
34589	0.121433805677094\\
34590	0.120446308788901\\
34591	0.120267864992915\\
34592	0.10990327319994\\
34593	0.105210248285953\\
34596	0.109396563929724\\
34597	0.109568126864543\\
34598	0.113186499156656\\
34599	0.10985393300269\\
34600	0.109861165820014\\
34603	0.110119288226281\\
34604	0.110377857346717\\
34605	0.109776691916825\\
34606	0.110154578409924\\
34607	0.111088016063458\\
34610	0.114383972289234\\
34611	0.108489953868064\\
34612	0.111164394831525\\
34613	0.110894141928499\\
34614	0.110528902259746\\
34617	0.112014884785311\\
34618	0.112945067332688\\
34619	0.117578508057912\\
34620	0.117586952035994\\
34621	0.119471753544258\\
34624	0.120439157220559\\
34625	0.1204666975286\\
34626	0.118894765606615\\
34627	0.119309460512431\\
34628	0.119216878584246\\
34631	0.121871459414059\\
34632	0.121112725674269\\
34633	0.130818452319702\\
34634	0.141971588846776\\
34635	0.141761490783971\\
34638	0.146875176547947\\
34639	0.149983405064355\\
34640	0.150164314643298\\
34641	0.149588863957098\\
34642	0.149361848756028\\
34645	0.149584474704146\\
34646	0.155091559392378\\
34647	0.15509556074874\\
34648	0.156529900733956\\
34649	0.156526544692298\\
34652	0.154738795695345\\
34653	0.154793781128023\\
34654	0.160685669689275\\
34655	0.160957966798096\\
34656	0.161898546103551\\
34659	0.161578242147998\\
34660	0.160892487453411\\
34661	0.162146796244754\\
34662	0.178298112738461\\
34663	0.176831805521841\\
34666	0.178917167326489\\
34667	0.183374954047091\\
34668	0.183476544650994\\
34669	0.180742356219497\\
34670	0.188686092568594\\
34673	0.190249635533658\\
34674	0.189125013276107\\
34675	0.18893504821257\\
34676	0.190787827652471\\
34677	0.212947801765113\\
34680	0.214190857080088\\
34681	0.247351958515258\\
34682	0.35695189874565\\
34683	0.356178331395985\\
34684	0.37117099508876\\
};
\addplot [color=mycolor1, forget plot]
  table[row sep=crcr]{%
34445	0.0979164012320792\\
34446	0.096906267299644\\
34449	0.0976817058243948\\
34450	0.097710159425356\\
34451	0.096419785902638\\
34452	0.0958583805913571\\
34453	0.0964634220198998\\
34457	0.0981116149203447\\
34458	0.0979594823976761\\
34459	0.0998912757058474\\
34460	0.100642409884084\\
34463	0.100331042909423\\
34464	0.100870879701125\\
34465	0.104837039081337\\
34466	0.104573932652833\\
34467	0.104580569572052\\
34470	0.104560791246132\\
34471	0.10140098849299\\
34472	0.101126411865497\\
34473	0.101228266980166\\
34474	0.100609926317031\\
34477	0.101007946149976\\
34478	0.102113193228233\\
34479	0.101178126156036\\
34480	0.113232337352606\\
34481	0.114790642018649\\
34485	0.116534112016107\\
34486	0.116847725337219\\
34487	0.121547405570621\\
34488	0.126733708495524\\
34491	0.125667635542414\\
34492	0.135220214839991\\
34493	0.134969232312664\\
34494	0.135784034056524\\
34495	0.136670474136252\\
34498	0.136468170683483\\
34499	0.13416709960394\\
34500	0.133887681993076\\
34501	0.133737578149743\\
34502	0.134261691988564\\
34505	0.136998588058504\\
34506	0.135416986168856\\
34507	0.134873223239816\\
34508	0.134396502025362\\
34509	0.13175335649371\\
34512	0.158971407576518\\
34513	0.162847161589091\\
34514	0.162859962027853\\
34515	0.164210682345015\\
34519	0.164205057756356\\
34520	0.164212375657467\\
34521	0.163806572894515\\
34522	0.163986843238409\\
34523	0.1617785539575\\
34526	0.16176870806619\\
34527	0.163587576536657\\
34528	0.163384263563617\\
34529	0.163625341619876\\
34530	0.163583862940259\\
34533	0.166735360435992\\
34534	0.16669973525847\\
34535	0.16559371255222\\
34536	0.165673375804286\\
34537	0.164715726344523\\
34540	0.166011246532526\\
34541	0.16600315468279\\
34542	0.165601576143292\\
34543	0.163089097227593\\
34544	0.164998742090112\\
34548	0.165014461193801\\
34549	0.169235649032271\\
34550	0.169139698248083\\
34551	0.169162052276435\\
34554	0.169873694031861\\
34555	0.170510645690965\\
34556	0.170317190950223\\
34557	0.16950601662176\\
34558	0.170003752941292\\
34561	0.161920470213552\\
34562	0.162552742128103\\
34563	0.160105722791456\\
34564	0.159617574387353\\
34565	0.15426855252803\\
34568	0.168307930581925\\
34569	0.179180435595199\\
34570	0.172997123261565\\
34571	0.173601672287778\\
34572	0.172673846554576\\
34576	0.173791418411339\\
34577	0.174299546411185\\
34578	0.174167342366129\\
34579	0.175721899983016\\
34582	0.175526274099313\\
34583	0.175401694102784\\
34584	0.17419593805049\\
34585	0.173493959327855\\
34586	0.174000734978208\\
34589	0.177814476382666\\
34590	0.176397153107553\\
34591	0.151947178694441\\
34592	0.148972874203762\\
34593	0.149978016427549\\
34596	0.152315294613092\\
34597	0.154341507521371\\
34598	0.155938918130118\\
34599	0.160096758000107\\
34600	0.159541408736031\\
34603	0.159272588717182\\
34604	0.159370537119583\\
34605	0.158613516124734\\
34606	0.158763750505576\\
34607	0.159774134862831\\
34610	0.163940264515118\\
34611	0.161335157425788\\
34612	0.161035953255095\\
34613	0.164080453939631\\
34614	0.164819009102693\\
34617	0.165169309292694\\
34618	0.166979125877231\\
34619	0.166963125294408\\
34620	0.1748122044437\\
34621	0.174676735176236\\
34624	0.17377147488899\\
34625	0.173741121890676\\
34626	0.175112360970021\\
34627	0.175994147799721\\
34628	0.176635790687303\\
34631	0.184647152817768\\
34632	0.184467801201225\\
34633	0.189968283972506\\
34634	0.189907235691099\\
34635	0.190200860106554\\
34638	0.199121526557819\\
34639	0.209631097454473\\
34640	0.209681089417166\\
34641	0.209636293708974\\
34642	0.209254940744907\\
34645	0.195742202579249\\
34646	0.187726118187443\\
34647	0.189962440834425\\
34648	0.196138421755748\\
34649	0.196235809041596\\
34652	0.193791943564196\\
34653	0.194662293539616\\
34654	0.208032194434223\\
34655	0.207271803403162\\
34656	0.208374141482616\\
34659	0.208076100531608\\
34660	0.206939275452204\\
34661	0.209939862066035\\
34662	0.233450523632716\\
34663	0.232300585985424\\
34666	0.238360281314161\\
34667	0.244222463344793\\
34668	0.244938001424136\\
34669	0.244589715652685\\
34670	0.242089899214118\\
34673	0.282029095992449\\
34674	0.285174275162262\\
34675	0.283171573739656\\
34676	0.282918609806411\\
34677	0.296068280465027\\
34680	0.318222577118724\\
34681	0.318285840266538\\
34682	0.317776113237994\\
34683	0.31659087873635\\
34684	0.315522192201191\\
};
\addplot [color=mycolor1, forget plot]
  table[row sep=crcr]{%
34445	0.107042659179854\\
34446	0.106533557785013\\
34449	0.107065442989759\\
34450	0.10832780505405\\
34451	0.108099311463929\\
34452	0.107375789550767\\
34453	0.108098364829611\\
34457	0.109199040811553\\
34458	0.111038112272494\\
34459	0.110963206067191\\
34460	0.11128563643733\\
34463	0.110871054067448\\
34464	0.110739000494892\\
34465	0.112499039697996\\
34466	0.112238334297416\\
34467	0.112249371749118\\
34470	0.112464729029595\\
34471	0.105816862055062\\
34472	0.105713360241347\\
34473	0.10516440753693\\
34474	0.10511018641309\\
34477	0.114336038527383\\
34478	0.119071020895356\\
34479	0.119949973113206\\
34480	0.117861871666807\\
34481	0.131777972358528\\
34485	0.140655746140098\\
34486	0.140256701526823\\
34487	0.141368159857347\\
34488	0.158940448900977\\
34491	0.158825908551659\\
34492	0.163985290963694\\
34493	0.163987057932291\\
34494	0.163007606221142\\
34495	0.162796145671219\\
34498	0.162813356648641\\
34499	0.162556540302948\\
34500	0.159638537347011\\
34501	0.159464203941076\\
34502	0.159457095074708\\
34505	0.159636270431218\\
34506	0.158721288679788\\
34507	0.158077868183135\\
34508	0.15785194485263\\
34509	0.155423082211619\\
34512	0.163828165500439\\
34513	0.170588399933373\\
34514	0.170640403965126\\
34515	0.171838222531126\\
34519	0.171861179865099\\
34520	0.171858896741842\\
34521	0.171291808186369\\
34522	0.172751471209363\\
34523	0.171026345838299\\
34526	0.170598383763256\\
34527	0.171159112491165\\
34528	0.171486118468594\\
34529	0.176766808999074\\
34530	0.177989772316876\\
34533	0.184096976981366\\
34534	0.19318285596058\\
34535	0.192462875436726\\
34536	0.191523671956535\\
34537	0.19394236420693\\
34540	0.198025350658092\\
34541	0.198178978539414\\
34542	0.198383458432163\\
34543	0.20005652873057\\
34544	0.20008061439335\\
34548	0.20008061439335\\
34549	0.204225308129087\\
34550	0.204181274034708\\
34551	0.204280018141092\\
34554	0.204192033370785\\
34555	0.204702357394241\\
34556	0.19945467877983\\
34557	0.196792575077018\\
34558	0.196149968982094\\
34561	0.201425701149116\\
34562	0.193286460523837\\
34563	0.18509192891566\\
34564	0.185659519879363\\
34565	0.183393063970951\\
34568	0.170727578008882\\
34569	0.171421384480751\\
34570	0.167447143639482\\
34571	0.167565481740152\\
34572	0.169695140491428\\
34576	0.170020665729295\\
34577	0.172138512757731\\
34578	0.172224632113554\\
34579	0.172915494119444\\
34582	0.173464003169732\\
34583	0.174023363331103\\
34584	0.173511636366237\\
34585	0.173022351636232\\
34586	0.176806567373651\\
34589	0.182660841142205\\
34590	0.181772314686454\\
34591	0.173423643817145\\
34592	0.168652117611631\\
34593	0.168568073664064\\
34596	0.166972160357477\\
34597	0.168832115468374\\
34598	0.168827507905955\\
34599	0.181364836101629\\
34600	0.180262650088722\\
34603	0.180907671567143\\
34604	0.191668957323906\\
34605	0.19085285081516\\
34606	0.191207233980511\\
34607	0.187948377183135\\
34610	0.192748294136387\\
34611	0.19229435620672\\
34612	0.19658141141272\\
34613	0.196841443299469\\
34614	0.196314292030984\\
34617	0.195317682365331\\
34618	0.192929237389037\\
34619	0.192886380488559\\
34620	0.206309012404113\\
34621	0.210887175802618\\
34624	0.211871331734627\\
34625	0.21303590422235\\
34626	0.220458824158943\\
34627	0.220592896330785\\
34628	0.22576591957787\\
34631	0.226783973017496\\
34632	0.225643959366849\\
34633	0.242968011089098\\
34634	0.246148044727798\\
34635	0.247092988817442\\
34638	0.245295401738211\\
34639	0.255717185788509\\
34640	0.256221596121565\\
34641	0.257561627609887\\
34642	0.262960524540806\\
34645	0.268641239298647\\
34646	0.286555105710857\\
34647	0.286786812793535\\
34648	0.295043895994036\\
34649	0.293210975105722\\
34652	0.295209678242939\\
34653	0.294798247183658\\
34654	0.301975293500341\\
34655	0.302619489978552\\
34656	0.302738655361081\\
34659	0.303078476750071\\
34660	0.302747306258544\\
34661	0.305489032021407\\
34662	0.336737391797936\\
34663	0.335170587814884\\
34666	0.343755036271534\\
34667	0.358324087784352\\
34668	0.359219791932772\\
34669	0.360038125338482\\
34670	0.383516437858686\\
34673	0.392792472347223\\
34674	0.39269427170889\\
34675	0.387957458948806\\
34676	0.387276143135207\\
34677	0.386052824417349\\
34680	0.382007735746285\\
34681	0.382007735746285\\
34682	0.381132450989362\\
34683	0.379588350298199\\
34684	0.376962467893633\\
};
\addplot [color=mycolor1, forget plot]
  table[row sep=crcr]{%
34445	0.122936293944083\\
34446	0.121856720551009\\
34449	0.12314211820313\\
34450	0.124463905063513\\
34451	0.124052998819532\\
34452	0.123093389812357\\
34453	0.123911430673423\\
34457	0.125332296725819\\
34458	0.127533366391486\\
34459	0.128905630454604\\
34460	0.133131671405301\\
34463	0.132696684113264\\
34464	0.132537153513513\\
34465	0.134842158574363\\
34466	0.134523889096654\\
34467	0.134934011454589\\
34470	0.136282706754203\\
34471	0.136090483800742\\
34472	0.136201804945091\\
34473	0.135737513550785\\
34474	0.135804889957175\\
34477	0.135820198514141\\
34478	0.137013826209524\\
34479	0.138814585934942\\
34480	0.139662871986593\\
34481	0.143688849939104\\
34485	0.14238455817329\\
34486	0.142382268637151\\
34487	0.142720648757552\\
34488	0.141854180644778\\
34491	0.157657006916842\\
34492	0.160896522972274\\
34493	0.160333637818956\\
34494	0.16220109004637\\
34495	0.164219017228607\\
34498	0.170507355196619\\
34499	0.169209416988558\\
34500	0.16876244806348\\
34501	0.168379801470523\\
34502	0.167016810436295\\
34505	0.167971138149294\\
34506	0.16616522129866\\
34507	0.166592466734542\\
34508	0.166299019718473\\
34509	0.164284651630986\\
34512	0.178671447298581\\
34513	0.192761036987757\\
34514	0.19310054959843\\
34515	0.194996146602166\\
34519	0.195440993737683\\
34520	0.195447577377255\\
34521	0.194790039138651\\
34522	0.196372433451584\\
34523	0.198635130071994\\
34526	0.197961664581552\\
34527	0.200221684968712\\
34528	0.200840978024499\\
34529	0.203251385947246\\
34530	0.208750428845285\\
34533	0.209707361238819\\
34534	0.20953054864954\\
34535	0.208497127876536\\
34536	0.20733125978394\\
34537	0.209232567777026\\
34540	0.209426204193502\\
34541	0.209466469277539\\
34542	0.209831713391185\\
34543	0.212622894924844\\
34544	0.212795408463899\\
34548	0.212552551999932\\
34549	0.218101854455588\\
34550	0.218215303332515\\
34551	0.218589078152147\\
34554	0.218451410220118\\
34555	0.219078389089847\\
34556	0.219364300600887\\
34557	0.218863491243523\\
34558	0.218141232466293\\
34561	0.217433108752017\\
34562	0.215856306741122\\
34563	0.215849177003042\\
34564	0.216948501122065\\
34565	0.216646886938922\\
34568	0.212407888635623\\
34569	0.203865929723346\\
34570	0.200236478926764\\
34571	0.201356790494443\\
34572	0.199654173287292\\
34576	0.198712734274299\\
34577	0.190376076926768\\
34578	0.187380720023191\\
34579	0.190843591809914\\
34582	0.191241336344434\\
34583	0.191526778805791\\
34584	0.192459957117035\\
34585	0.192385343019626\\
34586	0.194280140602219\\
34589	0.205093240040263\\
34590	0.202430247743152\\
34591	0.189226191856535\\
34592	0.181085378925684\\
34593	0.180808915975971\\
34596	0.206895056441016\\
34597	0.207315244904685\\
34598	0.224466926143553\\
34599	0.255947945650506\\
34600	0.253604038869471\\
34603	0.252439335880294\\
34604	0.26241093411988\\
34605	0.260850260698759\\
34606	0.260392818579152\\
34607	0.260715013281039\\
34610	0.264472110237058\\
34611	0.263809953642524\\
34612	0.264203280515727\\
34613	0.265429908797092\\
34614	0.274305047512668\\
34617	0.276830428811487\\
34618	0.283795263378607\\
34619	0.290890297842091\\
34620	0.297635964945518\\
34621	0.295707492864275\\
34624	0.295980525113729\\
34625	0.296172845563481\\
34626	0.292325148985471\\
34627	0.310894551806184\\
34628	0.31640890968084\\
34631	0.348376469242322\\
34632	0.352677305924765\\
34633	0.425485867828896\\
34634	0.433085305955821\\
34635	0.513688192930814\\
34638	0.534729784336677\\
34639	0.533708891837057\\
34640	0.534340770810786\\
34641	0.533634492906885\\
34642	0.535895993565041\\
34645	0.535892937229557\\
34646	0.556831213514574\\
34647	0.558722498116071\\
34648	0.573583362537249\\
34649	0.573594105507444\\
34652	0.575295431827696\\
34653	0.581359386459127\\
34654	0.645802994813998\\
34655	0.646265862296211\\
34656	0.646249407691892\\
34659	0.646322793652837\\
34660	0.646053560329622\\
34661	0.666405565252312\\
34662	0.666755999946239\\
34663	0.664422072476299\\
34666	0.664422072476299\\
34667	0.664228549496813\\
34668	0.670497565510376\\
34669	0.67048648287355\\
34670	0.663035394063081\\
34673	0.662668493698527\\
34674	0.656353931479053\\
34675	0.645089303447499\\
34676	0.645041865673206\\
34677	0.644867825274484\\
34680	0.640931902996757\\
34681	0.640828665604028\\
34682	0.640531169991515\\
34683	0.639386462447955\\
34684	0.63558938049213\\
};
\addplot [color=mycolor1, forget plot]
  table[row sep=crcr]{%
34445	0.140861969455864\\
34446	0.137971243147071\\
34449	0.137888638198972\\
34450	0.13813267091274\\
34451	0.139117579039872\\
34452	0.139081992469291\\
34453	0.140012048786298\\
34457	0.140299881832513\\
34458	0.139615427254911\\
34459	0.140165137839021\\
34460	0.142265425618046\\
34463	0.141675553818697\\
34464	0.141373185899863\\
34465	0.13940309472507\\
34466	0.137892090294041\\
34467	0.138376023326717\\
34470	0.138839113305\\
34471	0.132763891037553\\
34472	0.132564204560395\\
34473	0.130006464700337\\
34474	0.12683202462771\\
34477	0.126233109801286\\
34478	0.126841782728903\\
34479	0.126429121551776\\
34480	0.129321354284991\\
34481	0.128519473058973\\
34485	0.129102638103371\\
34486	0.129270604401216\\
34487	0.132352924472529\\
34488	0.135348051197859\\
34491	0.130012224126522\\
34492	0.127042428909819\\
34493	0.124267292558191\\
34494	0.122554040638729\\
34495	0.122655953700015\\
34498	0.123919313385787\\
34499	0.121466674217309\\
34500	0.120888852737948\\
34501	0.120842600413965\\
34502	0.12186992758418\\
34505	0.121779747388786\\
34506	0.120057577515483\\
34507	0.121290606023601\\
34508	0.119653083121425\\
34509	0.107761679768063\\
34512	0.110130963599777\\
34513	0.110173741195814\\
34514	0.11067725812044\\
34515	0.109334133605779\\
34519	0.113374650289963\\
34520	0.11340012062692\\
34521	0.111204442133167\\
34522	0.111774277902699\\
34523	0.107493738348147\\
34526	0.106575608067631\\
34527	0.107322933237499\\
34528	0.107452823081119\\
34529	0.109642085757969\\
34530	0.108169981846252\\
34533	0.113814802510614\\
34534	0.112945246376047\\
34535	0.111910891796874\\
34536	0.117803922394399\\
34537	0.119972875028494\\
34540	0.122957121407066\\
34541	0.123020158512287\\
34542	0.123835801921658\\
34543	0.126817503551953\\
34544	0.12689867145509\\
34548	0.126472779729533\\
34549	0.134671940797547\\
34550	0.13463836715368\\
34551	0.134990635711648\\
34554	0.13476607669385\\
34555	0.136386293109057\\
34556	0.136822691620667\\
34557	0.135565679883188\\
34558	0.136025370517021\\
34561	0.135937391436471\\
34562	0.142967293615939\\
34563	0.142532230566215\\
34564	0.143932989957048\\
34565	0.141619398768417\\
34568	0.139718486038101\\
34569	0.141856699271098\\
34570	0.141837510702262\\
34571	0.144107295368864\\
34572	0.144219538890081\\
34576	0.149093354315709\\
34577	0.149192655755077\\
34578	0.149192655755077\\
34579	0.156042456812735\\
34582	0.155832013989224\\
34583	0.15768846679546\\
34584	0.15792417786789\\
34585	0.156697153149425\\
34586	0.157038465297057\\
34589	0.165051383543641\\
34590	0.164479732385547\\
34591	0.16277869592485\\
34592	0.165688586163616\\
34593	0.165865895762514\\
34596	0.165411504446264\\
34597	0.168028107761921\\
34598	0.170186005693842\\
34599	0.170089784899783\\
34600	0.169305315186024\\
34603	0.169265781950489\\
34604	0.170034725844773\\
34605	0.169423478765734\\
34606	0.168870449794691\\
34607	0.168330357212649\\
34610	0.175603991430998\\
34611	0.176180077628268\\
34612	0.176309580475499\\
34613	0.180560092166602\\
34614	0.176941967531732\\
34617	0.176561501227055\\
34618	0.173219242155352\\
34619	0.173290187805049\\
34620	0.185185174455773\\
34621	0.192495899220322\\
34624	0.195502761995132\\
34625	0.196182164008431\\
34626	0.19336850614669\\
34627	0.194954794186001\\
34628	0.194726757695434\\
34631	0.197833548977225\\
34632	0.19627435634962\\
34633	0.224532717542729\\
34634	0.232669780207002\\
34635	0.236959423068116\\
34638	0.234982709521951\\
34639	0.236892067467205\\
34640	0.236754187996838\\
34641	0.235694213921818\\
34642	0.23426050068175\\
34645	0.23699658189686\\
34646	0.238804316335185\\
34647	0.239788896973262\\
34648	0.245503495854837\\
34649	0.24458244367922\\
34652	0.240626333758056\\
34653	0.240510473326688\\
34654	0.266423028391274\\
34655	0.262901609971844\\
34656	0.263149433279602\\
34659	0.261489131688703\\
34660	0.261269069988659\\
34661	0.26442846955283\\
34662	0.282399226648667\\
34663	0.280014887978652\\
34666	0.281057740357293\\
34667	0.289923823170136\\
34668	0.288905194827961\\
34669	0.30280732535742\\
34670	0.305963076551789\\
34673	0.303767937997764\\
34674	0.302520131686311\\
34675	0.302729563972445\\
34676	0.3032477742271\\
34677	0.305769670604133\\
34680	0.305204588366434\\
34681	0.309284309380946\\
34682	0.315231864367346\\
34683	0.315030329565117\\
34684	0.484253519621706\\
};
\addplot [color=mycolor1, forget plot]
  table[row sep=crcr]{%
34445	0.128447192569236\\
34446	0.12517913216108\\
34449	0.124647425626819\\
34450	0.124970410310271\\
34451	0.125237347997858\\
34452	0.125322286889506\\
34453	0.126411065595553\\
34457	0.126473006002965\\
34458	0.126279375782075\\
34459	0.127408033246782\\
34460	0.130084223943103\\
34463	0.129284513584441\\
34464	0.129054302385773\\
34465	0.130608634631479\\
34466	0.129556296996018\\
34467	0.129901202463343\\
34470	0.130332521902543\\
34471	0.124319721071464\\
34472	0.124094072852765\\
34473	0.123229376345846\\
34474	0.122300518262654\\
34477	0.122103022052142\\
34478	0.121906288435159\\
34479	0.121961605878538\\
34480	0.116124451067934\\
34481	0.121577684381736\\
34485	0.123267530444428\\
34486	0.124775813001753\\
34487	0.12394081926298\\
34488	0.119142037317239\\
34491	0.118396095930058\\
34492	0.11864817675587\\
34493	0.118563930076032\\
34494	0.118528902123038\\
34495	0.118143947588755\\
34498	0.118020148955299\\
34499	0.115280213224685\\
34500	0.114750864478893\\
34501	0.115965387691551\\
34502	0.11649162529259\\
34505	0.1169005320101\\
34506	0.115353438806547\\
34507	0.116517075219282\\
34508	0.114883292025138\\
34509	0.102818714819003\\
34512	0.104168989324811\\
34513	0.104403708583267\\
34514	0.105927153067221\\
34515	0.105725884065528\\
34519	0.105316907103782\\
34520	0.105325177919243\\
34521	0.102972519592929\\
34522	0.103140643151487\\
34523	0.104636756560659\\
34526	0.103697946569781\\
34527	0.104327537013985\\
34528	0.104552700948007\\
34529	0.10465424189879\\
34530	0.112251035092811\\
34533	0.112419465531234\\
34534	0.111598198371466\\
34535	0.113983872100714\\
34536	0.113149011329872\\
34537	0.112581674472277\\
34540	0.11531914181303\\
34541	0.115088073716866\\
34542	0.115538990873619\\
34543	0.114856357235055\\
34544	0.114781771197942\\
34548	0.114434483691885\\
34549	0.122001579306765\\
34550	0.122650551918913\\
34551	0.123528626964933\\
34554	0.123332216687049\\
34555	0.12487523562252\\
34556	0.125062825336537\\
34557	0.12674841846934\\
34558	0.126637702623972\\
34561	0.130378709477521\\
34562	0.127939688174973\\
34563	0.126938948130869\\
34564	0.129806750410596\\
34565	0.130720151154297\\
34568	0.134585064563773\\
34569	0.134560294044935\\
34570	0.134488102810885\\
34571	0.135542236116253\\
34572	0.134634794891745\\
34576	0.141174236205725\\
34577	0.141174236205725\\
34578	0.142556816781185\\
34579	0.146760257751686\\
34582	0.145283500840535\\
34583	0.147540543721731\\
34584	0.147089526639739\\
34585	0.146039098728954\\
34586	0.1505630123756\\
34589	0.15360803233837\\
34590	0.153116050569799\\
34591	0.15382252192683\\
34592	0.153335573947039\\
34593	0.153904803268549\\
34596	0.153369336127154\\
34597	0.153548686281728\\
34598	0.156264148725824\\
34599	0.155971993951054\\
34600	0.154501812222582\\
34603	0.153749329681636\\
34604	0.154150218603257\\
34605	0.153931985824845\\
34606	0.153719614434483\\
34607	0.1541011958229\\
34610	0.150620203782361\\
34611	0.151172211633917\\
34612	0.151173140608301\\
34613	0.152539592063129\\
34614	0.152844612362871\\
34617	0.152000122871684\\
34618	0.158560409153006\\
34619	0.162615641074206\\
34620	0.162674725487435\\
34621	0.166969163172004\\
34624	0.167998557296562\\
34625	0.168006419464805\\
34626	0.165539796029954\\
34627	0.166692986873664\\
34628	0.165753901018435\\
34631	0.167608625998049\\
34632	0.16584157378501\\
34633	0.165556269168114\\
34634	0.164495554820914\\
34635	0.165012198152281\\
34638	0.162197840330744\\
34639	0.164295492765722\\
34640	0.163826638924675\\
34641	0.159455212619924\\
34642	0.160746665321811\\
34645	0.157996560372076\\
34646	0.165343682659843\\
34647	0.164760600549784\\
34648	0.169072138270363\\
34649	0.168874045774796\\
34652	0.162428457580215\\
34653	0.162300541293337\\
34654	0.177727123410947\\
34655	0.175924344838217\\
34656	0.179239401145376\\
34659	0.177675971237736\\
34660	0.177928625409535\\
34661	0.184832602615242\\
34662	0.198483286844333\\
34663	0.197369261104069\\
34666	0.197756069082731\\
34667	0.204584825938911\\
34668	0.20482048678747\\
34669	0.204932915538549\\
34670	0.206856117449073\\
34673	0.221266098875669\\
34674	0.219421664599916\\
34675	0.219401967224559\\
34676	0.221646567545051\\
34677	0.230236567942418\\
34680	0.231069767130752\\
34681	0.232907182715043\\
34682	0.233441459788781\\
34683	0.237689524612705\\
34684	0.247829634565885\\
};
\addplot [color=mycolor1, forget plot]
  table[row sep=crcr]{%
34445	0.135122162971083\\
34446	0.134004744149523\\
34449	0.128391967154147\\
34450	0.127108715094762\\
34451	0.111238675174783\\
34452	0.101914473524128\\
34453	0.1027832080145\\
34457	0.10273121079556\\
34458	0.102864781765792\\
34459	0.102866829404697\\
34460	0.102164863544697\\
34463	0.101680352322503\\
34464	0.102601588771605\\
34465	0.103553774024908\\
34466	0.102454211942319\\
34467	0.103248794008649\\
34470	0.104235180467968\\
34471	0.103133133473262\\
34472	0.102890933099647\\
34473	0.103030862383496\\
34474	0.103674876526637\\
34477	0.103115732966485\\
34478	0.103162323795418\\
34479	0.101720541111673\\
34480	0.109214592443551\\
34481	0.10797851017962\\
34485	0.111835182321178\\
34486	0.112388943423011\\
34487	0.111249281127075\\
34488	0.108689724623361\\
34491	0.106906714396958\\
34492	0.107101816405055\\
34493	0.106532056153382\\
34494	0.105955084185773\\
34495	0.105336401747206\\
34498	0.105475911869107\\
34499	0.103765514585814\\
34500	0.103375389559048\\
34501	0.103340381296617\\
34502	0.103350945903576\\
34505	0.10327085415462\\
34506	0.103354979361532\\
34507	0.104235737734833\\
34508	0.102625588826506\\
34509	0.0914297546809913\\
34512	0.0944285430585683\\
34513	0.0944059956639355\\
34514	0.0959872726513108\\
34515	0.0959042139654157\\
34519	0.0955202225606502\\
34520	0.0955339741831239\\
34521	0.0932224066434886\\
34522	0.0940454193335673\\
34523	0.0928591836371171\\
34526	0.0919927738823886\\
34527	0.0923965396068478\\
34528	0.0924708373244505\\
34529	0.094314222879447\\
34530	0.0974085227859193\\
34533	0.0979021226278938\\
34534	0.0974496733936424\\
34535	0.0961615833231468\\
34536	0.0956833218876597\\
34537	0.0978933249343165\\
34540	0.0986291712094418\\
34541	0.0983171563590212\\
34542	0.0985008506753029\\
34543	0.0971186852355326\\
34544	0.0979383643481901\\
34548	0.0969124557376118\\
34549	0.103171301173592\\
34550	0.114997399316308\\
34551	0.127778074565687\\
34554	0.127603286569068\\
34555	0.128879855879512\\
34556	0.12933648738872\\
34557	0.129932403407399\\
34558	0.129848332766918\\
34561	0.123227875021149\\
34562	0.125609936255484\\
34563	0.121426586013111\\
34564	0.122267642323022\\
34565	0.122118565827615\\
34568	0.122401346027172\\
34569	0.122508351680972\\
34570	0.122095704642945\\
34571	0.122096703308691\\
34572	0.125432396734692\\
34576	0.125455879160491\\
34577	0.125619197642906\\
34578	0.127301951412153\\
34579	0.131760843309285\\
34582	0.131600632966762\\
34583	0.134206614437374\\
34584	0.134438262766418\\
34585	0.132085443044247\\
34586	0.133095717952115\\
34589	0.140591869993726\\
34590	0.14036817560641\\
34591	0.140066082057237\\
34592	0.140043503832342\\
34593	0.140168172532372\\
34596	0.13967798588121\\
34597	0.139937166523544\\
34598	0.141743045343603\\
34599	0.142470836271469\\
34600	0.141978332040398\\
34603	0.141779824537149\\
34604	0.142300927592129\\
34605	0.141921980646534\\
34606	0.141555578639292\\
34607	0.140840545407484\\
34610	0.140137673466658\\
34611	0.140220251413034\\
34612	0.140117648112194\\
34613	0.142781254527566\\
34614	0.14292245654458\\
34617	0.142647833441281\\
34618	0.144290372667335\\
34619	0.147030051582776\\
34620	0.146759196474421\\
34621	0.146493240412139\\
34624	0.146224524248758\\
34625	0.146870456065335\\
34626	0.144977120343031\\
34627	0.135083225925395\\
34628	0.123800241457634\\
34631	0.127518114586678\\
34632	0.125597274815648\\
34633	0.126198823494502\\
34634	0.12621391991746\\
34635	0.126878939820887\\
34638	0.133514283961347\\
34639	0.131668018858339\\
34640	0.13205514998161\\
34641	0.131294247983055\\
34642	0.131890894229311\\
34645	0.133811138902943\\
34646	0.136184859868551\\
34647	0.137451184105027\\
34648	0.14241599390058\\
34649	0.138089775154391\\
34652	0.138100718431764\\
34653	0.13806933877459\\
34654	0.149739568783796\\
34655	0.148880922480211\\
34656	0.148901604084496\\
34659	0.146862793148535\\
34660	0.147368455371547\\
34661	0.151570847197045\\
34662	0.162112430762706\\
34663	0.157603419314665\\
34666	0.159251559711121\\
34667	0.158003035438401\\
34668	0.164399590666996\\
34669	0.16641691734395\\
34670	0.171799724359557\\
34673	0.173410750715651\\
34674	0.172075526435055\\
34675	0.17147584361055\\
34676	0.171426323872609\\
34677	0.17197831373501\\
34680	0.173934797707262\\
34681	0.17458879872614\\
34682	0.174918545455745\\
34683	0.175007504550031\\
34684	0.174398834963178\\
};
\addplot [color=mycolor1, forget plot]
  table[row sep=crcr]{%
34445	0.105295147516824\\
34446	0.10528927495403\\
34449	0.102934116864591\\
34450	0.103177646475392\\
34451	0.103281533698424\\
34452	0.103442698735902\\
34453	0.104142912374043\\
34457	0.10132847050693\\
34458	0.101623226829133\\
34459	0.10174563550461\\
34460	0.101390949852296\\
34463	0.100631599652824\\
34464	0.0994193945186129\\
34465	0.101016783794983\\
34466	0.100686538950605\\
34467	0.101076592454682\\
34470	0.101444218445193\\
34471	0.0953403277857434\\
34472	0.0950340310286378\\
34473	0.0939675500167245\\
34474	0.0928297595649872\\
34477	0.0922996750328848\\
34478	0.0929589705915577\\
34479	0.0928420163036279\\
34480	0.0950563269608821\\
34481	0.0937030340455547\\
34485	0.0970634578385238\\
34486	0.0970608803055349\\
34487	0.0946078182639891\\
34488	0.0919180511376127\\
34491	0.091758101629777\\
34492	0.0922557728013692\\
34493	0.0904837724986276\\
34494	0.0892525603213681\\
34495	0.0885749005621457\\
34498	0.0880861338736255\\
34499	0.0867142647868936\\
34500	0.0858834958155084\\
34501	0.0859964761603776\\
34502	0.08601319970986\\
34505	0.086285723399322\\
34506	0.0846510386089259\\
34507	0.0852764725967603\\
34508	0.0851734717612307\\
34509	0.0823758805740134\\
34512	0.0850918554182353\\
34513	0.0866147260342318\\
34514	0.0866000776266398\\
34515	0.0863542538711539\\
34519	0.0859543605363595\\
34520	0.0857994963317297\\
34521	0.0826195720943574\\
34522	0.0816608267922821\\
34523	0.0821885457059329\\
34526	0.0814801691254598\\
34527	0.0817337524622451\\
34528	0.0816892560858067\\
34529	0.0820764979337188\\
34530	0.0857755576412173\\
34533	0.0862422693866347\\
34534	0.086219601439823\\
34535	0.0865272422466049\\
34536	0.0859215953314812\\
34537	0.0868605705634528\\
34540	0.0885156284610569\\
34541	0.0885808044160594\\
34542	0.0890841616459233\\
34543	0.0878137157874539\\
34544	0.0883590543383245\\
34548	0.0879976179445056\\
34549	0.0930231490081552\\
34550	0.0926303953242686\\
34551	0.0923956784114536\\
34554	0.0922457107354172\\
34555	0.0934426520726042\\
34556	0.0938757881992905\\
34557	0.0930066520123209\\
34558	0.0941163032698014\\
34561	0.0871172456153535\\
34562	0.0892556171753646\\
34563	0.0846883454359354\\
34564	0.085538956326607\\
34565	0.0862748471400731\\
34568	0.0859746360999694\\
34569	0.0862882865321761\\
34570	0.0859065197373477\\
34571	0.087705461313158\\
34572	0.087705461313158\\
34576	0.091857340690219\\
34577	0.0928284770945214\\
34578	0.092751830864273\\
34579	0.0965402428038283\\
34582	0.0964555317990484\\
34583	0.0972114491732267\\
34584	0.10167133488464\\
34585	0.100850710234592\\
34586	0.101943551403408\\
34589	0.107414666742547\\
34590	0.107377871798795\\
34591	0.104767295857857\\
34592	0.106357930943537\\
34593	0.13017553623565\\
34596	0.130499361151428\\
34597	0.145942612832039\\
34598	0.147151989402667\\
34599	0.14756727746138\\
34600	0.147164960940732\\
34603	0.147007893867859\\
34604	0.147897507312161\\
34605	0.147627939741359\\
34606	0.147420456850987\\
34607	0.147325748369996\\
34610	0.146055226438334\\
34611	0.146032127007178\\
34612	0.145720328440555\\
34613	0.146161288955995\\
34614	0.146095443100326\\
34617	0.145885082545019\\
34618	0.145347368227713\\
34619	0.147637119260783\\
34620	0.147483144678772\\
34621	0.151577910783292\\
34624	0.152159523536698\\
34625	0.152161995497332\\
34626	0.150639075980385\\
34627	0.151695748783811\\
34628	0.151690615393993\\
34631	0.152559598347469\\
34632	0.15142902795545\\
34633	0.151603876743213\\
34634	0.151878781985681\\
34635	0.151699851310935\\
34638	0.155582005570045\\
34639	0.154728599513737\\
34640	0.155120869945969\\
34641	0.154475390752696\\
34642	0.155133747427233\\
34645	0.155102820845323\\
34646	0.158989519298352\\
34647	0.158903502842452\\
34648	0.160480008397309\\
34649	0.159751994467865\\
34652	0.156814731463941\\
34653	0.156617448798056\\
34654	0.167340993938464\\
34655	0.167043408802953\\
34656	0.168275644824928\\
34659	0.167528779624383\\
34660	0.165531926438213\\
34661	0.167046157594928\\
34662	0.172965451552561\\
34663	0.170288368275746\\
34666	0.170324330454528\\
34667	0.172091887026179\\
34668	0.171115255533502\\
34669	0.154585431242374\\
34670	0.15990690536674\\
34673	0.148935672539153\\
34674	0.148235984178639\\
34675	0.147573374313955\\
34676	0.147409570035387\\
34677	0.147837803917007\\
34680	0.14808451017549\\
34681	0.14838560253203\\
34682	0.148252754224341\\
34683	0.148037733075207\\
34684	0.148173228184642\\
};
\addplot [color=mycolor1, forget plot]
  table[row sep=crcr]{%
34445	0.0942285275391563\\
34446	0.0946695298698276\\
34449	0.0942844257386956\\
34450	0.0945459033055612\\
34451	0.09405632408175\\
34452	0.094156135380637\\
34453	0.0948255842301286\\
34457	0.0950421653080748\\
34458	0.0948804372517508\\
34459	0.0945148968132641\\
34460	0.0929882819263119\\
34463	0.0922223262262946\\
34464	0.0919959239980023\\
34465	0.0912704425791225\\
34466	0.0902009766132746\\
34467	0.0905049534515547\\
34470	0.0907245804068483\\
34471	0.0849634784958581\\
34472	0.0847199216530439\\
34473	0.0836335675545596\\
34474	0.0826319534155587\\
34477	0.082073073497786\\
34478	0.082721009092431\\
34479	0.0826488301891427\\
34480	0.084550580320849\\
34481	0.0835304666674304\\
34485	0.0863424992443292\\
34486	0.0866596100204507\\
34487	0.085545120126562\\
34488	0.0854386262909233\\
34491	0.0809603629157762\\
34492	0.0809447276260121\\
34493	0.0803521045171297\\
34494	0.0790694568322296\\
34495	0.0788408271494341\\
34498	0.0788323596798869\\
34499	0.0764139458542382\\
34500	0.0760667207511389\\
34501	0.0759539544275271\\
34502	0.0759658793219674\\
34505	0.0761579696344753\\
34506	0.0743308829652097\\
34507	0.0748397570551816\\
34508	0.0747176502498514\\
34509	0.0721318799765361\\
34512	0.0745024498438172\\
34513	0.0758209878071082\\
34514	0.0758209878071082\\
34515	0.0755808650940404\\
34519	0.075233295719195\\
34520	0.0752268244396544\\
34521	0.0731630626628282\\
34522	0.072273957588329\\
34523	0.0726905754193833\\
34526	0.071974461050516\\
34527	0.0721019475732753\\
34528	0.0721103040245669\\
34529	0.0731853349945187\\
34530	0.075077731841669\\
34533	0.0754021158977134\\
34534	0.074892435031162\\
34535	0.0737025253175571\\
34536	0.0730256605562514\\
34537	0.07377274881152\\
34540	0.0751928616437888\\
34541	0.0752547079921039\\
34542	0.0756209733027333\\
34543	0.0764437982783207\\
34544	0.0764004566256149\\
34548	0.0761089150646779\\
34549	0.0801869783307044\\
34550	0.0802057400338801\\
34551	0.0803636160324672\\
34554	0.0802328134275437\\
34555	0.0810821493152313\\
34556	0.0814321974166022\\
34557	0.080587346851984\\
34558	0.081587238406108\\
34561	0.0753513764940635\\
34562	0.077155715661269\\
34563	0.0731953362125276\\
34564	0.0739225295504414\\
34565	0.0736977917259019\\
34568	0.0732397186639647\\
34569	0.0744596937062879\\
34570	0.074080875368968\\
34571	0.0756960437971888\\
34572	0.0756710323334793\\
34576	0.0791655567410002\\
34577	0.0802777771769744\\
34578	0.0802000928356719\\
34579	0.084240238835885\\
34582	0.0841396701665217\\
34583	0.0847276187573746\\
34584	0.0885342178573014\\
34585	0.0878561327152741\\
34586	0.0886914711798375\\
34589	0.09314666863508\\
34590	0.0927765488966985\\
34591	0.0915987614649963\\
34592	0.0925567237544868\\
34593	0.0951320757988151\\
34596	0.0951294303148253\\
34597	0.0961477458244241\\
34598	0.0974746162331689\\
34599	0.0978593700921518\\
34600	0.0973965825058556\\
34603	0.0972316974312602\\
34604	0.0982393468733106\\
34605	0.0979678577697118\\
34606	0.0976966159921775\\
34607	0.0969790850886073\\
34610	0.0962252626775857\\
34611	0.0961866609825206\\
34612	0.0961432149692544\\
34613	0.0974065827888157\\
34614	0.0976102565879802\\
34617	0.0968617740799557\\
34618	0.0958889495774806\\
34619	0.0981998102554365\\
34620	0.0981631287378393\\
34621	0.100126690765114\\
34624	0.103164663015396\\
34625	0.103591795287673\\
34626	0.102044037917808\\
34627	0.102547196507862\\
34628	0.102337813762863\\
34631	0.10316986787873\\
34632	0.102080835143294\\
34633	0.102391265926411\\
34634	0.102386438138231\\
34635	0.104070949247064\\
34638	0.107356322137624\\
34639	0.106092473791389\\
34640	0.10626168838635\\
34641	0.105468076885802\\
34642	0.105851981927224\\
34645	0.105807753968018\\
34646	0.108742768340866\\
34647	0.108655385354032\\
34648	0.109597257849317\\
34649	0.108801780589446\\
34652	0.105589790742507\\
34653	0.105220332211619\\
34654	0.11469915306997\\
34655	0.112890734993029\\
34656	0.113347195654395\\
34659	0.11259721002109\\
34660	0.116670462618997\\
34661	0.117464816242956\\
34662	0.122349475928183\\
34663	0.119707389234388\\
34666	0.119898410018574\\
34667	0.120474728961906\\
34668	0.119614711976674\\
34669	0.118168016887737\\
34670	0.12119899309474\\
34673	0.121624283388701\\
34674	0.120888633898983\\
34675	0.120350070153004\\
34676	0.120241530392296\\
34677	0.120299012306041\\
34680	0.119556556748827\\
34681	0.121007792869988\\
34682	0.120978115306683\\
34683	0.121963312975708\\
34684	0.121212235691791\\
};
\end{axis}

\begin{axis}[%
width=3.888in,
height=1.366in,
at={(0.703in,5.053in)},
scale only axis,
xmin=34445,
xmax=34684,
xtick={34425,34455,34486,34516,34547,34578,34608,34639,34669,34700},
xticklabels={{01/04},{01/05},{01/06},{01/07},{01/08},{01/09},{01/10},{01/11},{01/12},{01/01}},
ymin=4.16923925575053e-51,
ymax=1002.33198361266,
ylabel style={font=\color{white!15!black}},
ylabel={Value},
axis background/.style={fill=white},
title style={font=\bfseries},
title={$\nu\text{: Sensitivity to Underlying Price Volatility}$},
xmajorgrids,
ymajorgrids
]
\addplot [color=mycolor2, forget plot]
  table[row sep=crcr]{%
34445	232.600758402266\\
34446	166.790238545949\\
34449	218.651166020614\\
34450	189.626923044844\\
34451	149.045729506603\\
34452	175.110147063358\\
34453	187.75143748538\\
34457	239.905524165726\\
34458	319.403842370025\\
34459	241.633837696969\\
34460	255.127038470379\\
34463	266.204618002768\\
34464	196.0355487223\\
34465	209.157369009484\\
34466	190.100335483626\\
34467	225.781884385351\\
34470	240.732119830631\\
34471	197.972565268162\\
34472	209.442657500047\\
34473	188.053014364643\\
34474	176.678582019274\\
34477	211.230868962664\\
34478	253.690546313426\\
34479	446.001465534496\\
34480	492.103421381304\\
34481	647.656003654754\\
34485	652.70861406685\\
34486	732.889345642014\\
34487	636.177150039925\\
34488	579.309348616491\\
34491	533.817914977168\\
34492	538.724401588402\\
34493	454.424970570763\\
34494	491.075875524057\\
34495	426.711073469057\\
34498	513.161661843073\\
34499	438.638264300363\\
34500	421.296357455985\\
34501	462.619579358764\\
34502	482.672057151977\\
34505	607.254540945172\\
34506	681.672934024687\\
34507	641.044943031954\\
34508	671.518365394388\\
34509	771.613676225197\\
34512	738.984632152018\\
34513	719.188544266362\\
34514	644.565967274159\\
34515	704.86904265036\\
34519	574.930692008215\\
34520	597.092567801065\\
34521	632.783573095086\\
34522	585.375146059539\\
34523	601.560125714816\\
34526	533.888228135252\\
34527	588.234477841428\\
34528	479.593367399701\\
34529	376.031270295422\\
34530	319.833382066044\\
34533	302.443268950927\\
34534	287.594013602992\\
34535	309.903018045802\\
34536	260.219559469603\\
34537	226.662215975165\\
34540	253.257369751818\\
34541	220.095982519359\\
34542	293.038285035495\\
34543	273.726976203805\\
34544	299.729112446198\\
34548	152.740601964821\\
34549	161.25037630412\\
34550	173.6555914163\\
34551	147.856478891835\\
34554	138.5447091028\\
34555	141.62098080223\\
34556	140.395324831057\\
34557	169.211502569945\\
34558	174.556751091182\\
34561	129.459294314182\\
34562	124.293033086719\\
34563	47.7067484302449\\
34564	54.2184991825972\\
34565	46.12164094394\\
34568	62.172206584043\\
34569	57.7765853280228\\
34570	33.7679811445814\\
34571	19.6427018239658\\
34572	10.6686256529548\\
34576	12.4256413094123\\
34577	12.7425929759203\\
34578	23.5879194660847\\
34579	23.4639708616714\\
34582	15.1494918055543\\
34583	29.1582464083371\\
34584	30.1769499231977\\
34585	43.8882116829176\\
34586	85.4480381472979\\
34589	103.647415958157\\
34590	113.293579774836\\
34591	178.757175548064\\
34592	123.756189054268\\
34593	228.281858856257\\
34596	221.949499349935\\
34597	314.760247213861\\
34598	360.829361609197\\
34599	349.42777590906\\
34600	322.339977708784\\
34603	376.217752247943\\
34604	364.897131117795\\
34605	285.436723297622\\
34606	388.626208427902\\
34607	312.439351330984\\
34610	395.686089711508\\
34611	351.749897424582\\
34612	446.059556367147\\
34613	385.71774962801\\
34614	353.702440556306\\
34617	259.245335172926\\
34618	162.312488194046\\
34619	125.635175330971\\
34620	63.0183222634008\\
34621	114.783148902065\\
34624	92.1010484065834\\
34625	148.403058840005\\
34626	189.455893218086\\
34627	183.619439821309\\
34628	246.184765613605\\
34631	232.345234835211\\
34632	303.335996318955\\
34633	292.06181741724\\
34634	248.470062666588\\
34635	149.474252963223\\
34638	131.980296902824\\
34639	131.128526096507\\
34640	145.004843917524\\
34641	109.173841223415\\
34642	113.017626126832\\
34645	150.212541692046\\
34646	163.544890917177\\
34647	97.8884156170481\\
34648	95.9690924679449\\
34649	133.741209668943\\
34652	97.9051443878059\\
34653	42.9673099209404\\
34654	44.9544995049966\\
34655	57.5812008291238\\
34656	50.522161291785\\
34659	47.2215378853842\\
34660	84.6719161750461\\
34661	162.844642888883\\
34662	151.775042741741\\
34663	149.897371563618\\
34666	104.902233709468\\
34667	81.4620648717427\\
34668	53.015782038184\\
34669	94.6609536767114\\
34670	120.990454491902\\
34673	76.7554731840438\\
34674	93.4866343364755\\
34675	85.5042402574598\\
34676	74.4421941648487\\
34677	118.078651168617\\
34680	125.913617126396\\
34681	107.066868739305\\
34682	46.5388310655155\\
34683	31.2012663217561\\
34684	0.192151285569523\\
};
\addplot [color=mycolor2, forget plot]
  table[row sep=crcr]{%
34445	621.309473063467\\
34446	528.098209446439\\
34449	601.83000095191\\
34450	559.751132763755\\
34451	497.798710994064\\
34452	541.430964285729\\
34453	557.930088482844\\
34457	623.951847655184\\
34458	705.868249167872\\
34459	614.373958682362\\
34460	615.247820571683\\
34463	629.2129034341\\
34464	539.889669098451\\
34465	556.138783661194\\
34466	530.255432837124\\
34467	577.883857334009\\
34470	594.732556491864\\
34471	547.497788697938\\
34472	563.772000188492\\
34473	542.734955067331\\
34474	535.320248052273\\
34477	583.277060799391\\
34478	631.283659627916\\
34479	794.278193835194\\
34480	786.92872589709\\
34481	861.030818322986\\
34485	852.774670393007\\
34486	862.318666924201\\
34487	845.474344093886\\
34488	825.403195785978\\
34491	802.45732480067\\
34492	802.57407738786\\
34493	756.660453762833\\
34494	774.231128861305\\
34495	731.828994570815\\
34498	783.588279844989\\
34499	741.24632273166\\
34500	729.127612109538\\
34501	755.010292923315\\
34502	763.703592619913\\
34505	813.049050712472\\
34506	820.391478731681\\
34507	816.457519773471\\
34508	816.264888052094\\
34509	754.619364495028\\
34512	779.990168501788\\
34513	792.014707403652\\
34514	802.987110628555\\
34515	791.649477915016\\
34519	788.469890549717\\
34520	788.667573212467\\
34521	787.144274366174\\
34522	783.897844581351\\
34523	782.517991119859\\
34526	769.474222413592\\
34527	773.195251668797\\
34528	753.547002643676\\
34529	706.963706609912\\
34530	662.818454776453\\
34533	658.633753806514\\
34534	645.635311042348\\
34535	658.498421407937\\
34536	625.278590273848\\
34537	594.194004901662\\
34540	609.690354157947\\
34541	580.272116862158\\
34542	636.208227972765\\
34543	619.803642730121\\
34544	636.874832084305\\
34548	503.107137190731\\
34549	504.985474202308\\
34550	518.789256529867\\
34551	487.923668633301\\
34554	475.227803103802\\
34555	478.049983600579\\
34556	476.148647841981\\
34557	511.769749066217\\
34558	513.953562704205\\
34561	508.006776162688\\
34562	495.317828910035\\
34563	385.047717924461\\
34564	395.287287935777\\
34565	374.835921861188\\
34568	412.241040349329\\
34569	400.305011469077\\
34570	338.378353585945\\
34571	273.593078298362\\
34572	225.985647622004\\
34576	243.389424537006\\
34577	243.743029722615\\
34578	301.120574323938\\
34579	294.245118856518\\
34582	253.948744531711\\
34583	317.152339194617\\
34584	316.808161326482\\
34585	362.03296128557\\
34586	444.807203556557\\
34589	457.326107842831\\
34590	468.135387703164\\
34591	528.802586067679\\
34592	446.822211393196\\
34593	521.890946861773\\
34596	497.097817857503\\
34597	557.787524370364\\
34598	572.36007564563\\
34599	565.630450637362\\
34600	555.811862961472\\
34603	562.261762792888\\
34604	556.624172351362\\
34605	529.219231974815\\
34606	553.945137911046\\
34607	538.991399732393\\
34610	539.712907900117\\
34611	535.680905734949\\
34612	520.070079734869\\
34613	528.988394067721\\
34614	525.875288278836\\
34617	499.07142256687\\
34618	444.667414832898\\
34619	404.989620078017\\
34620	314.369939485249\\
34621	390.338086144076\\
34624	351.734276949143\\
34625	412.694301470099\\
34626	443.113141048944\\
34627	435.904541136201\\
34628	462.110400591972\\
34631	448.851708492883\\
34632	455.80070686047\\
34633	451.621838513667\\
34634	442.323075387221\\
34635	391.827491757842\\
34638	366.001598394556\\
34639	368.308023181203\\
34640	376.44252243055\\
34641	347.38111616911\\
34642	348.146847120232\\
34645	368.789776451882\\
34646	370.232046364523\\
34647	324.060136626488\\
34648	316.834893099846\\
34649	345.048219984259\\
34652	313.135465716623\\
34653	242.473849936515\\
34654	226.403488661281\\
34655	250.196887965614\\
34656	239.326600285627\\
34659	233.203384316746\\
34660	277.495509600156\\
34661	307.198795310478\\
34662	298.556610548843\\
34663	293.268999117697\\
34666	267.090218590332\\
34667	251.604998377842\\
34668	228.528428151818\\
34669	248.461745696237\\
34670	243.971901934066\\
34673	217.139891598472\\
34674	208.5763096706\\
34675	198.412598900135\\
34676	188.203433771443\\
34677	159.31119001646\\
34680	79.107530886813\\
34681	77.8527255900189\\
34682	98.1901175491687\\
34683	73.0952730497792\\
34684	61.8324616071431\\
};
\addplot [color=mycolor2, forget plot]
  table[row sep=crcr]{%
34445	908.802350301483\\
34446	854.885528234839\\
34449	896.07617642303\\
34450	868.376896150677\\
34451	823.60242283243\\
34452	854.090441819529\\
34453	862.520507464035\\
34457	895.480748263044\\
34458	925.604395117996\\
34459	887.261075929794\\
34460	886.173047741134\\
34463	890.507609745032\\
34464	843.034515649899\\
34465	854.230396540505\\
34466	839.114535924028\\
34467	863.770855392122\\
34470	867.399707579712\\
34471	846.543138545001\\
34472	854.39553173395\\
34473	843.960738099838\\
34474	838.500560577985\\
34477	859.992730012084\\
34478	877.837209472534\\
34479	903.675440051065\\
34480	902.46794955031\\
34481	868.455491206701\\
34485	864.717652038846\\
34486	822.11746185664\\
34487	865.854309585979\\
34488	875.617850238563\\
34491	874.643974920378\\
34492	872.49453285505\\
34493	876.048218473947\\
34494	872.619955752004\\
34495	871.559150127346\\
34498	861.22330357191\\
34499	863.041608456458\\
34500	860.977840830902\\
34501	857.369024656831\\
34502	853.348085676364\\
34505	821.962094778273\\
34506	787.415351378393\\
34507	802.496936884974\\
34508	785.402416525644\\
34509	689.028632288458\\
34512	744.299959766308\\
34513	756.010388023808\\
34514	784.592314358465\\
34515	756.625142143056\\
34519	790.460999988365\\
34520	781.811452451426\\
34521	766.843896546238\\
34522	778.066078283339\\
34523	770.767572447877\\
34526	778.067346818857\\
34527	762.404037460219\\
34528	783.606929808787\\
34529	793.343140794958\\
34530	791.510705603689\\
34533	783.755827659814\\
34534	780.384445948859\\
34535	779.223856216893\\
34536	773.989674924762\\
34537	766.406854284248\\
34540	762.679820108937\\
34541	755.25590811175\\
34542	761.288281563115\\
34543	757.000592865255\\
34544	756.464061196224\\
34548	720.382980010641\\
34549	717.333589132966\\
34550	719.92555618073\\
34551	708.006866488317\\
34554	698.874289278489\\
34555	697.920813944261\\
34556	695.591185406587\\
34557	705.402051305672\\
34558	703.929652257033\\
34561	693.725319231395\\
34562	688.006450186329\\
34563	654.441739956306\\
34564	657.801645719272\\
34565	642.552498798477\\
34568	662.074593477595\\
34569	660.405692193973\\
34570	633.880251222243\\
34571	602.116302299212\\
34572	570.956882345389\\
34576	572.385727374467\\
34577	570.563645319406\\
34578	600.601828930199\\
34579	593.651157523236\\
34582	567.557730504458\\
34583	597.364131483615\\
34584	592.709839480871\\
34585	609.714185420875\\
34586	630.055833237278\\
34589	623.542637119402\\
34590	622.703272103853\\
34591	622.680263413915\\
34592	615.54497964612\\
34593	614.139059773006\\
34596	606.281629157018\\
34597	587.249328091195\\
34598	569.902437135985\\
34599	570.593812021565\\
34600	572.565054157014\\
34603	540.070385650178\\
34604	541.935441971575\\
34605	560.526707041336\\
34606	520.004986545229\\
34607	541.457970835245\\
34610	494.986580224898\\
34611	506.897979260723\\
34612	453.162953172151\\
34613	482.731351679713\\
34614	491.394488493539\\
34617	507.94985510072\\
34618	524.719448977274\\
34619	524.411631351188\\
34620	510.97742411341\\
34621	516.499360146466\\
34624	502.73826497393\\
34625	498.598671123578\\
34626	485.812070375201\\
34627	482.843172091305\\
34628	460.196830706954\\
34631	449.910214354291\\
34632	416.781545605331\\
34633	417.827200563605\\
34634	432.879509800449\\
34635	454.966852780559\\
34638	443.286246074261\\
34639	438.118010640899\\
34640	431.278563923318\\
34641	429.450069951329\\
34642	424.169707458893\\
34645	399.974436064823\\
34646	391.3783335583\\
34647	399.160194790778\\
34648	394.135324647168\\
34649	381.171307284909\\
34652	369.028935521109\\
34653	364.538743937613\\
34654	356.165453695965\\
34655	354.989201324279\\
34656	348.571322093932\\
34659	331.227842292762\\
34660	317.026463496193\\
34661	273.481499493406\\
34662	281.110396361591\\
34663	269.82463636519\\
34666	258.005570360971\\
34667	259.536646728068\\
34668	260.179762281733\\
34669	227.129674094363\\
34670	198.5721006324\\
34673	190.921333721622\\
34674	166.855788827471\\
34675	152.674101952316\\
34676	140.327401964055\\
34677	101.210139792309\\
34680	40.0431654320237\\
34681	28.3614092468215\\
34682	29.1293932931509\\
34683	10.2420282015758\\
34684	5.69681064361962\\
};
\addplot [color=mycolor2, forget plot]
  table[row sep=crcr]{%
34445	1002.3025923741\\
34446	997.438963033527\\
34449	994.903321866492\\
34450	992.492317251868\\
34451	983.936207248388\\
34452	987.923452278478\\
34453	986.85013264665\\
34457	977.262332416707\\
34458	961.731875415985\\
34459	974.309004870228\\
34460	972.307774852702\\
34463	964.178735350442\\
34464	964.533429403089\\
34465	963.051146269106\\
34466	959.738972481434\\
34467	959.241482707391\\
34470	952.340959433492\\
34471	951.088072832021\\
34472	948.687509934942\\
34473	946.946204463577\\
34474	944.860684119354\\
34477	935.674010877583\\
34478	927.058455417854\\
34479	857.407128092467\\
34480	859.737560927058\\
34481	791.14915990859\\
34485	801.223320934766\\
34486	742.406154810338\\
34487	804.244349033024\\
34488	838.692814944781\\
34491	843.945694358072\\
34492	843.941568992449\\
34493	866.009895753392\\
34494	853.732830811478\\
34495	868.668285707598\\
34498	834.119461469806\\
34499	850.654114596737\\
34500	851.163428296783\\
34501	835.567472683156\\
34502	825.893850191817\\
34505	767.081110606626\\
34506	719.050669776411\\
34507	739.162238388913\\
34508	716.876153060324\\
34509	608.742439696023\\
34512	651.79017263517\\
34513	675.827966436458\\
34514	717.782605242156\\
34515	678.641369350097\\
34519	734.675394507839\\
34520	721.188573671925\\
34521	697.92461736668\\
34522	718.652855579573\\
34523	707.123095276221\\
34526	723.792364410338\\
34527	697.872055666724\\
34528	739.947893466384\\
34529	776.734300225691\\
34530	792.384059376954\\
34533	788.233156933229\\
34534	789.564550987187\\
34535	781.855537729453\\
34536	787.290803867554\\
34537	790.974128295457\\
34540	779.218131963172\\
34541	781.596021661665\\
34542	765.799547523132\\
34543	767.660159006569\\
34544	759.298009343836\\
34548	769.500831225007\\
34549	767.117252777976\\
34550	763.325105906458\\
34551	762.183393665343\\
34554	754.081175904475\\
34555	751.175991589169\\
34556	748.402168023911\\
34557	742.704331765486\\
34558	739.428692600266\\
34561	730.668829073604\\
34562	728.973033560508\\
34563	728.722169955887\\
34564	726.16771998945\\
34565	722.836568309727\\
34568	714.180956488428\\
34569	711.535353060848\\
34570	706.31236151603\\
34571	693.101927674711\\
34572	677.180373309091\\
34576	671.872614061593\\
34577	669.813153031383\\
34578	680.047468643048\\
34579	675.446245464792\\
34582	660.000262219064\\
34583	667.401262740226\\
34584	664.104252006502\\
34585	662.230027100682\\
34586	648.628507405723\\
34589	635.846969811639\\
34590	627.850807256464\\
34591	594.695567269348\\
34592	615.628077405104\\
34593	572.597696187573\\
34596	572.812627396562\\
34597	520.269955408367\\
34598	489.85442658601\\
34599	504.488284126685\\
34600	509.107361812001\\
34603	463.834715932601\\
34604	478.692859846523\\
34605	508.051808091803\\
34606	451.935207388644\\
34607	477.751856240794\\
34610	421.780809306759\\
34611	437.703324059519\\
34612	384.401435122684\\
34613	413.010314618013\\
34614	422.279815027121\\
34617	446.447610871128\\
34618	483.003423422321\\
34619	500.263673078872\\
34620	525.325381430298\\
34621	500.32955588308\\
34624	496.621448135924\\
34625	466.710991378521\\
34626	443.581931595931\\
34627	441.221664008837\\
34628	410.759521580835\\
34631	398.244829012347\\
34632	357.220986626408\\
34633	369.257721345338\\
34634	388.609501847271\\
34635	427.489444673514\\
34638	420.094665695167\\
34639	414.916788793871\\
34640	402.454750374387\\
34641	409.898671289771\\
34642	402.817118017812\\
34645	366.268543678198\\
34646	362.374830728639\\
34647	380.449600787079\\
34648	377.181353152198\\
34649	353.712418171724\\
34652	343.858405816757\\
34653	361.639685657159\\
34654	360.4114657906\\
34655	345.943938166418\\
34656	341.197800618327\\
34659	316.281421157093\\
34660	284.987661601564\\
34661	238.265439741097\\
34662	250.057382236084\\
34663	238.078123411546\\
34666	225.274249802076\\
34667	229.653266532226\\
34668	231.813582250472\\
34669	196.17574401984\\
34670	179.024204552264\\
34673	158.858186039473\\
34674	135.20805248663\\
34675	119.868306184277\\
34676	106.904937925657\\
34677	71.1282914913157\\
34680	18.3812942482605\\
34681	10.726820887488\\
34682	8.47797667327603\\
34683	1.53126551967849\\
34684	0.176633279967209\\
};
\addplot [color=mycolor2, forget plot]
  table[row sep=crcr]{%
34445	972.965889707247\\
34446	1000.87507576217\\
34449	968.104971208122\\
34450	983.471674078341\\
34451	999.065688039759\\
34452	981.913942481628\\
34453	975.222959163496\\
34457	940.014387729214\\
34458	901.273250584995\\
34459	943.904888127351\\
34460	944.449106746475\\
34463	928.20535345113\\
34464	958.138278416589\\
34465	953.52667049619\\
34466	958.069792419061\\
34467	940.068166584228\\
34470	926.650570616356\\
34471	934.181670895748\\
34472	925.112981482674\\
34473	928.287126692865\\
34474	927.477326502099\\
34477	900.571106873063\\
34478	879.430429868447\\
34479	772.592831050568\\
34480	779.930166461089\\
34481	688.212807981404\\
34485	682.410325574735\\
34486	611.558375111122\\
34487	687.026938481878\\
34488	713.467529128194\\
34491	757.482023696355\\
34492	759.58892023089\\
34493	796.817361012397\\
34494	780.085572282846\\
34495	809.684921135524\\
34498	765.176244331714\\
34499	789.858934739043\\
34500	792.750388133156\\
34501	769.290533146707\\
34502	754.128304891474\\
34505	679.175463075493\\
34506	620.061479945243\\
34507	645.452511418896\\
34508	619.612345620511\\
34509	504.541214617559\\
34512	569.503210046312\\
34513	612.927201431617\\
34514	656.124078710715\\
34515	617.799971238148\\
34519	676.348013208046\\
34520	661.822065236871\\
34521	636.955694022388\\
34522	659.510009232054\\
34523	652.632883467695\\
34526	669.437072341473\\
34527	644.907204971059\\
34528	689.938590320141\\
34529	733.794969403692\\
34530	759.69421615055\\
34533	756.173296629085\\
34534	758.550170576367\\
34535	746.400615351754\\
34536	757.573073139994\\
34537	768.42582124819\\
34540	751.266608488834\\
34541	759.63625963362\\
34542	730.123740106446\\
34543	736.649380795887\\
34544	723.529786860651\\
34548	760.97364047908\\
34549	760.904368154101\\
34550	753.075118896228\\
34551	758.341505786942\\
34554	750.540140713648\\
34555	746.538934905227\\
34556	743.205573426664\\
34557	727.659355261993\\
34558	723.051750500209\\
34561	713.053730712775\\
34562	713.062068753761\\
34563	731.180142262558\\
34564	724.725300094934\\
34565	725.722058716438\\
34568	704.058946444225\\
34569	702.521779638247\\
34570	713.061793878428\\
34571	721.359540512749\\
34572	723.215277093744\\
34576	708.796165238168\\
34577	705.48045493269\\
34578	691.097830799586\\
34579	690.385178557343\\
34582	686.299900945564\\
34583	668.154483911335\\
34584	665.221766082297\\
34585	645.476019035696\\
34586	607.332910226522\\
34589	595.404777586666\\
34590	581.608313316866\\
34591	523.310914407662\\
34592	551.625413028855\\
34593	487.544414298118\\
34596	520.170877702104\\
34597	464.193288866606\\
34598	457.122875760312\\
34599	488.443723358204\\
34600	490.059161157128\\
34603	449.989062824452\\
34604	460.795281670642\\
34605	483.557452297381\\
34606	436.092843248063\\
34607	458.576672920754\\
34610	410.93281138577\\
34611	422.030723954735\\
34612	375.729436721297\\
34613	398.025274626446\\
34614	412.773649213176\\
34617	430.699441761336\\
34618	464.261105557194\\
34619	481.099787283605\\
34620	506.618388392324\\
34621	477.338966681767\\
34624	473.25264446852\\
34625	442.880916149536\\
34626	416.316865013431\\
34627	422.945378347027\\
34628	397.658325684187\\
34631	399.480513974303\\
34632	372.827237918276\\
34633	399.186363031534\\
34634	408.936473972749\\
34635	440.355455786523\\
34638	431.38388030994\\
34639	425.035838973825\\
34640	416.290879130933\\
34641	417.320045777878\\
34642	410.738013553446\\
34645	383.465099574348\\
34646	378.404701459105\\
34647	385.692674801979\\
34648	381.794563548406\\
34649	366.588824818733\\
34652	352.926153569729\\
34653	361.179815963042\\
34654	360.671106318731\\
34655	349.569200623424\\
34656	344.083812917406\\
34659	321.467808281682\\
34660	302.968725673958\\
34661	280.63649235427\\
34662	276.515633004164\\
34663	266.93075826979\\
34666	247.213061495569\\
34667	243.328953186588\\
34668	240.999542154776\\
34669	217.816149728429\\
34670	199.109641216538\\
34673	172.713730287664\\
34674	152.905208286624\\
34675	137.748296593547\\
34676	124.829040629863\\
34677	96.5894312370589\\
34680	39.5409390221358\\
34681	26.6009737765457\\
34682	19.5085800479297\\
34683	5.96458893268781\\
34684	0.834762303879043\\
};
\addplot [color=mycolor2, forget plot]
  table[row sep=crcr]{%
34445	658.372655144641\\
34446	596.196820509858\\
34449	634.103418157047\\
34450	606.323220872183\\
34451	572.969360412785\\
34452	599.947090426413\\
34453	609.65845228443\\
34457	641.665494886068\\
34458	682.489127213411\\
34459	628.970204041401\\
34460	633.792378828616\\
34463	638.15868594595\\
34464	584.155698179689\\
34465	580.444850893321\\
34466	558.840862498282\\
34467	588.072254759583\\
34470	594.943270568633\\
34471	552.340468933594\\
34472	561.399251275747\\
34473	538.371066140187\\
34474	519.443672829984\\
34477	544.894301143324\\
34478	575.180398721937\\
34479	694.978892974187\\
34480	692.266049906438\\
34481	767.344110215086\\
34485	753.915364380432\\
34486	791.808469860275\\
34487	748.149697598405\\
34488	728.585795407586\\
34491	694.727363839666\\
34492	686.569258452683\\
34493	628.527873978224\\
34494	642.688870841058\\
34495	599.514851248062\\
34498	656.726511987052\\
34499	608.016491912919\\
34500	594.788429470786\\
34501	621.699385411039\\
34502	636.405043956147\\
34505	704.331586258347\\
34506	742.420142267781\\
34507	721.02327932128\\
34508	733.992757875942\\
34509	780.526442258308\\
34512	761.501694698189\\
34513	750.307209798936\\
34514	709.394775106226\\
34515	740.742088536319\\
34519	666.421098411464\\
34520	678.21886107189\\
34521	695.659015619132\\
34522	668.136194275595\\
34523	667.308942351515\\
34526	620.81363433584\\
34527	655.049256488726\\
34528	580.974845086469\\
34529	500.523054723337\\
34530	433.577332389278\\
34533	439.617165256206\\
34534	420.077822337336\\
34535	434.476069911167\\
34536	421.512617199553\\
34537	395.156258355044\\
34540	421.144008789087\\
34541	390.505512571417\\
34542	454.545223486933\\
34543	444.310635566565\\
34544	464.387544794788\\
34548	327.711304749881\\
34549	351.733590530168\\
34550	363.630701854719\\
34551	337.322147961371\\
34554	324.790944176636\\
34555	331.653269647331\\
34556	331.662715458449\\
34557	360.045506156366\\
34558	362.459501628586\\
34561	354.775895361503\\
34562	366.763592113705\\
34563	301.706878198724\\
34564	314.91636327714\\
34565	290.588341557965\\
34568	310.001609323939\\
34569	305.887550916779\\
34570	262.575953573379\\
34571	224.650098829556\\
34572	191.06949109323\\
34576	210.118017797924\\
34577	208.494638829701\\
34578	248.421389888149\\
34579	259.601335535997\\
34582	228.566477550895\\
34583	276.895637445565\\
34584	273.384959514228\\
34585	302.735305418476\\
34586	361.946402034293\\
34589	380.511257666869\\
34590	388.261133624173\\
34591	434.543058741132\\
34592	391.178091637731\\
34593	449.349928102332\\
34596	422.023259735926\\
34597	476.884056831233\\
34598	496.778803595591\\
34599	489.091364797882\\
34600	476.369739440163\\
34603	492.745099258436\\
34604	483.974014698034\\
34605	447.566423871167\\
34606	489.616700480701\\
34607	458.497343599588\\
34610	488.256985844361\\
34611	471.354916199863\\
34612	498.301760620074\\
34613	479.55326528998\\
34614	464.901616628079\\
34617	421.48409632504\\
34618	365.804751634417\\
34619	329.630186937579\\
34620	288.529027053289\\
34621	342.879742264361\\
34624	316.411940163297\\
34625	354.775481898172\\
34626	374.74092809079\\
34627	368.973367926435\\
34628	394.512392713156\\
34631	381.348343962963\\
34632	404.369479203751\\
34633	407.655283358447\\
34634	390.75029228889\\
34635	346.431477734562\\
34638	318.993672331638\\
34639	318.658887259302\\
34640	323.12513629622\\
34641	299.638496550352\\
34642	297.405511435247\\
34645	311.342934657131\\
34646	311.287514202775\\
34647	272.299796543705\\
34648	268.884731333779\\
34649	288.448009196818\\
34652	256.219768753661\\
34653	205.522386106759\\
34654	211.15547893228\\
34655	219.707606182584\\
34656	209.941246996315\\
34659	196.853081715293\\
34660	224.598036542829\\
34661	259.684423371527\\
34662	251.457445277436\\
34663	246.609415255206\\
34666	214.321056898153\\
34667	199.061299627692\\
34668	176.207181633933\\
34669	202.272425389102\\
34670	207.405252495116\\
34673	169.773384368299\\
34674	171.012203863129\\
34675	161.769856719147\\
34676	150.384012708694\\
34677	159.303146966798\\
34680	134.345382497664\\
34681	118.437597173857\\
34682	85.3354366916657\\
34683	66.187091843715\\
34684	31.0444743431013\\
};
\addplot [color=mycolor2, forget plot]
  table[row sep=crcr]{%
34445	799.859903992126\\
34446	740.105181002497\\
34449	777.157455840145\\
34450	749.578188428873\\
34451	711.834272303722\\
34452	740.58986796132\\
34453	750.184311879969\\
34457	780.965959243551\\
34458	819.739231485921\\
34459	771.519236522973\\
34460	775.579806437464\\
34463	779.536286110936\\
34464	727.887182958421\\
34465	735.87599937182\\
34466	717.735893519759\\
34467	744.873372934722\\
34470	750.520629838315\\
34471	716.5965755375\\
34472	725.462349556721\\
34473	711.070318934706\\
34474	703.46434799433\\
34477	728.673099641327\\
34478	752.470485032591\\
34479	840.473423928275\\
34480	828.509751071356\\
34481	868.466789401856\\
34485	859.042010491133\\
34486	862.059965655786\\
34487	853.113697237097\\
34488	839.877316495051\\
34491	823.657130732751\\
34492	822.040013842571\\
34493	789.677572925514\\
34494	801.405880938614\\
34495	771.307728237999\\
34498	803.990255227613\\
34499	774.200985546026\\
34500	764.954759392147\\
34501	784.022583240482\\
34502	790.912842414211\\
34505	817.964026963706\\
34506	820.158113116382\\
34507	817.810400506779\\
34508	816.120216310042\\
34509	770.80055720949\\
34512	786.860224119456\\
34513	792.76792266428\\
34514	803.016510634977\\
34515	791.332963457437\\
34519	788.004302079388\\
34520	788.479663195106\\
34521	787.215814545988\\
34522	783.426496336672\\
34523	782.36008212674\\
34526	768.019609635694\\
34527	772.944033065563\\
34528	749.762032665403\\
34529	696.762370986532\\
34530	664.103822710332\\
34533	648.149056530801\\
34534	632.697710837984\\
34535	652.875787670481\\
34536	618.590860624058\\
34537	582.7936190925\\
34540	603.591590383092\\
34541	572.533825123933\\
34542	631.365512419553\\
34543	610.757370589892\\
34544	628.907416394433\\
34548	487.085243574003\\
34549	503.179196497649\\
34550	519.083522845493\\
34551	490.90399229616\\
34554	477.851221923078\\
34555	484.055919598365\\
34556	483.151791941759\\
34557	524.648515699789\\
34558	525.337637446236\\
34561	527.339484200784\\
34562	508.875002965356\\
34563	432.135636325626\\
34564	451.948101740371\\
34565	435.928202599004\\
34568	477.86819459569\\
34569	465.355165745739\\
34570	415.21814584635\\
34571	361.597060688225\\
34572	313.360034084087\\
34576	342.241661901975\\
34577	340.1738578562\\
34578	395.412689756819\\
34579	395.211823445714\\
34582	354.645730705022\\
34583	413.49756355202\\
34584	407.827287767451\\
34585	443.11439992046\\
34586	510.302421438332\\
34589	513.723220216509\\
34590	520.894695829272\\
34591	559.83290234318\\
34592	520.925717617833\\
34593	564.51143655728\\
34596	543.565918116433\\
34597	572.716587968117\\
34598	577.566387823092\\
34599	573.079181558398\\
34600	566.671098557661\\
34603	563.800341109661\\
34604	559.540527112095\\
34605	545.116099626619\\
34606	554.19037256233\\
34607	545.934208969743\\
34610	539.635994681507\\
34611	536.568543992647\\
34612	524.681044958881\\
34613	528.933283147174\\
34614	526.073031077266\\
34617	507.682007178039\\
34618	479.310025212337\\
34619	454.656664395825\\
34620	398.295127110607\\
34621	444.226301540798\\
34624	416.55052085138\\
34625	449.085321765904\\
34626	462.932674834854\\
34627	457.425304364208\\
34628	468.133517110391\\
34631	455.171548823568\\
34632	455.697574844965\\
34633	451.60870625204\\
34634	444.033702230195\\
34635	405.998412612536\\
34638	377.738370338901\\
34639	378.122150361585\\
34640	383.934934981715\\
34641	355.632840813243\\
34642	356.885249928516\\
34645	371.971076669276\\
34646	373.159149199752\\
34647	331.233617754639\\
34648	326.109211502428\\
34649	349.854926329443\\
34652	317.635206641543\\
34653	251.455694814839\\
34654	246.111075045203\\
34655	264.341793484312\\
34656	256.000870549429\\
34659	247.152054759238\\
34660	284.451683555682\\
34661	307.553386712543\\
34662	299.365358859981\\
34663	293.791829802321\\
34666	268.646109627468\\
34667	254.948931776578\\
34668	234.839686847614\\
34669	249.417738189826\\
34670	243.963360530399\\
34673	217.565451383015\\
34674	208.650326872604\\
34675	198.590132804162\\
34676	188.38852132729\\
34677	161.766218430923\\
34680	85.6601794516693\\
34681	73.3043276451974\\
34682	85.8634037293265\\
34683	57.5508424043973\\
34684	60.3732248518862\\
};
\addplot [color=mycolor2, forget plot]
  table[row sep=crcr]{%
34445	944.380896073016\\
34446	916.235718751825\\
34449	929.865220350661\\
34450	912.100720008044\\
34451	860.231241659167\\
34452	867.369694511074\\
34453	875.059027796335\\
34457	901.110108214691\\
34458	928.299161089612\\
34459	891.17581608145\\
34460	888.177939237167\\
34463	892.008473360511\\
34464	846.479905149505\\
34465	851.977049058642\\
34466	834.809313265388\\
34467	861.779158247243\\
34470	866.981869411967\\
34471	849.393766619351\\
34472	856.965619685145\\
34473	846.875207381219\\
34474	843.592149684816\\
34477	862.386007134471\\
34478	878.602486062727\\
34479	903.671135701755\\
34480	902.524474262343\\
34481	865.574763612965\\
34485	863.137891474263\\
34486	817.928128402142\\
34487	863.159368933708\\
34488	873.890938260074\\
34491	874.116104502649\\
34492	871.778633716992\\
34493	876.07411649322\\
34494	873.02009791942\\
34495	870.372019187828\\
34498	860.946804685825\\
34499	863.155115697757\\
34500	860.681215876876\\
34501	857.729166568053\\
34502	853.280923080653\\
34505	808.614511238174\\
34506	756.404324690999\\
34507	782.495091498487\\
34508	755.711362476806\\
34509	573.179752806957\\
34512	632.315037778861\\
34513	655.384876646789\\
34514	730.196127304402\\
34515	665.115548250396\\
34519	758.833711779244\\
34520	740.978294927093\\
34521	703.711840586372\\
34522	736.690748257955\\
34523	721.661268822165\\
34526	749.725202148956\\
34527	712.65561378225\\
34528	770.596568713137\\
34529	794.098376562799\\
34530	786.708979155414\\
34533	776.790772602594\\
34534	770.018442020832\\
34535	773.83321328746\\
34536	758.645831152542\\
34537	739.540442523276\\
34540	745.25275825231\\
34541	725.657235441258\\
34542	754.893150784949\\
34543	745.237311988094\\
34544	751.179181968134\\
34548	654.639350930999\\
34549	657.437765970668\\
34550	684.938695678466\\
34551	677.617752059932\\
34554	666.865575190261\\
34555	668.197642090537\\
34556	666.841479103625\\
34557	687.733388238818\\
34558	687.003997104025\\
34561	675.221490036927\\
34562	668.578110108499\\
34563	610.084795097287\\
34564	619.7050485044\\
34565	603.341170208968\\
34568	627.858234037527\\
34569	617.691586672596\\
34570	573.112880625706\\
34571	515.369345851821\\
34572	475.169629089313\\
34576	483.636113970093\\
34577	482.174199112208\\
34578	540.606339108035\\
34579	537.077067008536\\
34582	500.150927648582\\
34583	556.267252417094\\
34584	552.37053886425\\
34585	581.022481372512\\
34586	620.422545240481\\
34589	617.633797704408\\
34590	618.867295664581\\
34591	622.746222530984\\
34592	614.632367268301\\
34593	614.082372145073\\
34596	606.349661634375\\
34597	584.918936128908\\
34598	565.199039003809\\
34599	565.637383177324\\
34600	568.753949818243\\
34603	531.387298776963\\
34604	534.617830565\\
34605	557.915049505348\\
34606	509.367181377242\\
34607	535.5717967511\\
34610	476.696941874815\\
34611	494.794879865739\\
34612	428.702163804935\\
34613	466.413978217951\\
34614	477.874208200448\\
34617	501.794413173799\\
34618	524.226536196161\\
34619	524.403836201255\\
34620	505.490023440902\\
34621	516.285984760394\\
34624	501.318892399572\\
34625	498.416985770982\\
34626	482.603860978966\\
34627	478.598933595273\\
34628	438.266150430652\\
34631	428.980300648002\\
34632	369.041443909511\\
34633	371.031056356407\\
34634	404.887321772063\\
34635	453.606964056731\\
34638	443.355653222176\\
34639	437.900896680352\\
34640	429.061675693418\\
34641	429.622941431611\\
34642	424.173064215482\\
34645	391.925361353598\\
34646	382.249049471545\\
34647	399.146020706012\\
34648	394.19557515035\\
34649	375.449815552658\\
34652	366.785301110146\\
34653	360.613375009657\\
34654	349.52229244196\\
34655	353.686432105334\\
34656	346.588827345247\\
34659	330.857996896362\\
34660	311.28238795162\\
34661	241.040500317647\\
34662	255.103266411431\\
34663	237.578912703759\\
34666	232.823222944257\\
34667	241.341272723581\\
34668	252.556359610939\\
34669	196.869086490682\\
34670	159.561083633806\\
34673	148.596387433375\\
34674	109.586785947071\\
34675	93.8628457932966\\
34676	81.7392243276172\\
34677	32.6794580059936\\
34680	2.06829181140161\\
34681	0.875677895950182\\
34682	1.36098464706457\\
34683	0.0733213861723493\\
34684	0.0232538276862317\\
};
\addplot [color=mycolor2, forget plot]
  table[row sep=crcr]{%
34445	1002.33198361266\\
34446	997.268002322466\\
34449	994.957572024848\\
34450	992.237557565727\\
34451	982.735159675869\\
34452	987.635601417302\\
34453	986.700154856963\\
34457	977.344223920665\\
34458	960.149480698157\\
34459	974.454453586751\\
34460	972.46280931841\\
34463	964.181777780413\\
34464	963.854351081386\\
34465	962.663930439934\\
34466	958.720442247754\\
34467	959.300899087131\\
34470	952.505915030751\\
34471	951.026952546843\\
34472	948.814522063184\\
34473	946.92855177739\\
34474	944.770140606495\\
34477	935.658350163311\\
34478	924.607724815149\\
34479	819.435237784382\\
34480	830.873311022324\\
34481	693.712494124211\\
34485	707.49151433488\\
34486	609.898673655735\\
34487	706.130490073917\\
34488	738.501915430216\\
34491	759.780083587389\\
34492	757.512078440726\\
34493	814.46107323565\\
34494	783.787909993267\\
34495	826.8080725197\\
34498	742.928425595466\\
34499	789.485016128773\\
34500	795.685624745317\\
34501	756.778368205816\\
34502	734.177191625893\\
34505	593.081685884731\\
34506	477.853207190274\\
34507	531.731192081318\\
34508	484.72798859999\\
34509	275.075749132387\\
34512	342.815637669508\\
34513	379.882688580612\\
34514	475.384065839496\\
34515	385.352979545272\\
34519	528.661722636737\\
34520	496.615032010118\\
34521	424.079446843873\\
34522	470.446551778553\\
34523	452.263384996636\\
34526	506.099578193774\\
34527	440.731235913494\\
34528	561.655266667728\\
34529	680.610790275684\\
34530	746.999431324023\\
34533	748.493753075094\\
34534	756.592845837584\\
34535	735.433113997896\\
34536	761.625193575598\\
34537	780.774833781857\\
34540	759.433757457164\\
34541	773.560299969892\\
34542	724.617059098751\\
34543	736.76847551729\\
34544	712.855076004935\\
34548	770.609746019535\\
34549	767.980587025113\\
34550	764.361280701811\\
34551	762.005020755282\\
34554	753.491195601667\\
34555	751.193377749682\\
34556	748.551904349699\\
34557	741.977163279362\\
34558	738.0391574415\\
34561	728.172985989232\\
34562	728.34346492355\\
34563	719.897080848905\\
34564	722.064435006644\\
34565	713.759202991602\\
34568	714.530487897798\\
34569	710.943907102979\\
34570	690.786736826258\\
34571	644.113231775218\\
34572	589.343451051394\\
34576	611.005887059894\\
34577	611.139420944996\\
34578	659.495517777637\\
34579	652.542105291881\\
34582	620.71576126847\\
34583	658.837355321278\\
34584	656.387459850633\\
34585	662.436003703713\\
34586	637.472192844621\\
34589	621.221189973087\\
34590	606.676893840735\\
34591	534.631804099703\\
34592	591.002869853924\\
34593	540.10111691009\\
34596	547.174715154949\\
34597	493.689017772719\\
34598	458.905772108211\\
34599	462.118765067898\\
34600	470.315188160924\\
34603	412.179721744571\\
34604	419.874921738703\\
34605	464.830642801995\\
34606	384.766230911184\\
34607	427.072875377638\\
34610	343.294222855285\\
34611	367.868687599963\\
34612	287.826295532255\\
34613	329.21411431246\\
34614	344.800732633128\\
34617	386.996329285622\\
34618	447.182162045479\\
34619	478.565138524467\\
34620	518.551747183856\\
34621	478.268018462328\\
34624	480.791080315832\\
34625	431.205312938409\\
34626	386.461752811951\\
34627	386.797256818423\\
34628	331.669669010985\\
34631	320.039266533216\\
34632	255.487113504877\\
34633	256.374463654065\\
34634	291.878089053111\\
34635	374.596111934147\\
34638	379.204280209266\\
34639	367.532754436208\\
34640	345.749106025047\\
34641	367.299358600032\\
34642	355.35616886051\\
34645	287.504757343765\\
34646	277.771702536943\\
34647	329.932953837903\\
34648	327.873355897318\\
34649	276.03089231465\\
34652	272.060277263947\\
34653	331.60347143844\\
34654	341.742253382614\\
34655	312.239287725656\\
34656	310.229658110537\\
34659	274.370876030618\\
34660	204.926892958739\\
34661	120.546687792718\\
34662	134.322035966737\\
34663	116.39413937203\\
34666	109.815335936346\\
34667	121.876907389236\\
34668	136.636691330387\\
34669	55.5560353909237\\
34670	35.4362118905683\\
34673	19.562304512718\\
34674	8.16046971902074\\
34675	4.95074009429578\\
34676	3.07703695523514\\
34677	0.292702629143594\\
34680	0.000163485685563399\\
34681	9.06997258841486e-06\\
34682	4.35682728803119e-06\\
34683	1.92134342428979e-10\\
34684	1.59609787907029e-15\\
};
\addplot [color=mycolor2, forget plot]
  table[row sep=crcr]{%
34445	943.380085080096\\
34446	988.509671390802\\
34449	939.995576036218\\
34450	964.380648636466\\
34451	989.761494886577\\
34452	964.668239879464\\
34453	955.457208941824\\
34457	903.635894119372\\
34458	839.178514724259\\
34459	907.31873774301\\
34460	902.504149896907\\
34463	877.017715548906\\
34464	933.166175275093\\
34465	923.989455851596\\
34466	933.612412389905\\
34467	900.765890336983\\
34470	879.487404061761\\
34471	887.029136464065\\
34472	869.031181485365\\
34473	876.910209720732\\
34474	875.696632747912\\
34477	820.893743874872\\
34478	775.760633756146\\
34479	549.724410798771\\
34480	575.375629129216\\
34481	390.717822174353\\
34485	415.189891150671\\
34486	315.244444489378\\
34487	418.145803394447\\
34488	466.945043218493\\
34491	463.220176907164\\
34492	458.852556353322\\
34493	547.936084845215\\
34494	494.314158900423\\
34495	567.549153912676\\
34498	441.701066162155\\
34499	496.249433584531\\
34500	507.470599730494\\
34501	447.432650688607\\
34502	417.185804732623\\
34505	265.451658031942\\
34506	168.61225368116\\
34507	208.292511088995\\
34508	173.64889245636\\
34509	58.8227449552837\\
34512	90.3863475661999\\
34513	111.042506780964\\
34514	168.629879784631\\
34515	113.208969690348\\
34519	205.332685402372\\
34520	181.867187519617\\
34521	132.958144747791\\
34522	159.234591941909\\
34523	148.241867352214\\
34526	181.753793050918\\
34527	138.260576858272\\
34528	225.245524063724\\
34529	355.437446979175\\
34530	459.489332453716\\
34533	469.065424433294\\
34534	484.351540558734\\
34535	435.439555953855\\
34536	489.407949725545\\
34537	550.919596326779\\
34540	507.337951566461\\
34541	556.083934199366\\
34542	436.413394430591\\
34543	471.793130049795\\
34544	425.226107131773\\
34548	638.028245423542\\
34549	657.050810565291\\
34550	628.434951748566\\
34551	666.398805096139\\
34554	660.197213609645\\
34555	652.870734911446\\
34556	648.009628078735\\
34557	579.500847387857\\
34558	572.649413311742\\
34561	530.923777498577\\
34562	552.885228526882\\
34563	653.026870556754\\
34564	630.290108206908\\
34565	650.040789672609\\
34568	568.66598941092\\
34569	586.098282837341\\
34570	655.707557307807\\
34571	710.239007004389\\
34572	724.143215253692\\
34576	706.544676032736\\
34577	702.826228530631\\
34578	656.360705053321\\
34579	665.567087812114\\
34582	676.334483323701\\
34583	615.799463711182\\
34584	618.503671208704\\
34585	553.355819248361\\
34586	432.207362868563\\
34589	416.135243247059\\
34590	384.568003472544\\
34591	268.497248415475\\
34592	358.95387540407\\
34593	245.581543086326\\
34596	262.514668732665\\
34597	164.502762542985\\
34598	130.982163025942\\
34599	136.07159325794\\
34600	144.560247243961\\
34603	93.032438292091\\
34604	102.025878845171\\
34605	146.062626647904\\
34606	75.9787996851101\\
34607	106.536311866558\\
34610	50.9774844901266\\
34611	64.4848455257052\\
34612	29.0926631882546\\
34613	47.6292878786782\\
34614	56.2239044746459\\
34617	82.3033202334948\\
34618	139.170521196495\\
34619	194.840445238974\\
34620	299.053366569472\\
34621	204.376940313577\\
34624	234.096931206086\\
34625	151.55956476715\\
34626	99.2664443664759\\
34627	100.883551370734\\
34628	58.9744789138297\\
34631	54.3112841663653\\
34632	25.2852224542786\\
34633	25.9552517280785\\
34634	39.571814486641\\
34635	102.83968366384\\
34638	118.567329708008\\
34639	103.900524187833\\
34640	83.5054375675658\\
34641	105.567378689621\\
34642	94.4015156053153\\
34645	44.1493873145733\\
34646	41.7829438040496\\
34647	79.8985348117206\\
34648	80.2938883496107\\
34649	41.4345314419037\\
34652	36.9196490015055\\
34653	87.449282401912\\
34654	123.837820625535\\
34655	79.6058446033838\\
34656	79.9987239892524\\
34659	49.151021962071\\
34660	19.7498565439646\\
34661	3.62586918746965\\
34662	5.75549841926285\\
34663	3.28831693542144\\
34666	2.53821546169573\\
34667	3.37331351821342\\
34668	4.64905908702636\\
34669	0.511535970972067\\
34670	0.151071186362081\\
34673	0.0527076736587112\\
34674	0.00574548585877048\\
34675	0.00140430046439974\\
34676	0.000346554136526968\\
34677	1.33238340170151e-06\\
34680	7.39416307218567e-15\\
34681	5.4000996385405e-18\\
34682	3.87549613803676e-20\\
34683	1.5988647803613e-31\\
34684	4.16923925575053e-51\\
};
\end{axis}

\begin{axis}[%
width=3.888in,
height=1.366in,
at={(0.703in,3.034in)},
scale only axis,
xmin=34445,
xmax=34684,
xtick={34425,34455,34486,34516,34547,34578,34608,34639,34669,34700},
xticklabels={{01/04},{01/05},{01/06},{01/07},{01/08},{01/09},{01/10},{01/11},{01/12},{01/01}},
ymin=1.69842232821915e-54,
ymax=0.999667132637736,
ylabel style={font=\color{white!15!black}},
ylabel={Value},
axis background/.style={fill=white},
title style={font=\bfseries},
title={$\Delta\text{: Sensitivity to Underlying Price Change}$},
xmajorgrids,
ymajorgrids
]
\addplot [color=mycolor3, forget plot]
  table[row sep=crcr]{%
34445	0.956368267488483\\
34446	0.971214007461628\\
34449	0.959237887056074\\
34450	0.965905248542776\\
34451	0.974648931089439\\
34452	0.968936824821471\\
34453	0.966044087965179\\
34457	0.953143161567232\\
34458	0.931579552816536\\
34459	0.952575714426021\\
34460	0.949106126702609\\
34463	0.945685074549057\\
34464	0.963254826753923\\
34465	0.960050040607162\\
34466	0.964487981385783\\
34467	0.955676094245231\\
34470	0.951448741242607\\
34471	0.961943422028576\\
34472	0.958984391622828\\
34473	0.96405000441491\\
34474	0.966609856090996\\
34477	0.957763967264242\\
34478	0.946384619354277\\
34479	0.882679017655049\\
34480	0.864782982966928\\
34481	0.784762442758468\\
34485	0.778902359366068\\
34486	0.716805542366958\\
34487	0.787395438150289\\
34488	0.818934157099328\\
34491	0.839888506761423\\
34492	0.836980785944962\\
34493	0.87472397452148\\
34494	0.858395479199058\\
34495	0.885211867333477\\
34498	0.845567830193088\\
34499	0.878218060386065\\
34500	0.884829064704696\\
34501	0.866813202277528\\
34502	0.857144586870107\\
34505	0.786677696871537\\
34506	0.72908559380654\\
34507	0.760616598249139\\
34508	0.734809644092073\\
34509	0.600692080180277\\
34512	0.650682118804557\\
34513	0.675030418919294\\
34514	0.747113553608538\\
34515	0.687274075576058\\
34519	0.791666212341801\\
34520	0.775255843746828\\
34521	0.746178162842494\\
34522	0.780915312881746\\
34523	0.768089471552369\\
34526	0.810219972347547\\
34527	0.772564232514026\\
34528	0.839842677952255\\
34529	0.889374157832404\\
34530	0.912064315816083\\
34533	0.917547559540641\\
34534	0.922751929557606\\
34535	0.913941812691075\\
34536	0.931847098083549\\
34537	0.94308277529162\\
34540	0.933223806149935\\
34541	0.944318353063704\\
34542	0.9177712368294\\
34543	0.924833892750333\\
34544	0.914389465920149\\
34548	0.964037402080658\\
34549	0.961457206815131\\
34550	0.957431071088884\\
34551	0.965113352834268\\
34554	0.967356203885888\\
34555	0.966292130679548\\
34556	0.966482984217311\\
34557	0.957293747459093\\
34558	0.955364237116146\\
34561	0.968589965235154\\
34562	0.970019968186356\\
34563	0.990334599528894\\
34564	0.988722800693842\\
34565	0.990619438029273\\
34568	0.986467113867977\\
34569	0.987532965904796\\
34570	0.993285952828285\\
34571	0.996369413389427\\
34572	0.998160168507146\\
34576	0.997771779489445\\
34577	0.997696879533448\\
34578	0.995353632953473\\
34579	0.995366307146676\\
34582	0.997126528610917\\
34583	0.993915345254894\\
34584	0.99364173402019\\
34585	0.990115161436771\\
34586	0.978066313799174\\
34589	0.971806058888621\\
34590	0.968304236028049\\
34591	0.943011171271939\\
34592	0.964183609213225\\
34593	0.920275094324883\\
34596	0.921935903615533\\
34597	0.870292232016286\\
34598	0.838161865232113\\
34599	0.844882160165117\\
34600	0.861827593742712\\
34603	0.817800630625905\\
34604	0.825327861031769\\
34605	0.879531910799203\\
34606	0.801188440695137\\
34607	0.859038119819578\\
34610	0.784633519100862\\
34611	0.822314409186046\\
34612	0.719313142565161\\
34613	0.786638354566797\\
34614	0.814312903947123\\
34617	0.881622242771065\\
34618	0.937305850771233\\
34619	0.954665702890329\\
34620	0.980450744337917\\
34621	0.958782684279176\\
34624	0.967771539553522\\
34625	0.940256145394886\\
34626	0.916081418916444\\
34627	0.918685291281303\\
34628	0.875218466004925\\
34631	0.880744742372924\\
34632	0.81671179003474\\
34633	0.82515826232957\\
34634	0.862810401801893\\
34635	0.932219680213183\\
34638	0.940234146428199\\
34639	0.939819916814436\\
34640	0.930199250171053\\
34641	0.951081677984243\\
34642	0.948073756143372\\
34645	0.920493473528184\\
34646	0.909227610338119\\
34647	0.9531933592527\\
34648	0.953615702776625\\
34649	0.927072579610386\\
34652	0.948580963243019\\
34653	0.981035130221003\\
34654	0.979712039840975\\
34655	0.972005249852371\\
34656	0.975605428940474\\
34659	0.975887376642053\\
34660	0.948826300347539\\
34661	0.871484654105233\\
34662	0.880784153790553\\
34663	0.878863714318168\\
34666	0.918405966307428\\
34667	0.939550779561067\\
34668	0.963619949131862\\
34669	0.919905955052687\\
34670	0.881879767835869\\
34673	0.926342642942671\\
34674	0.897509914251268\\
34675	0.903093852980846\\
34676	0.913713880823829\\
34677	0.813950649138906\\
34680	0.662077506918125\\
34681	0.701445171958346\\
34682	0.90251258626472\\
34683	0.924911596464115\\
34684	0.999667132637736\\
};
\addplot [color=mycolor3, forget plot]
  table[row sep=crcr]{%
34445	0.836297283825036\\
34446	0.872974959956116\\
34449	0.842459308162969\\
34450	0.859392181289425\\
34451	0.882155905822828\\
34452	0.865524555857679\\
34453	0.858701143518844\\
34457	0.828305733458424\\
34458	0.785804949709208\\
34459	0.831751904048978\\
34460	0.83083901995805\\
34463	0.822236713309941\\
34464	0.861031646737071\\
34465	0.854075744854052\\
34466	0.864094051167928\\
34467	0.843639806220222\\
34470	0.834328966032669\\
34471	0.854126701854999\\
34472	0.846529320532149\\
34473	0.85496632124303\\
34474	0.857529576831311\\
34477	0.834536239571302\\
34478	0.810090322957802\\
34479	0.694346502101877\\
34480	0.700215542020089\\
34481	0.590190580657459\\
34485	0.593410143945538\\
34486	0.523594022860672\\
34487	0.602145660002194\\
34488	0.636785860603733\\
34491	0.661176984552133\\
34492	0.658751618807513\\
34493	0.708202692975009\\
34494	0.688287261844116\\
34495	0.727020179512158\\
34498	0.66809992500261\\
34499	0.711323002107013\\
34500	0.720582284155441\\
34501	0.69344613028989\\
34502	0.681254512107402\\
34505	0.586934395951282\\
34506	0.516431772418249\\
34507	0.553246169528353\\
34508	0.52279009609915\\
34509	0.3702789393423\\
34512	0.418918065698277\\
34513	0.454500804196589\\
34514	0.522833831687103\\
34515	0.468960773520522\\
34519	0.56645233589725\\
34520	0.549374818111558\\
34521	0.519351862051758\\
34522	0.55189936995952\\
34523	0.541149645296336\\
34526	0.580312494739823\\
34527	0.543017751202932\\
34528	0.613996672240559\\
34529	0.68557382386364\\
34530	0.729791865268583\\
34533	0.729073006320278\\
34534	0.738994290269805\\
34535	0.724913477179602\\
34536	0.752544711323424\\
34537	0.775400837206786\\
34540	0.758843411777259\\
34541	0.779926785794868\\
34542	0.731814714721617\\
34543	0.74506491979951\\
34544	0.727082488194768\\
34548	0.82234537576513\\
34549	0.820610025811126\\
34550	0.810683839954679\\
34551	0.828677733790033\\
34554	0.832891970191683\\
34555	0.830171465317192\\
34556	0.83016254371566\\
34557	0.806020042609242\\
34558	0.803184278344398\\
34561	0.80308197697324\\
34562	0.810412277169383\\
34563	0.872412875335619\\
34564	0.866230428180523\\
34565	0.875884575181511\\
34568	0.853218660931635\\
34569	0.858973700030337\\
34570	0.889857579578477\\
34571	0.91812917673055\\
34572	0.936515159009591\\
34576	0.928155041229234\\
34577	0.927570257328369\\
34578	0.902013330821954\\
34579	0.904679941589752\\
34582	0.920811547586543\\
34583	0.890639616136626\\
34584	0.890100879368177\\
34585	0.864525868932089\\
34586	0.808678669000974\\
34589	0.793780462398688\\
34590	0.782380415495464\\
34591	0.716878779439171\\
34592	0.794777104858037\\
34593	0.715893326731882\\
34596	0.736108445960038\\
34597	0.639501575637656\\
34598	0.589598076110215\\
34599	0.601661870210972\\
34600	0.621247707924858\\
34603	0.554928203191311\\
34604	0.570200270098057\\
34605	0.646030009451881\\
34606	0.532392297381121\\
34607	0.602281634400867\\
34610	0.501493579773318\\
34611	0.545298001395984\\
34612	0.430895358469034\\
34613	0.499227810207924\\
34614	0.530470620017817\\
34617	0.617822713564951\\
34618	0.71807565661661\\
34619	0.764899642347959\\
34620	0.846355530094545\\
34621	0.774531992706618\\
34624	0.804984361539124\\
34625	0.730789132414373\\
34626	0.673772301657825\\
34627	0.680666286345955\\
34628	0.601602651970365\\
34631	0.606091295914379\\
34632	0.511338069848971\\
34633	0.521272755373846\\
34634	0.580505924068769\\
34635	0.708136014522046\\
34638	0.732125644441138\\
34639	0.722133012325932\\
34640	0.699785646878314\\
34641	0.742896656605504\\
34642	0.735210938728216\\
34645	0.66738634436907\\
34646	0.651136779852978\\
34647	0.740759376935835\\
34648	0.745701719084209\\
34649	0.680297310650921\\
34652	0.71883771482608\\
34653	0.821335913680237\\
34654	0.836444701002598\\
34655	0.800984308091082\\
34656	0.810163301901497\\
34659	0.799945257601158\\
34660	0.706336146807836\\
34661	0.549193739563397\\
34662	0.575984224422961\\
34663	0.562869994122815\\
34666	0.608126610853608\\
34667	0.650826462113022\\
34668	0.707269064845832\\
34669	0.581641525325464\\
34670	0.505727565799965\\
34673	0.556237194508115\\
34674	0.484258873610205\\
34675	0.472289489084925\\
34676	0.470563400487391\\
34677	0.328321639086506\\
34680	0.146649416571081\\
34681	0.16917850036036\\
34682	0.332971740556053\\
34683	0.272090913538325\\
34684	0.426058492114919\\
};
\addplot [color=mycolor3, forget plot]
  table[row sep=crcr]{%
34445	0.672047534810775\\
34446	0.719368720286533\\
34449	0.677846826957307\\
34450	0.70260918925868\\
34451	0.736252808322348\\
34452	0.711296961202997\\
34453	0.702728169788884\\
34457	0.662218941469246\\
34458	0.614780351925845\\
34459	0.667984569206801\\
34460	0.667289627001779\\
34463	0.654955568484644\\
34464	0.703558718413189\\
34465	0.692412746227142\\
34466	0.704762603955156\\
34467	0.678670786275975\\
34470	0.668070348474826\\
34471	0.68787468502165\\
34472	0.677515706771557\\
34473	0.686568535099116\\
34474	0.69000166091802\\
34477	0.659383694360841\\
34478	0.630990131778944\\
34479	0.505684920003945\\
34480	0.516227956982211\\
34481	0.425978271630263\\
34485	0.433513749895124\\
34486	0.376525544664917\\
34487	0.44462137843492\\
34488	0.473855502774695\\
34491	0.490980140150997\\
34492	0.493513785165064\\
34493	0.534382763392086\\
34494	0.517028209868972\\
34495	0.548950559920934\\
34498	0.500030983220919\\
34499	0.531561709207467\\
34500	0.538381353412985\\
34501	0.514924139866118\\
34502	0.503167038486898\\
34505	0.435406094282297\\
34506	0.386139888630492\\
34507	0.409457291214676\\
34508	0.388410227592052\\
34509	0.294631021101541\\
34512	0.356367261545804\\
34513	0.372306379235103\\
34514	0.4118555912073\\
34515	0.377994324551575\\
34519	0.43977856343046\\
34520	0.428077683335678\\
34521	0.407612455959089\\
34522	0.429063519269346\\
34523	0.419829519384839\\
34526	0.445365337470382\\
34527	0.420953307634199\\
34528	0.469847431241571\\
34529	0.522083220453858\\
34530	0.556069577425339\\
34533	0.56188056000046\\
34534	0.569671657152762\\
34535	0.55669696038872\\
34536	0.578384806666532\\
34537	0.599598494501527\\
34540	0.584779541319038\\
34541	0.603440462902617\\
34542	0.560706330446243\\
34543	0.573582283486356\\
34544	0.557156312331682\\
34548	0.644079477920063\\
34549	0.645927384224428\\
34550	0.63539467141608\\
34551	0.653530486673504\\
34554	0.655945166208026\\
34555	0.653072875879005\\
34556	0.652437835532354\\
34557	0.626189122796168\\
34558	0.623244134662775\\
34561	0.626402300500422\\
34562	0.633390444499571\\
34563	0.685683979475562\\
34564	0.676905863346205\\
34565	0.693320645557735\\
34568	0.653412720445488\\
34569	0.65352854186736\\
34570	0.690446222553974\\
34571	0.726220780871512\\
34572	0.754762510140358\\
34576	0.743437854298109\\
34577	0.742660631458483\\
34578	0.705301897480477\\
34579	0.710823981919938\\
34582	0.731528610193983\\
34583	0.690208922401959\\
34584	0.692469596693064\\
34585	0.65969049762665\\
34586	0.601268123879916\\
34589	0.589949797190339\\
34590	0.577455955659449\\
34591	0.518595806050271\\
34592	0.576078473826311\\
34593	0.495023114271983\\
34596	0.515664180420647\\
34597	0.437687914577179\\
34598	0.403316314848041\\
34599	0.412484797990836\\
34600	0.424820403706024\\
34603	0.376146729818786\\
34604	0.385464075262633\\
34605	0.436693318477653\\
34606	0.35763920666191\\
34607	0.40477620982725\\
34610	0.338717576637876\\
34611	0.362684663339681\\
34612	0.29015533046978\\
34613	0.334396960680166\\
34614	0.353407672583265\\
34617	0.407627399126886\\
34618	0.482120868252243\\
34619	0.527250973283033\\
34620	0.604899487416122\\
34621	0.536024781936823\\
34624	0.562136292639741\\
34625	0.493277385993902\\
34626	0.445434636193993\\
34627	0.449746836106292\\
34628	0.39237378230013\\
34631	0.397336579726216\\
34632	0.335774804865624\\
34633	0.345316711313301\\
34634	0.385679130138341\\
34635	0.484861456222076\\
34638	0.507137359722824\\
34639	0.501070242142594\\
34640	0.480577358574668\\
34641	0.514552872508515\\
34642	0.505388961153302\\
34645	0.437323214120544\\
34646	0.420976878219958\\
34647	0.502703498437326\\
34648	0.507207519650982\\
34649	0.44322490898843\\
34652	0.466149431612023\\
34653	0.570748354054997\\
34654	0.592003036828275\\
34655	0.548066217679276\\
34656	0.555855521064136\\
34659	0.531395968284997\\
34660	0.433347913241466\\
34661	0.309312363902631\\
34662	0.345898683192446\\
34663	0.330761420855599\\
34666	0.351910327183427\\
34667	0.383289145114473\\
34668	0.422658239136289\\
34669	0.318756099407182\\
34670	0.260476106400196\\
34673	0.299216833133494\\
34674	0.251665089495483\\
34675	0.233530916833175\\
34676	0.220718380287273\\
34677	0.146591410523582\\
34680	0.0581624916806191\\
34681	0.0433064507174286\\
34682	0.0528720889047334\\
34683	0.0190744367411827\\
34684	0.0141985539164656\\
};
\addplot [color=mycolor3, forget plot]
  table[row sep=crcr]{%
34445	0.520647697076636\\
34446	0.567698305603615\\
34449	0.524942008253809\\
34450	0.54993838463513\\
34451	0.582578400726176\\
34452	0.555260215988625\\
34453	0.547069592886904\\
34457	0.50773263601737\\
34458	0.463121126361781\\
34459	0.515372679171897\\
34460	0.515370392344935\\
34463	0.501743622259798\\
34464	0.551245795120757\\
34465	0.545208903550107\\
34466	0.557576646546735\\
34467	0.529856830307747\\
34470	0.517937106291341\\
34471	0.533451322935662\\
34472	0.521592012181667\\
34473	0.531040828612618\\
34474	0.533404870280733\\
34477	0.502964111741957\\
34478	0.479036261440228\\
34479	0.372769480738619\\
34480	0.376717255617547\\
34481	0.319030279141947\\
34485	0.335445672940752\\
34486	0.291016774035816\\
34487	0.341386479255226\\
34488	0.381789345123312\\
34491	0.394245735683519\\
34492	0.398019206670812\\
34493	0.430684928310807\\
34494	0.415464487370448\\
34495	0.441341800203235\\
34498	0.400175813745502\\
34499	0.425602549620025\\
34500	0.428779568293602\\
34501	0.409033422772373\\
34502	0.39902853031883\\
34505	0.342455035911432\\
34506	0.303218154938922\\
34507	0.320730956439683\\
34508	0.304054217905522\\
34509	0.231367015677237\\
34512	0.263285646489981\\
34513	0.282727466071933\\
34514	0.316645845956992\\
34515	0.287582227962435\\
34519	0.340351007824645\\
34520	0.329687021797589\\
34521	0.311039700666066\\
34522	0.33111717772841\\
34523	0.322095972723195\\
34526	0.34298163384774\\
34527	0.320796758768961\\
34528	0.364311922628227\\
34529	0.415651347801274\\
34530	0.447112060849972\\
34533	0.454852839001073\\
34534	0.465668882564293\\
34535	0.453595375979376\\
34536	0.471724257315127\\
34537	0.490601194511239\\
34540	0.478346754032467\\
34541	0.494098009675316\\
34542	0.457264918931526\\
34543	0.468007625044798\\
34544	0.454284167104219\\
34548	0.526821554716991\\
34549	0.530714754711936\\
34550	0.520968775950545\\
34551	0.536632170814738\\
34554	0.537959685706714\\
34555	0.53535826462144\\
34556	0.53389757371518\\
34557	0.508740941256796\\
34558	0.505709635018605\\
34561	0.504624681222236\\
34562	0.508913726706973\\
34563	0.554673818806712\\
34564	0.545361156732119\\
34565	0.556012662749315\\
34568	0.525694375713977\\
34569	0.532803433218948\\
34570	0.569576656297648\\
34571	0.61152490716504\\
34572	0.642960177260922\\
34576	0.628310731497715\\
34577	0.626338952431971\\
34578	0.581974686442009\\
34579	0.589112288719734\\
34582	0.610971086545151\\
34583	0.56247424673274\\
34584	0.56331595034082\\
34585	0.525788747462147\\
34586	0.464911285338242\\
34589	0.455248446952169\\
34590	0.44142774960908\\
34591	0.379488393577306\\
34592	0.424197605519619\\
34593	0.354025405397876\\
34596	0.367197482123923\\
34597	0.302743187755039\\
34598	0.273525326647508\\
34599	0.293461055558465\\
34600	0.301372969056088\\
34603	0.262526548328035\\
34604	0.281963533709691\\
34605	0.318766942250319\\
34606	0.260058603219044\\
34607	0.289322405483526\\
34610	0.241270046037857\\
34611	0.259233221059067\\
34612	0.212788848547058\\
34613	0.240854538465701\\
34614	0.25244643397779\\
34617	0.288009512417951\\
34618	0.341086566714252\\
34619	0.376542733793164\\
34620	0.450623315550112\\
34621	0.394373660104376\\
34624	0.412458738198766\\
34625	0.357944089101386\\
34626	0.327076714154165\\
34627	0.328886143578778\\
34628	0.291347876433734\\
34631	0.288322272362202\\
34632	0.242363420431098\\
34633	0.26211925054401\\
34634	0.291873374059985\\
34635	0.361291851879763\\
34638	0.371336382530866\\
34639	0.370741303991571\\
34640	0.353785518920169\\
34641	0.379252434264979\\
34642	0.373843796864816\\
34645	0.326975266447802\\
34646	0.329907034649923\\
34647	0.378302465281018\\
34648	0.383198663598064\\
34649	0.34009118183453\\
34652	0.349983025888468\\
34653	0.41351055618701\\
34654	0.430739206411821\\
34655	0.398469772277165\\
34656	0.40129931891109\\
34659	0.376785686508379\\
34660	0.31165069130904\\
34661	0.234675711826451\\
34662	0.265818763835469\\
34663	0.253073155470262\\
34666	0.259506796476056\\
34667	0.282014337634756\\
34668	0.302021311052993\\
34669	0.236484527160554\\
34670	0.215659302273042\\
34673	0.210941499213587\\
34674	0.175680161832472\\
34675	0.15712336199418\\
34676	0.143183699276708\\
34677	0.0892316026657941\\
34680	0.0224370355493471\\
34681	0.0135785936193214\\
34682	0.0120650891318388\\
34683	0.00221426159398779\\
34684	0.000304058050317172\\
};
\addplot [color=mycolor3, forget plot]
  table[row sep=crcr]{%
34445	0.401608589632635\\
34446	0.440780664158914\\
34449	0.404422117330754\\
34450	0.426840066384995\\
34451	0.454697448527883\\
34452	0.429549539630904\\
34453	0.423033666397651\\
34457	0.389947469721909\\
34458	0.354926258704427\\
34459	0.399462486524513\\
34460	0.403543769009252\\
34463	0.391350250910562\\
34464	0.431362473535507\\
34465	0.428168191653177\\
34466	0.437864399392187\\
34467	0.415094785284988\\
34470	0.405881858243253\\
34471	0.417939085586321\\
34472	0.408742903511943\\
34473	0.415207326550627\\
34474	0.416772500782769\\
34477	0.391027206068819\\
34478	0.371073711737707\\
34479	0.287727760757154\\
34480	0.293988983209916\\
34481	0.239735899991359\\
34485	0.239403697503638\\
34486	0.202957970010765\\
34487	0.243740756626638\\
34488	0.26000169536668\\
34491	0.295705698713633\\
34492	0.299197847853143\\
34493	0.328561376298857\\
34494	0.317246886749378\\
34495	0.343474271915716\\
34498	0.313374476830745\\
34499	0.334195871524754\\
34500	0.33811297820605\\
34501	0.320207604048381\\
34502	0.309500890317051\\
34505	0.261473565242904\\
34506	0.226803434244544\\
34507	0.242417815821387\\
34508	0.228230883057786\\
34509	0.169426665572092\\
34512	0.206365454496282\\
34513	0.234203465714181\\
34514	0.261682511662112\\
34515	0.23931710126784\\
34519	0.281429345903791\\
34520	0.272570917946481\\
34521	0.257034017941368\\
34522	0.273537500249348\\
34523	0.270502179962569\\
34526	0.285916542525404\\
34527	0.270274600012912\\
34528	0.304945630872809\\
34529	0.344984181683462\\
34530	0.374757497220074\\
34533	0.378397992370393\\
34534	0.383645253902118\\
34535	0.372136454026541\\
34536	0.387311046547849\\
34537	0.404523385199202\\
34540	0.391389792977007\\
34541	0.405476194022422\\
34542	0.371832292291025\\
34543	0.382780580125707\\
34544	0.370246100645331\\
34548	0.43496459397245\\
34549	0.440773863752094\\
34550	0.431468657925444\\
34551	0.445814739766187\\
34554	0.445912261139898\\
34555	0.443406232982342\\
34556	0.442154729057093\\
34557	0.419374137465961\\
34558	0.416002976631809\\
34561	0.412457765607465\\
34562	0.416337520720221\\
34563	0.456086271405736\\
34564	0.448190229554258\\
34565	0.456610837638831\\
34568	0.42824725547286\\
34569	0.428714025927285\\
34570	0.45691742390605\\
34571	0.492696164675592\\
34572	0.520711258421204\\
34576	0.505365556097436\\
34577	0.502343704516129\\
34578	0.459018466452754\\
34579	0.466682857653078\\
34582	0.485668932556379\\
34583	0.440421519813409\\
34584	0.440990502734798\\
34585	0.406931685458578\\
34586	0.354477938845757\\
34589	0.352134309035844\\
34590	0.337974073051566\\
34591	0.277087952183083\\
34592	0.306408837416712\\
34593	0.248383670023545\\
34596	0.289480620421251\\
34597	0.240876203734483\\
34598	0.239543811562535\\
34599	0.274303826627057\\
34600	0.277868344513105\\
34603	0.24774880253727\\
34604	0.261197882864394\\
34605	0.285561066863341\\
34606	0.243076764730843\\
34607	0.266055492463632\\
34610	0.23013679773471\\
34611	0.241992800752668\\
34612	0.204707706911896\\
34613	0.225347041748769\\
34614	0.241924742707407\\
34617	0.267592277806466\\
34618	0.309653392956278\\
34619	0.336948364541749\\
34620	0.383420992491553\\
34621	0.341903949485854\\
34624	0.351515602185658\\
34625	0.313086915944623\\
34626	0.28354513159887\\
34627	0.29807242712801\\
34628	0.272397064898317\\
34631	0.290229044001851\\
34632	0.262851906156137\\
34633	0.30865037123348\\
34634	0.328069860001848\\
34635	0.398082931982853\\
34638	0.407493478718349\\
34639	0.402746117187524\\
34640	0.393429192683989\\
34641	0.404332944944025\\
34642	0.39973060465218\\
34645	0.370532922121856\\
34646	0.371689333148288\\
34647	0.396627010824898\\
34648	0.400179383373953\\
34649	0.376874669580673\\
34652	0.378066399040661\\
34653	0.411273235356565\\
34654	0.432381026843349\\
34655	0.415672843871316\\
34656	0.415409475057016\\
34659	0.398607899792956\\
34660	0.365122232458132\\
34661	0.328922149756978\\
34662	0.331413909240028\\
34663	0.322117926834271\\
34666	0.31576599497819\\
34667	0.320633331258431\\
34668	0.330952154847528\\
34669	0.290149942229179\\
34670	0.261849884466953\\
34673	0.244706453028227\\
34674	0.215054103243463\\
34675	0.195686223382927\\
34676	0.181627155139956\\
34677	0.136820143261779\\
34680	0.0572363890894716\\
34681	0.040012625813243\\
34682	0.0322350513512386\\
34683	0.0101862423368003\\
34684	0.0016399580444406\\
};
\addplot [color=mycolor3, forget plot]
  table[row sep=crcr]{%
34445	0.820765440736607\\
34446	0.848200248637211\\
34449	0.829248499982232\\
34450	0.841509632886874\\
34451	0.855570040129843\\
34452	0.843364907523875\\
34453	0.838557475376901\\
34457	0.820551871778648\\
34458	0.797689148672584\\
34459	0.825456722299007\\
34460	0.822764532865728\\
34463	0.81819676129829\\
34464	0.84368660932311\\
34465	0.844393460526075\\
34466	0.8531484227637\\
34467	0.839417655518534\\
34470	0.834238173739671\\
34471	0.852199111604215\\
34472	0.847502562814829\\
34473	0.856692598465613\\
34474	0.863696933434354\\
34477	0.850808360539445\\
34478	0.836082337597928\\
34479	0.765722287412918\\
34480	0.767112450466251\\
34481	0.702430499124933\\
34485	0.708771850336991\\
34486	0.661758768457255\\
34487	0.711656466643916\\
34488	0.729028275518908\\
34491	0.75144626394853\\
34492	0.755689820094227\\
34493	0.793740270647436\\
34494	0.783263814327416\\
34495	0.808875670285567\\
34498	0.769233082303555\\
34499	0.799744879051867\\
34500	0.806605172781533\\
34501	0.788907254622631\\
34502	0.778156343602439\\
34505	0.718733548612637\\
34506	0.673215508356996\\
34507	0.697164577404697\\
34508	0.678849913592119\\
34509	0.581310700082739\\
34512	0.617825200025926\\
34513	0.636163218882522\\
34514	0.691860613536765\\
34515	0.645360461258497\\
34519	0.726950073853589\\
34520	0.713336089178502\\
34521	0.691271770836973\\
34522	0.719098945902886\\
34523	0.716936137643181\\
34526	0.753607013995443\\
34527	0.721038973149925\\
34528	0.781495496506492\\
34529	0.831822058802038\\
34530	0.865675351102766\\
34533	0.861387240834545\\
34534	0.870070413795007\\
34535	0.862198958177767\\
34536	0.868655284566731\\
34537	0.880672665167196\\
34540	0.866644147216815\\
34541	0.880629336515958\\
34542	0.847793904345833\\
34543	0.852876945150998\\
34544	0.840925299757767\\
34548	0.904620431370087\\
34549	0.894566986688793\\
34550	0.888569703913595\\
34551	0.89975166855557\\
34554	0.903444190766705\\
34555	0.900041163760309\\
34556	0.899485368135349\\
34557	0.885624867980805\\
34558	0.883801881321248\\
34561	0.885340668078371\\
34562	0.879466010338623\\
34563	0.90908319122772\\
34564	0.902798623815227\\
34565	0.912664731688438\\
34568	0.902089553463826\\
34569	0.903540986764466\\
34570	0.921855146145285\\
34571	0.93669246017422\\
34572	0.948705836833752\\
34576	0.940558603367731\\
34577	0.940785149247944\\
34578	0.924240911659623\\
34579	0.919414031817688\\
34582	0.931006547012538\\
34583	0.909184475018906\\
34584	0.910175119450752\\
34585	0.894874181844599\\
34586	0.860759093967034\\
34589	0.846033518119914\\
34590	0.83926556111586\\
34591	0.802198488834214\\
34592	0.834310932301205\\
34593	0.785401341693529\\
34596	0.802926635789425\\
34597	0.74666761889144\\
34598	0.718505178830165\\
34599	0.724958170992144\\
34600	0.737434243065976\\
34603	0.702638581120084\\
34604	0.711468268427448\\
34605	0.754765884523639\\
34606	0.692686840285937\\
34607	0.734035360374534\\
34610	0.672802911240772\\
34611	0.697932022678697\\
34612	0.633196021422536\\
34613	0.671108064237154\\
34614	0.692277461135186\\
34617	0.743463513681522\\
34618	0.802507065418845\\
34619	0.832994531808518\\
34620	0.864713702971549\\
34621	0.818511210228806\\
34624	0.835204803232317\\
34625	0.795364381490225\\
34626	0.768368384122614\\
34627	0.771633060927314\\
34628	0.731895555589997\\
34631	0.736041159738995\\
34632	0.687991252425667\\
34633	0.675709002782663\\
34634	0.704642487673185\\
34635	0.770128377313856\\
34638	0.791438907184673\\
34639	0.787553370635124\\
34640	0.776804873754722\\
34641	0.802122876966945\\
34642	0.800320680926791\\
34645	0.765832001758082\\
34646	0.759737996335509\\
34647	0.809110291373784\\
34648	0.809139106414439\\
34649	0.776423686415199\\
34652	0.804671492276907\\
34653	0.861112055591924\\
34654	0.852839606486597\\
34655	0.838178389711203\\
34656	0.845361642638336\\
34659	0.846889378199053\\
34660	0.801521004634127\\
34661	0.723321912713707\\
34662	0.731223502112109\\
34663	0.728936546995756\\
34666	0.763624749484121\\
34667	0.78423316496822\\
34668	0.817042140607433\\
34669	0.749737230678098\\
34670	0.715679896068486\\
34673	0.762897067901562\\
34674	0.73611872447381\\
34675	0.739820190077439\\
34676	0.749806866616116\\
34677	0.671719399465624\\
34680	0.584186291439979\\
34681	0.609651100935619\\
34682	0.75281828548199\\
34683	0.77415358785267\\
34684	0.882701200380586\\
};
\addplot [color=mycolor3, forget plot]
  table[row sep=crcr]{%
34445	0.749759979679722\\
34446	0.785563487113856\\
34449	0.759792055026536\\
34450	0.776287927253598\\
34451	0.797071071230285\\
34452	0.779874318959352\\
34453	0.773349397820835\\
34457	0.74855275600677\\
34458	0.716600742662558\\
34459	0.753096259299771\\
34460	0.749686767791389\\
34463	0.742816728104805\\
34464	0.776763214204755\\
34465	0.77104264200423\\
34466	0.781110002341498\\
34467	0.762748128866768\\
34470	0.755411241216125\\
34471	0.775511794304541\\
34472	0.768674341071425\\
34473	0.776649335862229\\
34474	0.780150610994805\\
34477	0.760280486948155\\
34478	0.74154983083771\\
34479	0.648423666565543\\
34480	0.661122813157701\\
34481	0.573994392955428\\
34485	0.580428659805004\\
34486	0.525533224399334\\
34487	0.587630366326813\\
34488	0.616406100366796\\
34491	0.63582737354543\\
34492	0.635156377400967\\
34493	0.678576916432367\\
34494	0.660868691125825\\
34495	0.694776379706311\\
34498	0.644627196345798\\
34499	0.681820169109741\\
34500	0.689971961080902\\
34501	0.66442561428782\\
34502	0.651695006902536\\
34505	0.575447807982497\\
34506	0.518985303338042\\
34507	0.548102280323584\\
34508	0.523993335356705\\
34509	0.397725660247724\\
34512	0.438001413440417\\
34513	0.457928188064575\\
34514	0.522577751775409\\
34515	0.46698216973907\\
34519	0.567826104680122\\
34520	0.5501292007565\\
34521	0.518590580459589\\
34522	0.553691365178342\\
34523	0.541917410535158\\
34526	0.583877286426583\\
34527	0.544194363997342\\
34528	0.620440473598114\\
34529	0.695860987206863\\
34530	0.728741335181104\\
34533	0.73753804776213\\
34534	0.748927832314124\\
34535	0.729626683442\\
34536	0.757441673285943\\
34537	0.78287306177649\\
34540	0.763236836811425\\
34541	0.784986634512284\\
34542	0.735831743441858\\
34543	0.752059249226374\\
34544	0.733852883219547\\
34548	0.83114859948063\\
34549	0.82162844264138\\
34550	0.810509334051074\\
34551	0.82703521359545\\
34554	0.831456963820577\\
34555	0.826820902739917\\
34556	0.82623510412695\\
34557	0.797870166890264\\
34558	0.795872764479007\\
34561	0.790365975854563\\
34562	0.80182434697462\\
34563	0.848848413303093\\
34564	0.836584288457553\\
34565	0.845067935482577\\
34568	0.815607849853274\\
34569	0.822573238316893\\
34570	0.852122329896864\\
34571	0.880081313539332\\
34572	0.902231990148165\\
34576	0.886238743248391\\
34577	0.886445714554114\\
34578	0.855823213906995\\
34579	0.85532533339714\\
34582	0.874854769929913\\
34583	0.839172577842057\\
34584	0.841416504742079\\
34585	0.815607517616279\\
34586	0.757470795385262\\
34589	0.746241469907751\\
34590	0.735216264599484\\
34591	0.678547891468784\\
34592	0.728665578162773\\
34593	0.659354911092194\\
34596	0.680453268738272\\
34597	0.607703417449421\\
34598	0.572287426701342\\
34599	0.579472676655312\\
34600	0.594051257309149\\
34603	0.546440228769755\\
34604	0.557387516701906\\
34605	0.612128341170012\\
34606	0.530148835481366\\
34607	0.580823851468198\\
34610	0.506898613873093\\
34611	0.539100496866223\\
34612	0.455215404839208\\
34613	0.505810136808896\\
34614	0.528446860439267\\
34617	0.593244506741869\\
34618	0.665614072125711\\
34619	0.704935234505159\\
34620	0.774681006611327\\
34621	0.711054422220646\\
34624	0.736605318343931\\
34625	0.676404983434698\\
34626	0.632908542583055\\
34627	0.6376897675546\\
34628	0.57965513642173\\
34631	0.583514642615699\\
34632	0.514160964145692\\
34633	0.521488795883429\\
34634	0.572682954867834\\
34635	0.68393793226773\\
34638	0.714300681697641\\
34639	0.706340038695305\\
34640	0.68605550958539\\
34641	0.730808398281782\\
34642	0.721698786025192\\
34645	0.659969459635703\\
34646	0.643398013671745\\
34647	0.7293467661225\\
34648	0.730997659832175\\
34649	0.669310121335771\\
34652	0.71028589829774\\
34653	0.810588153694891\\
34654	0.813459136138354\\
34655	0.781524409333106\\
34656	0.78761051606026\\
34659	0.779140292496488\\
34660	0.689698027250248\\
34661	0.545346015322248\\
34662	0.570226830862512\\
34663	0.558261391606671\\
34666	0.599629828281434\\
34667	0.637520859570256\\
34668	0.688977037077338\\
34669	0.573986532405923\\
34670	0.506629281568808\\
34673	0.550466287637045\\
34674	0.488383829690695\\
34675	0.478001555251139\\
34676	0.476450935196234\\
34677	0.34137738461368\\
34680	0.165464784974905\\
34681	0.154254844135455\\
34682	0.250060507504222\\
34683	0.178838384433694\\
34684	0.386957335789344\\
};
\addplot [color=mycolor3, forget plot]
  table[row sep=crcr]{%
34445	0.636418439382494\\
34446	0.672199233127078\\
34449	0.645424146419995\\
34450	0.666313070510033\\
34451	0.711830121589863\\
34452	0.701515128849023\\
34453	0.693016805826951\\
34457	0.656627194421051\\
34458	0.610898442281804\\
34459	0.664252831504181\\
34460	0.66537714747527\\
34463	0.653389809414757\\
34464	0.700896267281128\\
34465	0.694248132462064\\
34466	0.708024669924981\\
34467	0.680441603676799\\
34470	0.66847218814452\\
34471	0.685424909312374\\
34472	0.675156710502339\\
34473	0.684031865914117\\
34474	0.685626757048795\\
34477	0.656878863905766\\
34478	0.630003041924559\\
34479	0.505816723254857\\
34480	0.515602291965261\\
34481	0.419313661519281\\
34485	0.429352927127721\\
34486	0.370526883862647\\
34487	0.436405261845719\\
34488	0.463815622295367\\
34491	0.483465750929571\\
34492	0.482588255605678\\
34493	0.534246328043237\\
34494	0.512006841290267\\
34495	0.553153289243604\\
34498	0.489891090951264\\
34499	0.530894192345021\\
34500	0.539775744978042\\
34501	0.509441280151821\\
34502	0.494077055672007\\
34505	0.403889223937941\\
34506	0.342700823762506\\
34507	0.374191474746239\\
34508	0.345768834595604\\
34509	0.208339098155335\\
34512	0.248421399502805\\
34513	0.265682980517332\\
34514	0.330079492348654\\
34515	0.27582163098539\\
34519	0.373171997323703\\
34520	0.354069206460695\\
34521	0.317088062200991\\
34522	0.353525071978055\\
34523	0.338906007178608\\
34526	0.380145084998954\\
34527	0.337966946630548\\
34528	0.421518051922479\\
34529	0.513609533959912\\
34530	0.571045697421388\\
34533	0.581292565278059\\
34534	0.594801042914069\\
34535	0.573378884465661\\
34536	0.6107786217349\\
34537	0.643362480861577\\
34540	0.619227860421362\\
34541	0.650128498900851\\
34542	0.579462263980915\\
34543	0.601173460218781\\
34544	0.573876114852082\\
34548	0.716533727293701\\
34549	0.712539665265936\\
34550	0.680279722795967\\
34551	0.689213537766809\\
34554	0.693178564795919\\
34555	0.688612972429739\\
34556	0.687177669007296\\
34557	0.652757869682191\\
34558	0.649425978620674\\
34561	0.654485413162892\\
34562	0.661475314036922\\
34563	0.729662348641569\\
34564	0.717187905125713\\
34565	0.731530965713654\\
34568	0.695556156678322\\
34569	0.704776919857363\\
34570	0.74851843590288\\
34571	0.793968910184349\\
34572	0.820696810786013\\
34576	0.809061589199469\\
34577	0.808520522300826\\
34578	0.760638257785983\\
34579	0.762258817920982\\
34582	0.787072415579344\\
34583	0.733590221939691\\
34584	0.734856789909165\\
34585	0.696929097707424\\
34586	0.62207141262718\\
34589	0.604883879347883\\
34590	0.588932994540425\\
34591	0.517667325718836\\
34592	0.579046285672619\\
34593	0.492641519349435\\
34596	0.514480476182818\\
34597	0.428413720929192\\
34598	0.391057198590874\\
34599	0.398356489371697\\
34600	0.412146215997878\\
34603	0.358167938238882\\
34604	0.368959447633845\\
34605	0.426075591094207\\
34606	0.338115198758887\\
34607	0.388686886796019\\
34610	0.309127845190539\\
34611	0.339285464405737\\
34612	0.259291282982953\\
34613	0.307908914997328\\
34614	0.328463432375783\\
34617	0.38934364402812\\
34618	0.475135742803216\\
34619	0.527337380152646\\
34620	0.619411362303847\\
34621	0.537796307126898\\
34624	0.568881455964591\\
34625	0.487305351305372\\
34626	0.428929983405346\\
34627	0.427269582961674\\
34628	0.339055564208566\\
34631	0.34320348334111\\
34632	0.257696783330387\\
34633	0.264590279489965\\
34634	0.320227029234383\\
34635	0.465660346552016\\
34638	0.498945413629286\\
34639	0.487395499846788\\
34640	0.455173085078694\\
34641	0.509151952601038\\
34642	0.494850029816779\\
34645	0.398967538740762\\
34646	0.384037051849559\\
34647	0.49568588361838\\
34648	0.501816159276671\\
34649	0.410977190471286\\
34652	0.444593385175703\\
34653	0.591406088166582\\
34654	0.619052949819127\\
34655	0.558877099433602\\
34656	0.570034855593884\\
34659	0.536600478032426\\
34660	0.399573468384914\\
34661	0.239674668194585\\
34662	0.276687786707937\\
34663	0.25206682914493\\
34666	0.277076043420715\\
34667	0.314472593029477\\
34668	0.377401083874628\\
34669	0.238008419927654\\
34670	0.178352767955398\\
34673	0.188781750324466\\
34674	0.128140693894695\\
34675	0.110192650635428\\
34676	0.0978996489164887\\
34677	0.033280565938255\\
34680	0.00188377170138937\\
34681	0.000830085220167291\\
34682	0.00155330094613699\\
34683	8.31881790216957e-05\\
34684	3.50864703863058e-05\\
};
\addplot [color=mycolor3, forget plot]
  table[row sep=crcr]{%
34445	0.520420822403597\\
34446	0.568092366637297\\
34449	0.524592519617853\\
34450	0.55074141765396\\
34451	0.584832613032718\\
34452	0.556082500870815\\
34453	0.547575961000562\\
34457	0.505755571439957\\
34458	0.456633427792086\\
34459	0.513742145303482\\
34460	0.513621429093221\\
34463	0.498573179795427\\
34464	0.553364282188116\\
34465	0.546591034314698\\
34466	0.560399760025059\\
34467	0.529526044338714\\
34470	0.516330273514686\\
34471	0.533754268452956\\
34472	0.520583520461668\\
34473	0.531135878226761\\
34474	0.533856241861204\\
34477	0.496243217446249\\
34478	0.464236482356001\\
34479	0.329020139250118\\
34480	0.341410641416682\\
34481	0.243275283254601\\
34485	0.25614435372767\\
34486	0.202037507685231\\
34487	0.256676319682617\\
34488	0.278503476637083\\
34491	0.297668269665953\\
34492	0.297403956056021\\
34493	0.347731297740457\\
34494	0.320842073142725\\
34495	0.364217290586958\\
34498	0.293356076034716\\
34499	0.333794146333947\\
34500	0.341389737044468\\
34501	0.308187285274504\\
34502	0.291660224668346\\
34505	0.204936029313754\\
34506	0.149052348843239\\
34507	0.174705646517797\\
34508	0.153264894369269\\
34509	0.0723160550784112\\
34512	0.0970603435745747\\
34513	0.111691972678603\\
34514	0.15255670331432\\
34515	0.114586754726026\\
34519	0.181498698501638\\
34520	0.166080316322916\\
34521	0.132799496976307\\
34522	0.154118730137098\\
34523	0.146347717195027\\
34526	0.174233408654752\\
34527	0.143263863554897\\
34528	0.206218987242793\\
34529	0.288963580474963\\
34530	0.356326132727453\\
34533	0.366528101698294\\
34534	0.380363312741709\\
34535	0.35613897328865\\
34536	0.394713531015115\\
34537	0.435328019984094\\
34540	0.407801009428797\\
34541	0.442542714445398\\
34542	0.363398327039467\\
34543	0.382987662021119\\
34544	0.354428319401477\\
34548	0.516185090608753\\
34549	0.524217975423728\\
34550	0.502880698023122\\
34551	0.537629617079145\\
34554	0.541090665826354\\
34555	0.535254475769681\\
34556	0.532949417791465\\
34557	0.480304696755695\\
34558	0.474890257820116\\
34561	0.466713897326571\\
34562	0.481179056621504\\
34563	0.582390203905571\\
34564	0.561960473645818\\
34565	0.584106971825453\\
34568	0.522468493364331\\
34569	0.536598134232796\\
34570	0.608031785856258\\
34571	0.68306960074759\\
34572	0.739545520815763\\
34576	0.707148064441834\\
34577	0.703971206847631\\
34578	0.626578054008688\\
34579	0.635341028657929\\
34582	0.673514074385957\\
34583	0.588950363820115\\
34584	0.587398561369214\\
34585	0.523797577805704\\
34586	0.418377482248194\\
34589	0.403900064517352\\
34590	0.381882471597615\\
34591	0.289737526059875\\
34592	0.365503581252564\\
34593	0.306064727484487\\
34596	0.324684974339858\\
34597	0.271043831305644\\
34598	0.241261844089784\\
34599	0.246350557874253\\
34600	0.256145215956904\\
34603	0.211838396968457\\
34604	0.220216341704081\\
34605	0.263595618125271\\
34606	0.195558481176027\\
34607	0.232736077749764\\
34610	0.170733885792662\\
34611	0.190950523813356\\
34612	0.135317865947973\\
34613	0.165048146566944\\
34614	0.178270524730066\\
34617	0.219570006809039\\
34618	0.285262382813243\\
34619	0.332443030918614\\
34620	0.419428503265872\\
34621	0.343669128256171\\
34624	0.367967985854659\\
34625	0.294880291795961\\
34626	0.245028083387465\\
34627	0.248920824303192\\
34628	0.19650098235471\\
34631	0.193406456278709\\
34632	0.140880600895322\\
34633	0.143328268433908\\
34634	0.175105284345757\\
34635	0.266098997486669\\
34638	0.288044108213236\\
34639	0.276672441866161\\
34640	0.252490303292663\\
34641	0.287621669038721\\
34642	0.27587018468039\\
34645	0.203899135531151\\
34646	0.197171448682075\\
34647	0.268536232380326\\
34648	0.27192239022622\\
34649	0.20724876156273\\
34652	0.216097107046454\\
34653	0.319075107287363\\
34654	0.355728152920027\\
34655	0.301256552356355\\
34656	0.307558079651491\\
34659	0.268038909530808\\
34660	0.171278306340904\\
34661	0.0848156334776429\\
34662	0.100564238279187\\
34663	0.0855511352807775\\
34666	0.0867238373716111\\
34667	0.102959065497104\\
34668	0.124744964308158\\
34669	0.0406613087061668\\
34670	0.0247422089396647\\
34673	0.0139528905881918\\
34674	0.00544382251446224\\
34675	0.00328584300145633\\
34676	0.00205726450746521\\
34677	0.0001736303239331\\
34680	8.78475008208822e-08\\
34681	4.98429983269539e-09\\
34682	2.68883328197304e-09\\
34683	1.16958662178294e-13\\
34684	1.14308491134274e-18\\
};
\addplot [color=mycolor3, forget plot]
  table[row sep=crcr]{%
34445	0.362446290536349\\
34446	0.414129149407188\\
34449	0.365905822311568\\
34450	0.393353826714035\\
34451	0.429383645758358\\
34452	0.397913305010392\\
34453	0.389573053908002\\
34457	0.345951502924806\\
34458	0.297959725663118\\
34459	0.352190234327747\\
34460	0.349036942211101\\
34463	0.331659003828278\\
34464	0.38682612881716\\
34465	0.378520620770169\\
34466	0.391274677911097\\
34467	0.35848912262206\\
34470	0.34403132586384\\
34471	0.350948950294834\\
34472	0.336404844773299\\
34473	0.344697233503045\\
34474	0.345010194145009\\
34477	0.304442383722532\\
34478	0.27450072010571\\
34479	0.15934569144953\\
34480	0.1711396281198\\
34481	0.100693339882333\\
34485	0.110692669291102\\
34486	0.0778215328744693\\
34487	0.112266112957315\\
34488	0.130661249887759\\
34491	0.129720242355201\\
34492	0.128437667781152\\
34493	0.165373453400832\\
34494	0.142986795053424\\
34495	0.17505694698622\\
34498	0.123919780864311\\
34499	0.145710986734724\\
34500	0.150863623694573\\
34501	0.126916370929305\\
34502	0.115773369051838\\
34505	0.0652321730280604\\
34506	0.0375918701764713\\
34507	0.0486277196915443\\
34508	0.03917804588584\\
34509	0.0112092665808476\\
34512	0.018479904664552\\
34513	0.0235418104639208\\
34514	0.0385607145097398\\
34515	0.0242030243455847\\
34519	0.0495821997654306\\
34520	0.0429511946651728\\
34521	0.0296097551338087\\
34522	0.0367107808557372\\
34523	0.0338479990930527\\
34526	0.0435493942414437\\
34527	0.0315407558633135\\
34528	0.056953647924511\\
34529	0.102324629160043\\
34530	0.146308704841924\\
34533	0.15264245686445\\
34534	0.160512600501214\\
34535	0.138249328718882\\
34536	0.164119299606009\\
34537	0.197432513543912\\
34540	0.17670487637049\\
34541	0.204611594258909\\
34542	0.143230479548283\\
34543	0.161092268788606\\
34544	0.139419904454497\\
34548	0.269003552702719\\
34549	0.287102905027569\\
34550	0.265711911126289\\
34551	0.299319629573751\\
34554	0.29999732829819\\
34555	0.295624627231058\\
34556	0.293419799648658\\
34557	0.240467596738307\\
34558	0.237263870897867\\
34561	0.212069553174385\\
34562	0.228451265142093\\
34563	0.312736840457555\\
34564	0.293141653890813\\
34565	0.314970894655796\\
34568	0.248849343096183\\
34569	0.264932195208201\\
34570	0.335938566407631\\
34571	0.429667161260735\\
34572	0.50460304331832\\
34576	0.467762717979072\\
34577	0.465325943982542\\
34578	0.367967550203205\\
34579	0.388505659544874\\
34582	0.430629789736995\\
34583	0.333274140192369\\
34584	0.341093402421839\\
34585	0.273309791787303\\
34586	0.182639018786588\\
34589	0.176794980265221\\
34590	0.158404349634143\\
34591	0.0971598626157922\\
34592	0.145450926240154\\
34593	0.0878667562174205\\
34596	0.0977538828616825\\
34597	0.0544662226781708\\
34598	0.0416137197969267\\
34599	0.0438551352441588\\
34600	0.0474626629259739\\
34603	0.0285939575609134\\
34604	0.0321155987106012\\
34605	0.0497020598446376\\
34606	0.0230241074760983\\
34607	0.0344372515362022\\
34610	0.0149127625752371\\
34611	0.0196579545303344\\
34612	0.00802556520786355\\
34613	0.014106200244296\\
34614	0.0171740798964659\\
34617	0.0272954815025848\\
34618	0.051568988425558\\
34619	0.0794394056989299\\
34620	0.142595115559933\\
34621	0.0861744629691929\\
34624	0.106337278464788\\
34625	0.061372925483826\\
34626	0.0369351925736493\\
34627	0.0380239777218981\\
34628	0.0204072149338837\\
34631	0.0190970117557362\\
34632	0.00808414465004946\\
34633	0.00840650602895919\\
34634	0.013670622971255\\
34635	0.0422657693069782\\
34638	0.052174318343603\\
34639	0.0448955127515382\\
34640	0.0349353119817455\\
34641	0.0469080484426777\\
34642	0.0414953974677227\\
34645	0.0176300900377301\\
34646	0.0168079759405529\\
34647	0.0364328284848008\\
34648	0.0372193011320464\\
34649	0.0173623660435848\\
34652	0.01587931282879\\
34653	0.0446504338780471\\
34654	0.0705007765228182\\
34655	0.0415145719441791\\
34656	0.0425844417385839\\
34659	0.0252915519103693\\
34660	0.00908411860555829\\
34661	0.00143047958765996\\
34662	0.00242527901909657\\
34663	0.00134512627639514\\
34666	0.00109230934036054\\
34667	0.00152766688010207\\
34668	0.00222860603130567\\
34669	0.000213094655559636\\
34670	6.05872276218241e-05\\
34673	2.22737363955701e-05\\
34674	2.29539920683281e-06\\
34675	5.5652560505858e-07\\
34676	1.37665091531822e-07\\
34677	4.8199256934686e-10\\
34680	2.4481616168408e-18\\
34681	1.8331435191199e-21\\
34682	1.42931304303289e-23\\
34683	5.87435439604487e-35\\
34684	1.69842232821915e-54\\
};
\end{axis}

\begin{axis}[%
width=3.888in,
height=1.366in,
at={(0.703in,1.014in)},
scale only axis,
xmin=34445,
xmax=34684,
xtick={34425,34455,34486,34516,34547,34578,34608,34639,34669,34700},
xticklabels={{01/04},{01/05},{01/06},{01/07},{01/08},{01/09},{01/10},{01/11},{01/12},{01/01}},
xlabel style={font=\color{white!15!black}},
xlabel={Date (dd/mm)},
ymin=1.38294515684673e-54,
ymax=0.00979457822350064,
ylabel style={font=\color{white!15!black}},
ylabel={Value},
axis background/.style={fill=white},
title style={font=\bfseries},
title={$\Gamma\text{: Sensitivity to Underlying Delta Change}$},
xmajorgrids,
ymajorgrids
]
\addplot [color=mycolor4, forget plot]
  table[row sep=crcr]{%
34445	0.000509198878927506\\
34446	0.000361028644306322\\
34449	0.000485653143700924\\
34450	0.000414509139907036\\
34451	0.000323013362236267\\
34452	0.000388170244544841\\
34453	0.000416254669705978\\
34457	0.000544275619636002\\
34458	0.000731248523614158\\
34459	0.000537349087760047\\
34460	0.000556412332616497\\
34463	0.000593550916527007\\
34464	0.000430097764419088\\
34465	0.000456364998497487\\
34466	0.000415525503619661\\
34467	0.000500103690344183\\
34470	0.000538987968223209\\
34471	0.000456779420269409\\
34472	0.000488542984791967\\
34473	0.000444878412324633\\
34474	0.000424402132485693\\
34477	0.000520359969711313\\
34478	0.000628087536489547\\
34479	0.0011779759269697\\
34480	0.00116796199170659\\
34481	0.00160996972798087\\
34485	0.0015338301162502\\
34486	0.00176947923400071\\
34487	0.00150262879911991\\
34488	0.00138924964772832\\
34491	0.00130151824868568\\
34492	0.00130379185538373\\
34493	0.00108649556410802\\
34494	0.00117850748801114\\
34495	0.00101272930451922\\
34498	0.00126680728029676\\
34499	0.00109216539813013\\
34500	0.00105390941733271\\
34501	0.00117806284394734\\
34502	0.00124237435827138\\
34505	0.00163170711826112\\
34506	0.00191316133240475\\
34507	0.00177509602490847\\
34508	0.00188968842262366\\
34509	0.00243147484999255\\
34512	0.00231133942013519\\
34513	0.00224513199774954\\
34514	0.00197306039594837\\
34515	0.00223586372602892\\
34519	0.00175198094166394\\
34520	0.00184160642079199\\
34521	0.00200470755240466\\
34522	0.00183805986884974\\
34523	0.0018866442429973\\
34526	0.00169180413176121\\
34527	0.0018867141396413\\
34528	0.00149865208448899\\
34529	0.00111672666114995\\
34530	0.000911111548750976\\
34533	0.000870797126957889\\
34534	0.000825770255283588\\
34535	0.000901986799286974\\
34536	0.000762153402561599\\
34537	0.000654332877145467\\
34540	0.000739235317532493\\
34541	0.000639590667589224\\
34542	0.000874999197505645\\
34543	0.00080916056066084\\
34544	0.000899343598997229\\
34548	0.000450972992957555\\
34549	0.000462688078097266\\
34550	0.000504964993647509\\
34551	0.000428592749529883\\
34554	0.000410437783193868\\
34555	0.000421644650122797\\
34556	0.000421716051883935\\
34557	0.000523657495693337\\
34558	0.000540923526910715\\
34561	0.000448173163042556\\
34562	0.000427097040361694\\
34563	0.000176196019593065\\
34564	0.000201706489233644\\
34565	0.000171889209324458\\
34568	0.000241252805448597\\
34569	0.00022367406469548\\
34570	0.000130956246642096\\
34571	7.45951471245079e-05\\
34572	4.06348016089525e-05\\
34576	4.9048889827397e-05\\
34577	5.04882674662309e-05\\
34578	9.64672363137074e-05\\
34579	9.46475532016283e-05\\
34582	6.23436812395646e-05\\
34583	0.000123801056130529\\
34584	0.000127688277012446\\
34585	0.000192873926209038\\
34586	0.000390530616256045\\
34589	0.000471114551388874\\
34590	0.000525837793312896\\
34591	0.000867727941299082\\
34592	0.000578960388482359\\
34593	0.00107338144067843\\
34596	0.000997062808013524\\
34597	0.00149596567915346\\
34598	0.00175669440595569\\
34599	0.00170440939522422\\
34600	0.001598246432193\\
34603	0.00197451998047052\\
34604	0.00188481408594921\\
34605	0.00147256346707757\\
34606	0.00210416700961717\\
34607	0.00171022000374406\\
34610	0.0023144677919432\\
34611	0.00206200930260923\\
34612	0.00268596100355248\\
34613	0.00233403340436472\\
34614	0.00215564723123608\\
34617	0.00161918478113748\\
34618	0.00100510117041491\\
34619	0.000751178946656756\\
34620	0.0003715084722833\\
34621	0.000696565809741602\\
34624	0.000566157679958611\\
34625	0.000947205292596488\\
34626	0.00126885108129991\\
34627	0.00123474520976407\\
34628	0.00172031289804221\\
34631	0.00168658280562505\\
34632	0.00230903637991228\\
34633	0.00224304197486862\\
34634	0.00180942935432047\\
34635	0.00102502082240543\\
34638	0.000907810249087308\\
34639	0.0009305019005279\\
34640	0.00105798316450355\\
34641	0.000807570088012744\\
34642	0.000854033065214134\\
34645	0.00124543749685556\\
34646	0.00135063868680122\\
34647	0.000808330814344426\\
34648	0.000795958396758461\\
34649	0.0011668746731709\\
34652	0.000925987149394102\\
34653	0.000406244514172504\\
34654	0.000402822453609195\\
34655	0.000543378234775403\\
34656	0.000491261833397718\\
34659	0.000515743975928562\\
34660	0.000992117879012144\\
34661	0.00204348327399046\\
34662	0.0018485184667199\\
34663	0.00193508437946667\\
34666	0.00155199622188222\\
34667	0.00123707977084513\\
34668	0.00084639829602844\\
34669	0.00167224436884397\\
34670	0.00229483465664646\\
34673	0.00174453718151662\\
34674	0.00234054841574386\\
34675	0.00236816850788124\\
34676	0.00228247654507642\\
34677	0.00415202041744912\\
34680	0.00731389441119292\\
34681	0.00721353937092007\\
34682	0.00389536546518401\\
34683	0.00394945599377291\\
34684	4.61134548901798e-05\\
};
\addplot [color=mycolor4, forget plot]
  table[row sep=crcr]{%
34445	0.00116189705804501\\
34446	0.000979747908122343\\
34449	0.00114520446254609\\
34450	0.00104762104168815\\
34451	0.000922633305204096\\
34452	0.00102622810605181\\
34453	0.00105809980456157\\
34457	0.00121106086905388\\
34458	0.00138141014676252\\
34459	0.00117984766248501\\
34460	0.00117902422084964\\
34463	0.00123231409596693\\
34464	0.00104036019039971\\
34465	0.00106353956653062\\
34466	0.00101591443738466\\
34467	0.00112225202666113\\
34470	0.00116729096873574\\
34471	0.00110742649722754\\
34472	0.00115269356776256\\
34473	0.00111575679912253\\
34474	0.00110615784314661\\
34477	0.0012344251746169\\
34478	0.00134268624469756\\
34479	0.00180148378993336\\
34480	0.0017489004738001\\
34481	0.00199641119781329\\
34485	0.00182443779465668\\
34486	0.00186253447212876\\
34487	0.00180081191736708\\
34488	0.00178676409274272\\
34491	0.001766477088592\\
34492	0.00174951377458621\\
34493	0.00160134850024059\\
34494	0.00167078423150238\\
34495	0.00155777952876341\\
34498	0.00173403272540094\\
34499	0.00165528846053153\\
34500	0.00163657721410555\\
34501	0.00173374894019714\\
34502	0.00180157128452323\\
34505	0.00199855808280302\\
34506	0.00211074122691738\\
34507	0.00208693412689975\\
34508	0.00213607767968809\\
34509	0.00224585464982475\\
34512	0.00229120057653056\\
34513	0.00199323970880165\\
34514	0.00179561405991102\\
34515	0.00182983581711012\\
34519	0.00179760841140182\\
34520	0.00181997679204862\\
34521	0.00181063416359144\\
34522	0.00179696800657318\\
34523	0.00181174060255727\\
34526	0.00179037396485007\\
34527	0.00180395496863292\\
34528	0.00172419715543688\\
34529	0.00159042292042056\\
34530	0.0014792380488629\\
34533	0.00141360966234667\\
34534	0.00139029384220306\\
34535	0.001444655804027\\
34536	0.00137665024409623\\
34537	0.00129422061870488\\
34540	0.00135131911247827\\
34541	0.00128174376432343\\
34542	0.00144483391254512\\
34543	0.00139578236319204\\
34544	0.00145549569192465\\
34548	0.00113078041052126\\
34549	0.00110771507354653\\
34550	0.0011531052987217\\
34551	0.00108099421010145\\
34554	0.00107561835345547\\
34555	0.00108756098509448\\
34556	0.00109316881774172\\
34557	0.00120927266248683\\
34558	0.00122201637120534\\
34561	0.00123260136177349\\
34562	0.00120566288806364\\
34563	0.000990745316210927\\
34564	0.00103502870486878\\
34565	0.000982328487140943\\
34568	0.0011259769699272\\
34569	0.0010923785918137\\
34570	0.000922108111862535\\
34571	0.0007383834890063\\
34572	0.000605721603423398\\
34576	0.000668770064078112\\
34577	0.000673425953890049\\
34578	0.000858918837098562\\
34579	0.00083238780554768\\
34582	0.0007323830424326\\
34583	0.00094363868372891\\
34584	0.000942426256833238\\
34585	0.0011141431804951\\
34586	0.00141755414285896\\
34589	0.00146390665115824\\
34590	0.00153552092554091\\
34591	0.00180090001634213\\
34592	0.00164602452760452\\
34593	0.00209286308078539\\
34596	0.00196444376116673\\
34597	0.00229233117542256\\
34598	0.00233385399805134\\
34599	0.00239685364761844\\
34600	0.00236944049030047\\
34603	0.00252192862152249\\
34604	0.00251066007855342\\
34605	0.00238080628794522\\
34606	0.00259108309283632\\
34607	0.00248530828468983\\
34610	0.00257801871030736\\
34611	0.00270334261810094\\
34612	0.00267072687495247\\
34613	0.00271308722208141\\
34614	0.00272221886005693\\
34617	0.00260229047702762\\
34618	0.00227122927900758\\
34619	0.00198409231576534\\
34620	0.00151830208341821\\
34621	0.00193332671635437\\
34624	0.00179457511292788\\
34625	0.00219258294680757\\
34626	0.00246637901233972\\
34627	0.00245468978138245\\
34628	0.00270358948260604\\
34631	0.00270617049553709\\
34632	0.00288074275626526\\
34633	0.00268621132567431\\
34634	0.00243191897057841\\
34635	0.00212212805572806\\
34638	0.00201765624607608\\
34639	0.0020361171526384\\
34640	0.00213996227187475\\
34641	0.00200135232430631\\
34642	0.00206150488365237\\
34645	0.00239243026370174\\
34646	0.00238366694865773\\
34647	0.00208704579566431\\
34648	0.00207311472142601\\
34649	0.00236600534700237\\
34652	0.00234799970565529\\
34653	0.00181735878727707\\
34654	0.00167457907678282\\
34655	0.0019347469457285\\
34656	0.00189852129662375\\
34659	0.00208076537505694\\
34660	0.00265529411339315\\
34661	0.00314669689425123\\
34662	0.00288202937417564\\
34663	0.00299307476560574\\
34666	0.00308887493747355\\
34667	0.00296845916029238\\
34668	0.00281904617786121\\
34669	0.00339667498685739\\
34670	0.00345661417007514\\
34673	0.00377632190455613\\
34674	0.00402837313541963\\
34675	0.00422512582032623\\
34676	0.00440832011384014\\
34677	0.00385010397042482\\
34680	0.00311285747618974\\
34681	0.00330473304388117\\
34682	0.00376874857327427\\
34683	0.00423165961132644\\
34684	0.00669787605176389\\
};
\addplot [color=mycolor4, forget plot]
  table[row sep=crcr]{%
34445	0.0014664661429671\\
34446	0.00137080592531048\\
34449	0.00146923167546148\\
34450	0.00141349246311125\\
34451	0.00134449444559554\\
34452	0.00142515074927492\\
34453	0.00144095555945961\\
34457	0.00152044410459928\\
34458	0.00161201538633701\\
34459	0.00148698090807758\\
34460	0.00148014799158602\\
34463	0.0015194941973704\\
34464	0.00140623968095157\\
34465	0.00138032243337226\\
34466	0.00135769521519679\\
34467	0.00142060953865811\\
34470	0.00145280416014509\\
34471	0.00145856214690592\\
34472	0.00148968589996292\\
34473	0.00147080600580093\\
34474	0.00147534975243491\\
34477	0.00154769814641596\\
34478	0.00158758756418659\\
34479	0.00174033488547087\\
34480	0.00155488712662055\\
34481	0.0015418441407369\\
34485	0.00153573917560068\\
34486	0.00150064289653143\\
34487	0.00148322451384725\\
34488	0.00142805689127671\\
34491	0.00144724925825863\\
34492	0.00134865536479806\\
34493	0.00133605232979923\\
34494	0.00134077089576705\\
34495	0.00131606401332741\\
34498	0.00135416937007926\\
34499	0.00136589858713998\\
34500	0.00136790292999139\\
34501	0.00138608808274996\\
34502	0.00138903895916085\\
34505	0.00137646838217413\\
34506	0.00137268660227728\\
34507	0.00139536511118082\\
34508	0.00139137226382116\\
34509	0.00130986070360515\\
34512	0.00117457226428268\\
34513	0.00116259265918929\\
34514	0.00118524578709528\\
34515	0.00116305256743241\\
34519	0.00119834283568291\\
34520	0.00119961465091542\\
34521	0.00119967714650271\\
34522	0.00120813893078828\\
34523	0.00122517157831017\\
34526	0.00124082891694786\\
34527	0.0012263299740715\\
34528	0.00123602137878212\\
34529	0.0012222960438329\\
34530	0.00120500059060608\\
34533	0.00118871571111272\\
34534	0.00118629964509607\\
34535	0.00120860050557131\\
34536	0.00119356719264431\\
34537	0.00118298378837768\\
34540	0.00120067676840382\\
34541	0.00118505411420401\\
34542	0.00123330915882277\\
34543	0.00124553089423054\\
34544	0.00124903684003859\\
34548	0.00116809285137329\\
34549	0.00113958454974269\\
34550	0.0011594236828137\\
34551	0.00113700822315351\\
34554	0.00114107184394532\\
34555	0.00114543623201923\\
34556	0.00115213298837315\\
34557	0.00120013830338917\\
34558	0.00120506180564792\\
34561	0.00127704557237904\\
34562	0.00126700569655436\\
34563	0.00120213144068505\\
34564	0.00122778236841058\\
34565	0.00124380739073988\\
34568	0.00122236785932719\\
34569	0.00115081213367438\\
34570	0.00113357298341038\\
34571	0.00106046775751097\\
34572	0.00100343704188714\\
34576	0.0010431801583075\\
34577	0.00104675009566786\\
34578	0.00113700375955732\\
34579	0.00112022839222823\\
34582	0.00109118778926668\\
34583	0.00118717033579689\\
34584	0.00119706997002263\\
34585	0.00127006813420599\\
34586	0.00135921337545174\\
34589	0.00136309471780032\\
34590	0.00139465676720584\\
34591	0.00167848742863255\\
34592	0.00167287993626514\\
34593	0.00172765974191469\\
34596	0.00172080761369033\\
34597	0.00171329675430121\\
34598	0.00168672769655204\\
34599	0.00165908587317082\\
34600	0.00168079060712894\\
34603	0.00167481545985298\\
34604	0.001692962474835\\
34605	0.00174523815602119\\
34606	0.00168761468903984\\
34607	0.00173589686744919\\
34610	0.00164966677608061\\
34611	0.00172018831039617\\
34612	0.00160644131754126\\
34613	0.00167330409442307\\
34614	0.00170584322309081\\
34617	0.00179622317227746\\
34618	0.00181283350738672\\
34619	0.00180924623763953\\
34620	0.00165999306199465\\
34621	0.00174970278816936\\
34624	0.00177778038602541\\
34625	0.00183672044288441\\
34626	0.00183594395216727\\
34627	0.00184326582371953\\
34628	0.00181717882919278\\
34631	0.00179034827321911\\
34632	0.00172944666040964\\
34633	0.00171139341949001\\
34634	0.00177924790706739\\
34635	0.00183654779374535\\
34638	0.00180251396980015\\
34639	0.00173288679026686\\
34640	0.00175579087847857\\
34641	0.00176548004240177\\
34642	0.00179277385796237\\
34645	0.00198287208571425\\
34646	0.00208176551352193\\
34647	0.00209886845349165\\
34648	0.00205811825663287\\
34649	0.00208480706554795\\
34652	0.00220947806400572\\
34653	0.00217265681799365\\
34654	0.00203479628254386\\
34655	0.00213171754555359\\
34656	0.00214840067653345\\
34659	0.00229496347400622\\
34660	0.00235854969225703\\
34661	0.00216359924043745\\
34662	0.0020725285059347\\
34663	0.0020962500445234\\
34666	0.00223969844392954\\
34667	0.00229913733317906\\
34668	0.00240414054501988\\
34669	0.00229451069332262\\
34670	0.00219276729711869\\
34673	0.00223982335462584\\
34674	0.00213719729321437\\
34675	0.00216919527150575\\
34676	0.00221654837080733\\
34677	0.00175926588255419\\
34680	0.00106057100253159\\
34681	0.000935596100485703\\
34682	0.00125588326101123\\
34683	0.000667077815209096\\
34684	0.000725932891798761\\
};
\addplot [color=mycolor4, forget plot]
  table[row sep=crcr]{%
34445	0.00147944902723475\\
34446	0.00145485520228568\\
34449	0.0014882988614743\\
34450	0.0014571765342377\\
34451	0.00143268783775723\\
34452	0.00147164804089028\\
34453	0.0014712141756369\\
34457	0.00149082602606445\\
34458	0.00147765210530182\\
34459	0.00146993876308279\\
34460	0.00146869722770425\\
34463	0.00148879914292051\\
34464	0.00146553598266628\\
34465	0.00145017631943904\\
34466	0.00144682632303327\\
34467	0.00146984390401109\\
34470	0.00148297143802797\\
34471	0.00157030455304023\\
34472	0.00158231753865011\\
34473	0.00158851558941442\\
34474	0.00159131235163699\\
34477	0.00148760780697614\\
34478	0.00143782689268395\\
34479	0.00139281653278832\\
34480	0.00142308263854645\\
34481	0.00122353090741398\\
34485	0.00117894165406345\\
34486	0.00112896814738168\\
34487	0.001184524855199\\
34488	0.00109066539384309\\
34491	0.00110491437590485\\
34492	0.0010756906055638\\
34493	0.00108703497081693\\
34494	0.00109267813742264\\
34495	0.00110119617934326\\
34498	0.00109932677994238\\
34499	0.00111117205935642\\
34500	0.00113417258009667\\
34501	0.0011329084363706\\
34502	0.00113193296619918\\
34505	0.00110240267011164\\
34506	0.00106946105853266\\
34507	0.00109657644959336\\
34508	0.00108126780158616\\
34509	0.000980996810360927\\
34512	0.000998090691981964\\
34513	0.000992125632880804\\
34514	0.0010348794397452\\
34515	0.000996874622062478\\
34519	0.0010641549562834\\
34520	0.00105735892402398\\
34521	0.00104414465093108\\
34522	0.00105927032101804\\
34523	0.00106322806825418\\
34526	0.00109453163468855\\
34527	0.00107287285496373\\
34528	0.00111201362188116\\
34529	0.00110773955295266\\
34530	0.00110869377393037\\
34533	0.001082761923016\\
34534	0.00103571380543656\\
34535	0.00104338339537105\\
34536	0.00105021241114727\\
34537	0.00103691718390617\\
34540	0.00102839420695106\\
34541	0.00102727128114162\\
34542	0.00103561164719264\\
34543	0.00102967338227407\\
34544	0.00103389117952096\\
34548	0.00102905836963048\\
34549	0.00100987919926093\\
34550	0.0010183422564555\\
34551	0.00101359083700052\\
34554	0.00102428157903787\\
34555	0.00102691647144921\\
34556	0.00105851699679198\\
34557	0.00108839601995556\\
34558	0.00109710103763065\\
34561	0.00108125045417038\\
34562	0.00112899075980698\\
34563	0.00115787671867869\\
34564	0.00116527078981791\\
34565	0.00117700722813098\\
34568	0.0012998827067476\\
34569	0.00129603191785827\\
34570	0.00130496904595056\\
34571	0.00126468839822164\\
34572	0.00121101169440488\\
34576	0.00125165338094793\\
34577	0.00124425901215845\\
34578	0.00130192491421273\\
34579	0.0012952632065257\\
34582	0.001284003866516\\
34583	0.00133686373975968\\
34584	0.0013465518417615\\
34585	0.00138322152021548\\
34586	0.0013770741474734\\
34589	0.00135311325202805\\
34590	0.00136460350307215\\
34591	0.00140453325836415\\
34592	0.00147787872075097\\
34593	0.00143315568748205\\
34596	0.00148309831591002\\
34597	0.00138760701332424\\
34598	0.00133913020508309\\
34599	0.00129485859090783\\
34600	0.00132271361149919\\
34603	0.00126637959103211\\
34604	0.00124340607024161\\
34605	0.00131464329826637\\
34606	0.00121783652483275\\
34607	0.00130205534778349\\
34610	0.00119559669891377\\
34611	0.00124622872164245\\
34612	0.00111628700036654\\
34613	0.00119335752006316\\
34614	0.00123073495837679\\
34617	0.00133505066522969\\
34618	0.00144425906652832\\
34619	0.00149397500583899\\
34620	0.0014460606026586\\
34621	0.00140389798262557\\
34624	0.001440349649249\\
34625	0.00140213486901925\\
34626	0.00133154019946221\\
34627	0.00134383333046711\\
34628	0.00126900112580323\\
34631	0.00129030407853019\\
34632	0.00121180437547586\\
34633	0.00118253699156165\\
34634	0.00123233269719819\\
34635	0.00132831148536042\\
34638	0.00138666117416094\\
34639	0.00134535186098856\\
34640	0.00134083551847329\\
34641	0.00137155098788741\\
34642	0.00135481171274644\\
34645	0.00132304274135663\\
34646	0.0012627274002989\\
34647	0.0013250848561331\\
34648	0.00130933633792821\\
34649	0.00129477413118614\\
34652	0.00135149376358426\\
34653	0.00142324760619466\\
34654	0.0014184919639771\\
34655	0.00142286459746485\\
34656	0.00144745741595982\\
34659	0.00150449146512036\\
34660	0.00144923298166662\\
34661	0.00129541584766131\\
34662	0.00127810576616592\\
34663	0.00128193315316994\\
34666	0.00135599109520198\\
34667	0.00138659124904167\\
34668	0.00146056581402089\\
34669	0.00134632905108719\\
34670	0.00124789704077477\\
34673	0.00133813419052848\\
34674	0.00125765544225668\\
34675	0.00124309177176205\\
34676	0.00123359688766773\\
34677	0.000948189146900571\\
34680	0.000405551736825257\\
34681	0.000294833471590995\\
34682	0.00030475817986781\\
34683	8.31814634322441e-05\\
34684	1.88394848273527e-05\\
};
\addplot [color=mycolor4, forget plot]
  table[row sep=crcr]{%
34445	0.001250476517922\\
34446	0.00127629263030366\\
34449	0.00125914114999202\\
34450	0.00125673407838968\\
34451	0.00126763533412481\\
34452	0.00127592670337428\\
34453	0.00126834198058853\\
34457	0.00124941332233252\\
34458	0.00120565461906665\\
34459	0.00122585152141705\\
34460	0.00119251762431302\\
34463	0.00119751437603235\\
34464	0.0012163830308389\\
34465	0.00119791891003519\\
34466	0.00120504209069802\\
34467	0.00119829872600065\\
34470	0.00119078104065447\\
34471	0.00119928283301597\\
34472	0.00119760127305976\\
34473	0.00120647240065504\\
34474	0.00120898344871067\\
34477	0.0012053153055171\\
34478	0.00118533922051682\\
34479	0.00108448250536559\\
34480	0.00108946254188212\\
34481	0.000976110705868836\\
34485	0.000991925162184494\\
34486	0.000916106060234191\\
34487	0.00100229305244458\\
34488	0.00103957330606709\\
34491	0.000999066793025992\\
34492	0.000986760721645345\\
34493	0.00102297348485219\\
34494	0.00100338272104536\\
34495	0.00101753053520113\\
34498	0.000962957109786126\\
34499	0.000991191933658446\\
34500	0.000999227953784198\\
34501	0.000987817895634554\\
34502	0.000986791455121653\\
34505	0.00092763657457689\\
34506	0.000880917593216747\\
34507	0.000908613424736157\\
34508	0.000887093602193799\\
34509	0.000769217558792052\\
34512	0.000799634984559689\\
34513	0.000796287041551971\\
34514	0.00083595166290497\\
34515	0.000799727142541434\\
34519	0.000861473103515674\\
34520	0.000853211240132773\\
34521	0.000837975312598324\\
34522	0.000855165698304222\\
34523	0.000844903912098688\\
34526	0.000872404497163024\\
34527	0.000847536853272321\\
34528	0.000885311384370066\\
34529	0.000910137723827865\\
34530	0.000906321681798289\\
34533	0.000911869297971377\\
34534	0.000917397740143701\\
34535	0.000919467295417766\\
34536	0.0009335211690224\\
34537	0.000933742536929619\\
34540	0.000937528389752974\\
34541	0.000944607885539188\\
34542	0.000933496342658292\\
34543	0.00092968112687166\\
34544	0.000926321498816478\\
34548	0.000957942006945528\\
34549	0.00093796785655745\\
34550	0.000940054911710997\\
34551	0.000942465474102314\\
34554	0.00095292589399193\\
34555	0.00095360646998322\\
34556	0.000955762602958571\\
34557	0.00095881427506822\\
34558	0.000964650936264383\\
34561	0.000977500949245742\\
34562	0.000988878536018342\\
34563	0.000996235037618747\\
34564	0.000995231039208862\\
34565	0.00100032193446871\\
34568	0.00103000174196904\\
34569	0.00107596799453376\\
34570	0.00110170447730416\\
34571	0.00109535901537619\\
34572	0.00109926545264078\\
34576	0.00112978163231513\\
34577	0.00118497116718564\\
34578	0.0012160644753276\\
34579	0.00119954080122825\\
34582	0.00121105468645058\\
34583	0.00121606016008628\\
34584	0.00121602215965208\\
34585	0.00121253199895674\\
34586	0.00117343244422218\\
34589	0.00112846466229162\\
34590	0.00113509696202999\\
34591	0.00113272419297708\\
34592	0.00123331227842416\\
34593	0.00113766249699076\\
34596	0.00108691941205175\\
34597	0.00100823186637119\\
34598	0.000939895509544034\\
34599	0.000888356389650508\\
34600	0.000905012428919306\\
34603	0.000880445677172898\\
34604	0.000874246515793232\\
34605	0.000915493758285693\\
34606	0.000862913165810031\\
34607	0.000900972553172019\\
34610	0.000848944645517798\\
34611	0.000875865339912762\\
34612	0.000811839786555299\\
34613	0.000852878192254813\\
34614	0.000860982447755585\\
34617	0.000908718333378948\\
34618	0.000943735082425293\\
34619	0.000952690548129113\\
34620	0.000966655776566434\\
34621	0.000955199577287378\\
34624	0.000982527169646663\\
34625	0.000957053671092205\\
34626	0.000942466118259952\\
34627	0.000914010690647853\\
34628	0.00087658518651334\\
34631	0.000842560452499238\\
34632	0.000809187902065834\\
34633	0.000730003268923219\\
34634	0.000737043882342586\\
34635	0.000658171014642472\\
34638	0.000653193919874524\\
34639	0.000660322209776759\\
34640	0.000665047650023936\\
34641	0.000673972164673912\\
34642	0.000677869307210694\\
34645	0.000694376069608595\\
34646	0.000678566916128235\\
34647	0.000689526507717905\\
34648	0.000681743232378353\\
34649	0.000685959493145417\\
34652	0.000711799878384036\\
34653	0.000720788849307373\\
34654	0.000663759867513847\\
34655	0.000673250507291987\\
34656	0.000683803838572963\\
34659	0.000717066047705433\\
34660	0.000721974261177416\\
34661	0.000699438128960687\\
34662	0.000713791241235444\\
34663	0.00072504733127107\\
34666	0.000769877657983645\\
34667	0.000792552574967279\\
34668	0.000813507427074432\\
34669	0.000802702194868289\\
34670	0.0008027985185169\\
34673	0.00086235039368764\\
34674	0.000850937907644403\\
34675	0.000859111090976685\\
34676	0.000864816513323596\\
34677	0.000770829340179992\\
34680	0.000519968967572016\\
34681	0.000435846456444544\\
34682	0.000417277024626354\\
34683	0.000192356120798734\\
34684	5.28056992339136e-05\\
};
\addplot [color=mycolor4, forget plot]
  table[row sep=crcr]{%
34445	0.000738475464779621\\
34446	0.000671461245062216\\
34449	0.000736529588949852\\
34450	0.000698124292887453\\
34451	0.000648271526662498\\
34452	0.000689968207081043\\
34453	0.000701721866905348\\
34457	0.000761879013136527\\
34458	0.000833974138887625\\
34459	0.000751227688005165\\
34460	0.000748885247828943\\
34463	0.000771135156474125\\
34464	0.000695250601556322\\
34465	0.000705356738509016\\
34466	0.000685730209988158\\
34467	0.000730965748086874\\
34470	0.000750447628634842\\
34471	0.000726850218309702\\
34472	0.000746699560297834\\
34473	0.000730552978874803\\
34474	0.000725006661975253\\
34477	0.000784668118641539\\
34478	0.000837427658004283\\
34479	0.00107110358526564\\
34480	0.00104433622414382\\
34481	0.00121680398080197\\
34485	0.0012086030526858\\
34486	0.0013064238797696\\
34487	0.00117696277666867\\
34488	0.00111263249409782\\
34491	0.00111113208382648\\
34492	0.00112957600145988\\
34493	0.00104111317356667\\
34494	0.0010940855585515\\
34495	0.00100870961501391\\
34498	0.00113719288598524\\
34499	0.00106289648717525\\
34500	0.00104659868286826\\
34501	0.0011123382473304\\
34502	0.00114124104872212\\
34505	0.00132688302205804\\
34506	0.00145982511590914\\
34507	0.00139409488546163\\
34508	0.00146051813910643\\
34509	0.00181414835787175\\
34512	0.00173465135852303\\
34513	0.00170545800245559\\
34514	0.00157691503454805\\
34515	0.00171013830764745\\
34519	0.00146325475421046\\
34520	0.00150696082999754\\
34521	0.00160310890600074\\
34522	0.00152206265186347\\
34523	0.00159638708542367\\
34526	0.00150277049816227\\
34527	0.00160603090700168\\
34528	0.00139340523618825\\
34529	0.00115083403927258\\
34530	0.00099823027719849\\
34533	0.000976789145148891\\
34534	0.000942502482147506\\
34535	0.000997143997322278\\
34536	0.000914145351889308\\
34537	0.000837414114862499\\
34540	0.000895155124819296\\
34541	0.000826820582314611\\
34542	0.000984732236601772\\
34543	0.000940136085922207\\
34544	0.000996990612435233\\
34548	0.00069331450680075\\
34549	0.00070218890513711\\
34550	0.000735684889877623\\
34551	0.000678844487826465\\
34554	0.000668441465877042\\
34555	0.00068050265278859\\
34556	0.000683825906432052\\
34557	0.000765926921791635\\
34558	0.000775493464637645\\
34561	0.000777921508913388\\
34562	0.000767944579539048\\
34563	0.000622529520018129\\
34564	0.000651841525643291\\
34565	0.00061274159247137\\
34568	0.000689460924114659\\
34569	0.000673280698386483\\
34570	0.000572723492121044\\
34571	0.000476641259954475\\
34572	0.0004020505166085\\
34576	0.000446379107274953\\
34577	0.000446871458446291\\
34578	0.000549014124563692\\
34579	0.000551652118182972\\
34582	0.000494979759157318\\
34583	0.000612102087640854\\
34584	0.000609033828674456\\
34585	0.000698211575023324\\
34586	0.00086516293859872\\
34589	0.00089613891221796\\
34590	0.000932586855504038\\
34591	0.0010934045711549\\
34592	0.000955859602396917\\
34593	0.00114300117573797\\
34596	0.00110299133543315\\
34597	0.00127797885504309\\
34598	0.00134721897390836\\
34599	0.00133855465450644\\
34600	0.00131775833045291\\
34603	0.00143784071732779\\
34604	0.00141707303068193\\
34605	0.0013046153550646\\
34606	0.001493893718139\\
34607	0.0013952114172256\\
34610	0.00151915650390454\\
34611	0.00146479110299128\\
34612	0.00161342887581286\\
34613	0.00151057155253292\\
34614	0.00150330235018136\\
34617	0.00139429280426213\\
34618	0.00121827729261377\\
34619	0.00109571834359334\\
34620	0.000884829525396947\\
34621	0.00105402256190705\\
34624	0.000994522520582866\\
34625	0.00115741376345506\\
34626	0.00128248799146809\\
34627	0.00127157272594412\\
34628	0.00141308331745695\\
34631	0.00141636832820078\\
34632	0.00157700862712081\\
34633	0.00141269266708821\\
34634	0.00131090215882387\\
34635	0.00112247923893191\\
34638	0.00109915453704611\\
34639	0.0011153477848736\\
34640	0.00116505721491439\\
34641	0.00109563348254809\\
34642	0.0011228235873341\\
34645	0.00127480532043952\\
34646	0.00130160515209836\\
34647	0.00113428968383038\\
34648	0.00112175006076135\\
34649	0.00126580321633764\\
34652	0.00123547645374596\\
34653	0.000991412559404824\\
34654	0.000941955285160104\\
34655	0.0010401754281243\\
34656	0.00102461989365872\\
34659	0.00108532477713706\\
34660	0.00132345870506313\\
34661	0.00163110329150969\\
34662	0.0015325669419049\\
34663	0.00158942602712426\\
34666	0.00157784248983115\\
34667	0.00148543880128083\\
34668	0.00138041847773183\\
34669	0.0016505348751882\\
34670	0.00181218159977846\\
34673	0.00184918728068793\\
34674	0.00206484048337029\\
34675	0.00214993680680081\\
34676	0.00221615581032321\\
34677	0.00268120047059913\\
34680	0.00371000095729721\\
34681	0.00402077355443789\\
34682	0.00370884246060446\\
34683	0.00433221359036551\\
34684	0.00257754422642157\\
};
\addplot [color=mycolor4, forget plot]
  table[row sep=crcr]{%
34445	0.000983891826164664\\
34446	0.000918716214306283\\
34449	0.000998583283007464\\
34450	0.000953969970193818\\
34451	0.000894648815113024\\
34452	0.000945227800761344\\
34453	0.000956371679409577\\
34457	0.0010286529906666\\
34458	0.0011074740280044\\
34459	0.0010137514064417\\
34460	0.00100223483367839\\
34463	0.00103225398326958\\
34464	0.000949011304276731\\
34465	0.000954449645677727\\
34466	0.000937369077796316\\
34467	0.000986271449413554\\
34470	0.00100847816592612\\
34471	0.00100705394357711\\
34472	0.00103077571055864\\
34473	0.00101796595473373\\
34474	0.00101823086264663\\
34477	0.00108480916861824\\
34478	0.00113990568874388\\
34479	0.00134278916975904\\
34480	0.0013919111153214\\
34481	0.0014557900346165\\
34485	0.00144232096815463\\
34486	0.00147357014832191\\
34487	0.0014331786084879\\
34488	0.00145704805668305\\
34491	0.00144658697834852\\
34492	0.00144814410469071\\
34493	0.00137096842665313\\
34494	0.00141060819975992\\
34495	0.00134732091969056\\
34498	0.00146178402022598\\
34499	0.00142603998111739\\
34500	0.00141802443140943\\
34501	0.00146176205113572\\
34502	0.00148379617440171\\
34505	0.00160527062171144\\
34506	0.00167844726548974\\
34507	0.0016460130267234\\
34508	0.0016913611000493\\
34509	0.00187767049014311\\
34512	0.0018950030234442\\
34513	0.00190156051161819\\
34514	0.00186507316968726\\
34515	0.00188928689934978\\
34519	0.00186259167830038\\
34520	0.00188627023693572\\
34521	0.00195912079933521\\
34522	0.00193409509658883\\
34523	0.00192272353412083\\
34526	0.00191069508193577\\
34527	0.0019494925004875\\
34528	0.00184810280937233\\
34529	0.00167839336280939\\
34530	0.00147338604363841\\
34533	0.00145800263660877\\
34534	0.0014366791382549\\
34535	0.00147113162312535\\
34536	0.00139674512115047\\
34537	0.00131613840770383\\
34540	0.00136792769165649\\
34541	0.00129577984664858\\
34542	0.00146601887931168\\
34543	0.00142691101971759\\
34544	0.00149273039281053\\
34548	0.00113889578707896\\
34549	0.00110885423967752\\
34550	0.0011528368885213\\
34551	0.00107958795676616\\
34554	0.00107462321954141\\
34555	0.00108476453962367\\
34556	0.00108983910329193\\
34557	0.00119372859206269\\
34558	0.00120729684076293\\
34561	0.00120560271873632\\
34562	0.00119065567909951\\
34563	0.00100118194183866\\
34564	0.00103728579906373\\
34565	0.000995850959940213\\
34568	0.00110334374887872\\
34569	0.00107982113654413\\
34570	0.000955154452261919\\
34571	0.000815682664865608\\
34572	0.000706317022445593\\
34576	0.000767849806584984\\
34577	0.000770514220719005\\
34578	0.000914543966307527\\
34579	0.000892940682837631\\
34582	0.000823777693715712\\
34583	0.000976943062408327\\
34584	0.000975461097096604\\
34585	0.00109655848111171\\
34586	0.00127223992233654\\
34589	0.00129999674488987\\
34590	0.00134402386101696\\
34591	0.00149067897063019\\
34592	0.00137545077176174\\
34593	0.00154753211268665\\
34596	0.00153219875644502\\
34597	0.00167952494458587\\
34598	0.00170585274818203\\
34599	0.00171037898993351\\
34600	0.0017177489134847\\
34603	0.001811213887764\\
34604	0.00180715422162992\\
34605	0.00174887479515217\\
34606	0.00185757701559557\\
34607	0.00181467953906783\\
34610	0.00195751957631065\\
34611	0.00194329033797194\\
34612	0.0019813174063709\\
34613	0.00197217083589923\\
34614	0.00196930070235885\\
34617	0.00195081748458462\\
34618	0.00174387236777804\\
34619	0.00161052389876543\\
34620	0.0013904687099266\\
34621	0.00157433648415956\\
34624	0.00152362042242645\\
34625	0.00171079333183251\\
34626	0.0018506459494286\\
34627	0.00184366902984747\\
34628	0.00196987527893505\\
34631	0.00199541405143684\\
34632	0.00210330652545551\\
34633	0.00212251575738415\\
34634	0.00210703968126453\\
34635	0.00188904907524281\\
34638	0.00188563934256751\\
34639	0.00190827632301847\\
34640	0.0020005402334\\
34641	0.00192211667157225\\
34642	0.00196357823092998\\
34645	0.00228459106803421\\
34646	0.00225354491203606\\
34647	0.00200810666271996\\
34648	0.00197550740553504\\
34649	0.00222356082689892\\
34652	0.00226898437572831\\
34653	0.0017975077989653\\
34654	0.00164580030637758\\
34655	0.00187022833967698\\
34656	0.00183432129648131\\
34659	0.00200542613280181\\
34660	0.00246124656959755\\
34661	0.00276366698397265\\
34662	0.00259594841850362\\
34663	0.00268640810198188\\
34666	0.00281089802298828\\
34667	0.00269607276623266\\
34668	0.00259501939862392\\
34669	0.00300725335647953\\
34670	0.00315287841328672\\
34673	0.00325333119626853\\
34674	0.00347338758249212\\
34675	0.00364166572284267\\
34676	0.00379830356433216\\
34677	0.0036158704958071\\
34680	0.00312448316598933\\
34681	0.00330464374377339\\
34682	0.00503928814249499\\
34683	0.00499264367598505\\
34684	0.00979457822350064\\
};
\addplot [color=mycolor4, forget plot]
  table[row sep=crcr]{%
34445	0.00110427857700139\\
34446	0.00106244612697476\\
34449	0.00115995392308162\\
34450	0.00114128058370652\\
34451	0.00121721406743847\\
34452	0.00136130490818714\\
34453	0.00137201545257076\\
34457	0.00146120141615611\\
34458	0.00153961277228243\\
34459	0.00145033911865092\\
34460	0.00146138973688408\\
34463	0.00150185731780935\\
34464	0.00138816908377566\\
34465	0.00139374151404131\\
34466	0.00137867516353486\\
34467	0.00143561564281455\\
34470	0.00145664042106289\\
34471	0.00143889429145467\\
34472	0.00146854287283759\\
34473	0.00145006355392128\\
34474	0.00144042770056835\\
34477	0.00152028073079175\\
34478	0.00157281220860799\\
34479	0.00173104647258477\\
34480	0.00161218869381763\\
34481	0.00163367870792176\\
34485	0.00159734211354315\\
34486	0.00155222724442896\\
34487	0.00161547983175382\\
34488	0.00166184989445882\\
34491	0.00170019842650472\\
34492	0.00170133280453354\\
34493	0.00169274202195886\\
34494	0.00171901862283125\\
34495	0.0017052234254315\\
34498	0.00175150593548241\\
34499	0.00176631662938366\\
34500	0.00177104295178463\\
34501	0.00179455433949157\\
34502	0.00180433803710016\\
34505	0.00179636383321019\\
34506	0.00172767981770909\\
34507	0.00176049659953622\\
34508	0.00175323190002203\\
34509	0.00157019220984951\\
34512	0.00167989054950501\\
34513	0.00173850860777602\\
34514	0.00187156352769787\\
34515	0.00175056963777482\\
34519	0.00197760001666995\\
34520	0.00195430912246857\\
34521	0.00193447576217918\\
34522	0.00199460531456983\\
34523	0.00199849453078034\\
34526	0.00210250786437449\\
34527	0.00202953468825896\\
34528	0.00214763374946695\\
34529	0.00212257487273463\\
34530	0.00201135197302641\\
34533	0.00200648945878743\\
34534	0.00200235669684898\\
34535	0.00206685559503071\\
34536	0.00202566329374067\\
34537	0.00192071676856925\\
34540	0.00197478386945338\\
34541	0.00192248346615304\\
34542	0.00205604578567864\\
34543	0.00205908848373286\\
34544	0.00208957651278017\\
34548	0.0018074173097213\\
34549	0.00171321937419212\\
34550	0.00162242194369251\\
34551	0.00144064679279291\\
34554	0.00144949424446115\\
34555	0.00145089559466031\\
34556	0.00145448273900812\\
34557	0.0015264491237761\\
34558	0.00153978986562784\\
34561	0.00163326969207573\\
34562	0.0015933402082956\\
34563	0.00147762502424601\\
34564	0.00151001283887494\\
34565	0.00147537760488255\\
34568	0.00159395105042747\\
34569	0.00157430981064538\\
34570	0.00145218160244522\\
34571	0.00129058208592457\\
34572	0.00114961476779576\\
34576	0.00122103061890614\\
34577	0.00122739153184927\\
34578	0.00140019347084102\\
34579	0.00135160998018571\\
34582	0.00128255113305225\\
34583	0.00144483166393983\\
34584	0.00144551644971706\\
34585	0.00158972885901798\\
34586	0.00174977889255999\\
34589	0.00170764598103396\\
34590	0.00174183350380504\\
34591	0.00182105789975925\\
34592	0.00177690673991554\\
34593	0.00184840152679318\\
34596	0.00187670754653456\\
34597	0.00188215509167939\\
34598	0.0018403424096881\\
34599	0.00184814684917687\\
34600	0.00187613701986599\\
34603	0.00185120438954793\\
34604	0.0018704388830656\\
34605	0.00194140880508334\\
34606	0.00185404857746304\\
34607	0.00194785059993328\\
34610	0.00185855705258585\\
34611	0.00193196344493458\\
34612	0.00174661084418\\
34613	0.0018579181656886\\
34614	0.00191306290828849\\
34617	0.00205461024063521\\
34618	0.00209591944649619\\
34619	0.00205449778982517\\
34620	0.00195606630941136\\
34621	0.00208546215069484\\
34624	0.00210672820696376\\
34625	0.00217196566525351\\
34626	0.00220292262978545\\
34627	0.00238040254727252\\
34628	0.00246915975175515\\
34631	0.00247183655417244\\
34632	0.00224912807567699\\
34633	0.00228764802356302\\
34634	0.00250402043908397\\
34635	0.0027448911501689\\
34638	0.00268866761017312\\
34639	0.00275759400055211\\
34640	0.00277356636439206\\
34641	0.00282006015120057\\
34642	0.00284439466736828\\
34645	0.00284222225295391\\
34646	0.00280270453179255\\
34647	0.00290060808896632\\
34648	0.00283491877954093\\
34649	0.00291819495030038\\
34652	0.00308163343306669\\
34653	0.00303021791473959\\
34654	0.00277420145450307\\
34655	0.00295688267487627\\
34656	0.00298939038024967\\
34659	0.00324788762906376\\
34660	0.00325193949679326\\
34661	0.00264130234779917\\
34662	0.00270843339002831\\
34663	0.00272053329007988\\
34666	0.00302508064822453\\
34667	0.00330459316379533\\
34668	0.00347696314385367\\
34669	0.0029230379476064\\
34670	0.00248287868080199\\
34673	0.00283521147174849\\
34674	0.00232623061045069\\
34675	0.00220228590151128\\
34676	0.00213083216518745\\
34677	0.000977913622353299\\
34680	0.000100223063135671\\
34681	5.26630399884607e-05\\
34682	0.000106599814844488\\
34683	8.63898904648014e-06\\
34684	5.36099585806162e-06\\
};
\addplot [color=mycolor4, forget plot]
  table[row sep=crcr]{%
34445	0.001504046535448\\
34446	0.00147179601568764\\
34449	0.00154811709231892\\
34450	0.00152951944734326\\
34451	0.00149768808754773\\
34452	0.00152715783927891\\
34453	0.00152686028653224\\
34457	0.00160675882212693\\
34458	0.00161189269382714\\
34459	0.00160334609579809\\
34460	0.0016122836135341\\
34463	0.00164029242996603\\
34464	0.00163124840240377\\
34465	0.00161436388889764\\
34466	0.00161110950136142\\
34467	0.00163241825208999\\
34470	0.00164436053536169\\
34471	0.00174274649354302\\
34472	0.00176036412307498\\
34473	0.00177776462831653\\
34474	0.00180165402042862\\
34477	0.00184274036350361\\
34478	0.00183684187633395\\
34479	0.00171979645949264\\
34480	0.00170526285831344\\
34481	0.00150877770250222\\
34485	0.00150855686687065\\
34486	0.00134022344054175\\
34487	0.00155405255583948\\
34488	0.00166063376707659\\
34491	0.00172178514201438\\
34492	0.00171623163285892\\
34493	0.00185280565984362\\
34494	0.00183212886749828\\
34495	0.00192641070511128\\
34498	0.00180978927362656\\
34499	0.00193324145203257\\
34500	0.0019707690198104\\
34501	0.00190267472351382\\
34502	0.00186541776636477\\
34505	0.00157690710468915\\
34506	0.00133260940974454\\
34507	0.00146228869582617\\
34508	0.00135497980046587\\
34509	0.000836376393181406\\
34512	0.0010107026310287\\
34513	0.00109834246168085\\
34514	0.00135053249870019\\
34515	0.00112640603011202\\
34519	0.00153107730595069\\
34520	0.00145841371539888\\
34521	0.00131538543323561\\
34522	0.00146691827888972\\
34523	0.00141505904234133\\
34526	0.00160240867329868\\
34527	0.00141887738822715\\
34528	0.00177191479339562\\
34529	0.0020904795029522\\
34530	0.00216884066291782\\
34533	0.00219478981614825\\
34534	0.00222370373365042\\
34535	0.00218300481105246\\
34536	0.00226466202020559\\
34537	0.00228537571104095\\
34540	0.00224228753963794\\
34541	0.00227465190851455\\
34542	0.00218220393525705\\
34543	0.00225139602048579\\
34544	0.00219794979667664\\
34548	0.00234314685978961\\
34549	0.00221960881890062\\
34550	0.00224773594108733\\
34551	0.00224044999686928\\
34554	0.002265541684121\\
34555	0.00224969094204396\\
34556	0.00224944485292294\\
34557	0.0023006800369954\\
34558	0.00228219691877255\\
34561	0.00249144344912572\\
34562	0.00244276128472761\\
34563	0.00249996878037273\\
34564	0.00251489090908297\\
34565	0.00247052608270704\\
34568	0.00258255748771576\\
34569	0.00257257298024229\\
34570	0.00248770576209255\\
34571	0.00224546645141922\\
34572	0.00203917831901303\\
34576	0.00210683428094046\\
34577	0.00210520427219196\\
34578	0.00234439794732128\\
34579	0.0022413067556238\\
34582	0.00217168694955255\\
34583	0.00236248182272275\\
34584	0.00227131397312285\\
34585	0.00237383236194226\\
34586	0.00234725985837477\\
34589	0.00224806917385046\\
34590	0.00223213483312379\\
34591	0.00209013669051339\\
34592	0.00224973769684103\\
34593	0.00175051081835801\\
34596	0.00181267130444166\\
34597	0.00152322532137072\\
34598	0.00143931668964412\\
34599	0.00145776622746831\\
34600	0.00149674143457085\\
34603	0.00138485276459713\\
34604	0.00141340588557645\\
34605	0.00155497987717077\\
34606	0.00134479577096002\\
34607	0.00148487185625138\\
34610	0.00128421526281707\\
34611	0.00137920512215524\\
34612	0.0011275703371653\\
34613	0.00128106834224541\\
34614	0.00135035402161282\\
34617	0.00154940433545085\\
34618	0.00177488508583359\\
34619	0.00186720251774485\\
34620	0.00199676070935194\\
34621	0.00186708885281879\\
34624	0.00194165434037171\\
34625	0.00181372931604477\\
34626	0.00169776195639699\\
34627	0.00171312904603519\\
34628	0.00152503494711121\\
34631	0.00154140943784876\\
34632	0.0012914545521396\\
34633	0.00131582673469943\\
34634	0.00150008315885245\\
34635	0.00189588969000444\\
34638	0.00197345151552579\\
34639	0.00196951932637832\\
34640	0.00190267676329338\\
34641	0.00204916757465626\\
34642	0.00202590537925285\\
34645	0.00179875607650975\\
34646	0.00174453318186604\\
34647	0.00207394846137311\\
34648	0.00209253547032905\\
34649	0.00185453626040513\\
34652	0.0020129976583122\\
34653	0.0024564515289263\\
34654	0.00242714670649964\\
34655	0.00232655298354518\\
34656	0.00236771541220279\\
34659	0.00236112906716498\\
34660	0.00190594175697704\\
34661	0.00119856769205504\\
34662	0.00133661508567447\\
34663	0.00123355329909334\\
34666	0.00133407626979473\\
34667	0.00153219083627829\\
34668	0.00180726199922779\\
34669	0.000888008421557315\\
34670	0.000592421834095042\\
34673	0.000434584776178848\\
34674	0.000201082927715203\\
34675	0.000134972424421832\\
34676	9.32831658044234e-05\\
34677	1.01892087047603e-05\\
34680	9.30491554344488e-09\\
34681	6.41788786544805e-10\\
34682	4.02630486004688e-10\\
34683	2.67621808359229e-14\\
34684	4.33096205591613e-19\\
};
\addplot [color=mycolor4, forget plot]
  table[row sep=crcr]{%
34445	0.00158183922910559\\
34446	0.00162252193020558\\
34449	0.00159677761720591\\
34450	0.00162229836505277\\
34451	0.00165634237333642\\
34452	0.00163876389586417\\
34453	0.00162378877914806\\
34457	0.00158384172994054\\
34458	0.00150892597641627\\
34459	0.0016070934754076\\
34460	0.00163150577108522\\
34463	0.00162805484441441\\
34464	0.00170675114803901\\
34465	0.00171497238249892\\
34466	0.00175129723082037\\
34467	0.00171185417232156\\
34470	0.0016977014517596\\
34471	0.00182399495997757\\
34472	0.00180863175122897\\
34473	0.00184973475543512\\
34474	0.00187602284210192\\
34477	0.00181816490902617\\
34478	0.00173187871180811\\
34479	0.0012960307026593\\
34480	0.0013276158322166\\
34481	0.000953273911863247\\
34485	0.000995218327225231\\
34486	0.000775879822821508\\
34487	0.00101774802191151\\
34488	0.00112962568551582\\
34491	0.00118973569722959\\
34492	0.00118485302363708\\
34493	0.00140366309589789\\
34494	0.00130428480771545\\
34495	0.00148561801390252\\
34498	0.00120229905187053\\
34499	0.00137898693259937\\
34500	0.00141912326769136\\
34501	0.00127366078659301\\
34502	0.00120019351969543\\
34505	0.000799651260182969\\
34506	0.000535501267125502\\
34507	0.00065269682447531\\
34508	0.000553334674577458\\
34509	0.000204252587809509\\
34512	0.000304356827859276\\
34513	0.00036675819433543\\
34514	0.000547171907615474\\
34515	0.000378084576271378\\
34519	0.000679415054946709\\
34520	0.000609154194097819\\
34521	0.000465705979445339\\
34522	0.000561002744738722\\
34523	0.000524429602139008\\
34526	0.000651469810225715\\
34527	0.000504572809296806\\
34528	0.000805001672356591\\
34529	0.00122434839683467\\
34530	0.0015241763392582\\
34533	0.00157316697547936\\
34534	0.00163886597909827\\
34535	0.00151743425669248\\
34536	0.00171222152653815\\
34537	0.00189865795590435\\
34540	0.00176336452941725\\
34541	0.00192471781265452\\
34542	0.00154825712243391\\
34543	0.00165612292224911\\
34544	0.00151632238920131\\
34548	0.00224305696165355\\
34549	0.00220298907658282\\
34550	0.00213429776224957\\
34551	0.00225270239646746\\
34554	0.00228224215168064\\
34555	0.00225329583696671\\
34556	0.00224487587029264\\
34557	0.00207380021013838\\
34558	0.00204270267523523\\
34561	0.00210020453549589\\
34562	0.00214509850807793\\
34563	0.00262382860644611\\
34564	0.00254021698337681\\
34565	0.00263395320630414\\
34568	0.00241273801671166\\
34569	0.00245772572914635\\
34570	0.00273832650695205\\
34571	0.00286881356491464\\
34572	0.00290407740242213\\
34576	0.00282684637717342\\
34577	0.00279954718459125\\
34578	0.00269842080821683\\
34579	0.00261983171734523\\
34582	0.00271264095086623\\
34583	0.00253350626949694\\
34584	0.00245780005619769\\
34585	0.00227623699632835\\
34586	0.00182923704415749\\
34589	0.00173657748543122\\
34590	0.00163761835111639\\
34591	0.0012005931770544\\
34592	0.00157015616550133\\
34593	0.0010891503859602\\
34596	0.00119299906875791\\
34597	0.000770418832521183\\
34598	0.000620182636458559\\
34599	0.000647275843508443\\
34600	0.000695132127288955\\
34603	0.000472589613186713\\
34604	0.000517050177905105\\
34605	0.000736298772994398\\
34606	0.000400709875431347\\
34607	0.00056271075799353\\
34610	0.000289452783629314\\
34611	0.000367051720608904\\
34612	0.000172742083799078\\
34613	0.000278107049486432\\
34614	0.000329565452726361\\
34617	0.000496287964341958\\
34618	0.000837281766713125\\
34619	0.001142916210722\\
34620	0.00173012154091534\\
34621	0.00120784607529905\\
34624	0.00139437443689623\\
34625	0.000936380917067516\\
34626	0.000643758167485587\\
34627	0.000660962276590947\\
34628	0.000401939538099427\\
34631	0.000386804467333461\\
34632	0.000189601525245702\\
34633	0.000197240720834276\\
34634	0.000301685527255455\\
34635	0.000758693163718159\\
34638	0.000894231594551983\\
34639	0.000812022297088377\\
34640	0.000670829610397789\\
34641	0.000862631510919665\\
34642	0.00078875396415828\\
34645	0.000404905914489357\\
34646	0.000383670474983917\\
34647	0.000734502526880559\\
34648	0.000750361326691523\\
34649	0.0004087430280664\\
34652	0.000405695907133153\\
34653	0.000964241584730986\\
34654	0.00128319536539308\\
34655	0.000877690499933054\\
34656	0.000906441395154932\\
34659	0.000629326566277857\\
34660	0.000260612633261122\\
34661	5.1268203971241e-05\\
34662	8.09653659323971e-05\\
34663	4.95752028315816e-05\\
34666	4.3803560862808e-05\\
34667	6.05776721701331e-05\\
34668	8.79676473381457e-05\\
34669	1.06962275074667e-05\\
34670	3.3322195880584e-06\\
34673	1.43386009533287e-06\\
34674	1.73602099667808e-07\\
34675	4.69457866892119e-08\\
34676	1.28799102184492e-08\\
34677	5.69989019444358e-11\\
34680	5.21264275047097e-19\\
34681	4.68561226965095e-22\\
34682	4.38893931106089e-24\\
34683	2.70315812925921e-35\\
34684	1.38294515684673e-54\\
};
\end{axis}
\end{tikzpicture}% }
	\caption{Black-Scholes Parameters}
	\label{fig:q2-bls-parameters}
	\vspace{-1cm}
\end{center}
\end{wrapfigure}

Then I calculated the absolute error between the Black-Scholes option price and the actual traded price. They were plotted in Figure \ref{fig:q2-error-both}. Finally, using Matlab's Financial Toolbox, I calculated the greeks for each option over that period and plotted them in Figure \ref{fig:q2-bls-parameters}.\\

As can be seen from Figure \ref{fig:q2-prices}, the call option prices closely followed the Black-Scholes solution. One exception here is a spike which can also be seen in Figure \ref{fig:q2-error-both} as well where the absolute error was quite substantial compared to other call options.\\

On the other hand, there was a systematic difference between the Black-Scholes formula and actual prices for the put options, seen in both Figure \ref{fig:q2-prices} and \ref{fig:q2-error-both}. All put options were traded at a higher price than that derived from Black-Scholes formula. Thus the absolute errors were also quite large and dispersed.\\

In Figure \ref{fig:q2-bls-parameters}, one of the volatilities has a spike near the end. This is precisely the call option whose price spiked at the same time. This leads me to believe that the price is correlated to the volatility of the underlying asset. Vega, $\nu=\frac{\partial c}{\partial \sigma}$ demonstrates exactly this. Clearly, as time to maturity draws near, the effect steadily goes down to 0. Delta $\Delta=\frac{\partial c}{\partial S}$ is also depicted in the same Figure over the same time period. Note that the horizontal scale is the time from the beginning of observation to the maturity date.




























\begin{thebibliography}{9}
\bibitem{hull} 
J. C. Hull
\textit{Options, Futures and Other Derivatives}. 
Prentice Hall, 2009

\end{thebibliography}

\end{document}











