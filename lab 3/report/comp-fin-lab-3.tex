\documentclass[11pt, fleqn]{article}

\usepackage[letterpaper, margin=0.7in]{geometry}
\usepackage{graphicx}
\usepackage{subcaption}
\usepackage{cleveref}
\usepackage{listings}
\usepackage{multicol}
\usepackage{tikz}
\usepackage{bm}
\usepackage{minted}
\usetikzlibrary{arrows}
\usepackage{wrapfig}
\usepackage{amsmath}
\usepackage{amssymb}
\usepackage[section]{placeins}
\usepackage{pgfplots,wrapfig}  
\pgfplotsset{compat=newest} 
\pgfplotsset{plot coordinates/math parser=false} 
\newlength\figureheight 
\newlength\figurewidth 
\setlength{\parindent}{0pt}
\newcommand\tab[1][1cm]{\hspace*{#1}}
\usepackage{algorithm}
\usepackage[noend]{algpseudocode}

\let\oldReturn\Return
\renewcommand{\Return}{\State\oldReturn}

\raggedbottom

\begin{document}

\begin{center}

\Large{COMP6212 Assignment 3 : Shakib-Bin Hamid 25250094 sh3g12}

\end{center}

Black-Scholes arrived at a closed form solution of the derivative pricing problem under some strict conditions. Some of these conditions can be eased by creating a binomial lattice based on probabilities on returns. However, these are still parametric methods and sensitive to the underlying stochastic process for $S(t)$, the stock price over time. Misspecification of it will lead to systematic errors in the option's price calculation, i.e. the performance of the parametric models is closely tied with the ability to capture the dynamics of underlying asset or stock.\\

On the other hand, learning networks like RBF (Radial Basis Function), MLP (Multi Layer Perceptron) etc. 'learn' the underlying dynamics based on training data and target outputs. They can adapt to the structural changes to the data generating process. It is true that such methods are not needed if the option in question is very well understood, or a new option is made and cannot be captured as a combination of other options or if there is not enough training data. However, these are rare circumstances indeed.\\

In this exercise we are primarily concerned with RBF networks. It has been shown that RBFs have the best approximation property, i.e. there is always a choice for the parameters that is beter than any other possible choice - not shared by MLPs. The paper poses the following challange - " if option prices are truly determined by the Black-Scholes formula exactly, can learning networks like RBF 'learn' the formula?". In other words, if we generated a dataset of options prices given strike prices, time to maturity and stock price, and trained a RBF on it, then will it generate the same prices on unseen data as Black-Scholes' formula would? If it does, then we can use this non-parametric model in future, even without the strict assumptions in Black-Scholes.\\

I begin by creating the training dataset from Black-Scholes formula. For $T/4 + 1$ to $3T/4$ days', I calculated the historical volatility as the standard deviation of log returns from the previous $T/4$ days for each of the call options. I fixed interest rate, $r = 0.06$. Given the strike price and FTSE 100 index as the underlying asset price, I used Matlab's Financial Toolbox to calculate the Black-Scholes call option price for those days. I then normalised the index, $S$ and option price, $C$ by the strike price $X$. Time to maturity, $T-t$ was also calculated for the days. This is my training dataset. The final $T/4$ days' data was prepared similarly as the validation set to test how closely the RBF network will perform on unseen data.

\begin{thebibliography}{9}
\bibitem{Hutchinson} 
J. Hutchinson, A. Lo, and T. Poggio,
\textit{A nonparametric  approach  to  pricing  and  hedging  derivative
securities via learning networks}. 
The Journal of Finance, vol. 49, no. 3, pp. 851–889, 1994.

\end{thebibliography}

\end{document}











